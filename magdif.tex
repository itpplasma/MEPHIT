%% LyX 2.2.2 created this file.  For more info, see http://www.lyx.org/.
%% Do not edit unless you really know what you are doing.
\documentclass[british,english]{article}
\usepackage[T1]{fontenc}
\usepackage[latin9]{inputenc}
\usepackage{geometry}
\geometry{verbose,tmargin=2cm,bmargin=2cm,lmargin=2cm,rmargin=2cm}
\usepackage{amsmath}
\usepackage{esint}
\usepackage{babel}
\begin{document}
\selectlanguage{british}%
\global\long\def\tht{\vartheta}
\global\long\def\ph{\varphi}
\global\long\def\balpha{\boldsymbol{\alpha}}
\global\long\def\btheta{\boldsymbol{\theta}}
\global\long\def\bJ{\boldsymbol{J}}
\global\long\def\bGamma{\boldsymbol{\Gamma}}
\global\long\def\bOmega{\boldsymbol{\Omega}}
\global\long\def\d{\text{d}}
\global\long\def\t#1{\text{#1}}
\global\long\def\m{\text{m}}
\global\long\def\bm{\text{\textbf{m}}}
\global\long\def\k{\text{k}}
\global\long\def\i{\text{i}}

\selectlanguage{english}%

\title{Numerical solution of magnetic differential equations}

\maketitle
Magnetic differential equations arise from a number of problems in
plasma physics. We consider for example the magnetohydrodynamic equilibrium
\begin{align}
\nabla p & =\boldsymbol{J}\times\boldsymbol{B},
\end{align}
with pressure $p$, current $\boldsymbol{J}$ and magnetic field $\boldsymbol{B}$.
Scalar multiplication with $\boldsymbol{B}$ yields the homogenous
magnetic differential equation
\begin{align}
\boldsymbol{B}\cdot\nabla p & =0\,.
\end{align}
In the linear perturbation theory, a source term enters the right-hand
side with
\begin{align}
\boldsymbol{B}\cdot\nabla p & =q\,.
\end{align}
For an axisymmetric plasma in a tokamak, we reduce the dimensionality
by introducing cylindrical coordinates and an expansion in $\varphi$,
\begin{align}
p(R,Z,\ph) & =\sum_{n}p_{n}(R,Z)e^{in\ph}\,.
\end{align}
The remaining equation in the poloidal $RZ$ plane for each harmonic
are
\begin{align}
\boldsymbol{B}\cdot\nabla_{RZ}p_{n}+inB^{\varphi}p_{n} & =q_{n}\,.
\end{align}


\section*{Finite Difference Method}

or in components of coordinates $x^{k}$ with $k=1,2$ in the poloidal
plans,
\begin{align}
B^{k}\frac{\partial}{\partial x^{k}}p_{n}+inB^{\varphi}p_{n} & =q_{n}\,.
\end{align}
Generating $B$ from the stream function $\psi=A_{\ph}$ we obtain
\begin{align}
B^{1} & =-\frac{1}{R\sqrt{g_{p}}}\frac{\partial\psi}{\partial x^{2}}\\
B^{2} & =\frac{1}{R\sqrt{g_{p}}}\frac{\partial\psi}{\partial x^{1}}
\end{align}
Here, $g_{p}$ is the metric determinant of the 2D metric tensor in
the poloidal plane. Using $\psi$ as one coordinate $x^{1}$, and
the distance $s$ in the poloidal direction with $ds=\sqrt{dR^{2}+dZ^{2}}$
yields
\begin{align}
\frac{1}{R\sqrt{g_{p}}}\frac{\partial p_{n}}{\partial s}+inB^{\varphi}p_{n} & =q_{n}\,.
\end{align}
This is a one-dimensional problem along the poloidal $\boldsymbol{B}$
direction. Multplying by $-iR\sqrt{g_{p}}$ and using (TODO $\sqrt{g_{p}}$)
the central difference formula around the $k$-th point $p_{n}^{k}$,
\begin{align}
nR\sqrt{g_{p}}B^{\varphi k}p_{n}^{k}-ip_{H}^{\prime} & =-iR\sqrt{g_{p}}q_{n}^{k}\,
\end{align}
we use the finite difference scheme
\begin{align*}
p_{H}^{\prime} & =\frac{\Delta s^{k-1}}{\Delta s^{k}(\Delta s^{k}+\Delta s^{k-1})}p^{k+1}+\frac{\Delta s^{k}-\Delta s^{k-1}}{\Delta s^{k}\Delta s^{k-1}}p^{k}-\frac{\Delta s^{k}}{\Delta s^{k-1}(\Delta s^{k}+\Delta s^{k-1})}p^{k-1}
\end{align*}

We need to solve
\begin{align}
A\boldsymbol{p} & =\boldsymbol{q}
\end{align}
Vectors $\boldsymbol{p}$ and $\boldsymbol{q}$ contain values $p_{n}^{k}$
and $q_{n}^{k}$ at the nodes. With our choice of variables, we have
an orthogonal system with
\begin{align*}
\hat{g}_{P} & =\left(\begin{array}{cc}
g_{\psi\psi}\\
 & 1
\end{array}\right)
\end{align*}
This means that $\sqrt{g_{p}}=\sqrt{g_{\psi\psi}}=1/|\nabla\psi|$.

\section*{Analytical solution}

\begin{align*}
i(mB^{\vartheta}+nB^{\varphi})p_{mn} & =q_{mn}\\
p_{mn} & =\frac{q_{mn}}{i(mB^{\vartheta}+nB^{\varphi})}\,.
\end{align*}
We take circular flux surfaces in the large aspect ratio limit, such
that the scaling with $R$ vanishes and as coordinates minor $r$
and $\vartheta$.

In this case

\begin{align*}
B^{\tht} & =\frac{1}{R_{0}\sqrt{g_{P}}}\frac{\partial\psi}{\partial r}\,.
\end{align*}
We set $\psi=r^{2}/4$ so $\frac{\partial\psi}{\partial r}=|\nabla\psi|=r/2$.
Due to the circular flux surfaces, we have a orthogonal system and
$\sqrt{g_{P}}=r$, so $B^{\tht}=1/(2R_{0})$.

\section*{Finite Volume Method}

Now we solve the same problem using a FVM scheme. We write the conservative
form
\begin{align*}
\nabla\cdot(\boldsymbol{h}j_{\parallel n})+inh^{\varphi}j_{\parallel n} & =-\nabla\cdot\boldsymbol{j}_{\perp}^{\text{pol}}-inj_{\perp n}^{\varphi}\,.
\end{align*}
The divergence operator is defined via
\begin{align*}
\nabla\cdot\boldsymbol{u} & =\frac{1}{R\sqrt{g_{p}}}\frac{\partial}{\partial x^{k}}(R\sqrt{g_{p}}u^{k})\,.
\end{align*}
When working with $R,Z$ as coordinates in the poloidal plane, $\sqrt{g_{p}}=1$.
We multiply by $R$ to obtain
\begin{align*}
\frac{\partial}{\partial x^{k}}(Rh^{k}j_{\parallel n})+inRh^{\varphi}j_{\parallel n} & =-\frac{\partial}{\partial x^{k}}(Rj_{\perp n}^{k})-inRj_{\perp n}^{\varphi}\,.
\end{align*}
Integration over a triangle yields
\begin{align}
\oint\,Rj_{\parallel n}\boldsymbol{h}\cdot d\boldsymbol{\Gamma}+in\int Rh^{\varphi}j_{\parallel n}d\Omega & =-\oint\,R\boldsymbol{j}_{\perp n}^{\text{pol}}\cdot d\boldsymbol{\Gamma}-in\int Rh^{\varphi}j_{\perp n}d\Omega\label{eq:fvmj}
\end{align}
where scalar products with $d\boldsymbol{\Gamma}$ pointing towards
the outer normal vector of the edge are taken component-wise in $R$
and $Z$. We assume a field-aligned mesh with $\boldsymbol{h}$ parallel
to edge no.~3. The in- and outflux are over edges 1 and 2.

\section*{General Finite Volume Method}

Eq.~(\ref{eq:fvmj}) is of the form
\begin{align*}
\oint\,u\,\boldsymbol{h}\cdot d\boldsymbol{\Gamma}+in\int u\,h^{\varphi}d\Omega & =-\oint\,\boldsymbol{v}\cdot d\boldsymbol{\Gamma}-in\int v\,h^{\varphi}d\Omega
\end{align*}
with $u=Rj_{\parallel n}$, $\boldsymbol{v}=R\boldsymbol{j}_{\perp n}^{\text{pol}}$
and $v=j_{\perp n}$.

\section*{Old}

In harmonics in $\varphi$ this becomes
\begin{align}
\boldsymbol{B}\cdot\nabla_{RZ}p_{n}+inB^{\varphi}p_{n} & =s_{n}\,.
\end{align}
If we take no toroidicity and harmonic RHS term with poloidal harmonic
$m$ we obtain
\begin{align*}
i(mB^{\vartheta}+nB^{\varphi})p_{mn} & =s_{mn}\\
p_{mn} & =\frac{s_{mn}}{i(mB^{\vartheta}+nB^{\varphi})}\,.
\end{align*}
Without toroidicity:
\begin{align*}
\sqrt{g} & =r\\
B^{\vartheta} & =\frac{1}{r}\partial_{r}A_{\varphi}
\end{align*}
So for $A_{\varphi}=r^{2}/4$ we get $B^{\vartheta}=1/2$. Furthermore,
we choose $B^{\varphi}=1$. We have
\begin{align*}
R & =R_{0}+r\cos\vartheta\\
Z & =r\sin\vartheta\\
\\
\frac{\partial R}{\partial r} & =(R-R_{0})/r\\
\frac{\partial Z}{\partial r} & =Z/r\\
\\
\frac{\partial R}{\partial\vartheta} & =-z\\
\frac{\partial Z}{\partial\vartheta} & =R-R_{0}\\
\\
B^{R} & =\frac{\partial R}{\partial\vartheta}B^{\vartheta}=-z/2\\
B^{Z} & =\frac{\partial Z}{\partial\vartheta}B^{\vartheta}=(R-R_{0})/2
\end{align*}
In real and imaginary parts this is
\begin{align}
\boldsymbol{B}\cdot\nabla_{RZ}(\Re p_{n}+i\Im p_{n})+inB^{\varphi}(\Re p_{n}+i\Im p_{n}) & =(\Re s_{n}+i\Im s_{n})\,\\
\boldsymbol{B}\cdot\nabla_{RZ}\Re p_{n}-nB^{\varphi}\Im p_{n} & =\Re s_{n}\,\\
\boldsymbol{B}\cdot\nabla_{RZ}\Im p_{n}+nB^{\varphi}\Re p_{n} & =\Im s_{n}\,
\end{align}
Combining
\begin{align*}
\boldsymbol{B}\cdot\nabla_{RZ}(\boldsymbol{B}\cdot\nabla_{RZ}\Re p_{n})-nB^{\varphi}(\Im s_{n}-nB^{\varphi}\Re p_{n}) & =\boldsymbol{B}\cdot\nabla_{RZ}\cdot\Re s_{n}\,\\
\boldsymbol{B}\cdot\nabla_{RZ}(\boldsymbol{B}\cdot\nabla_{RZ}\Im p_{n})+nB^{\varphi}(\Re s_{n}+nB^{\varphi}\Im p_{n}) & =\boldsymbol{B}\cdot\nabla_{RZ}\cdot\Im s_{n}\,
\end{align*}
In Flat space:
\begin{align*}
B^{R}\partial_{R}(\boldsymbol{B}\cdot\nabla_{RZ}\Re p_{n})+B^{Z}\partial_{Z}(\boldsymbol{B}\cdot\nabla_{RZ}\Re p_{n}) & \,\\
=(B^{R}\partial_{R}+B^{Z}\partial_{Z})(B^{R}\partial_{R}+B^{Z}\partial_{Z})\Re p_{n}\\
=\left((B^{R})^{2}\partial_{R}^{2}+2B^{R}B^{Z}\partial_{R}\partial_{Z}+\left(B^{Z}\right)^{2}\partial_{Z}^{2}\right)\Re p_{n}
\end{align*}
This equation is parabolic and not, as such, suited for FEM.

New:
\begin{align*}
\boldsymbol{B}\cdot\nabla_{RZ}(\boldsymbol{B}\cdot\nabla_{RZ}\Re p_{n}) & =\nabla_{RZ}\cdot(\boldsymbol{B}(\boldsymbol{B}\cdot\nabla_{RZ}\Re p_{n}))\\
 & =\nabla_{RZ}\cdot(\boldsymbol{B}\nabla_{RZ}\cdot(\boldsymbol{B}\Re p_{n}))
\end{align*}
In real and imaginary parts this is
\begin{align*}
i(mB^{\vartheta}+nB^{\varphi})(\Re p_{mn}+i\Im p_{mn}) & =(\Re s_{mn}+i\Im s_{mn})\\
\Im p_{mn} & =-\Re s_{mn}/(mB^{\vartheta}+nB^{\varphi})\\
\Re p_{mn} & =\Im s_{mn}/(mB^{\vartheta}+nB^{\varphi})
\end{align*}
We have
\begin{align*}
s & =\sum_{n}s_{n}(\vartheta)e^{in\varphi}=\sum_{mn}s_{mn}e^{i(m\vartheta+n\varphi)}\\
\\
 & =\sum_{n}(\Re s_{n}+i\Im s_{n})(\cos n\varphi+i\sin n\varphi)\\
 & =\sum_{n}(\Re s_{n}\cos n\varphi-\Im s_{n}\sin n\varphi)+i(\Re s_{n}\sin n\varphi+\Im s_{n}\cos n\varphi)\\
\\
 & =\sum_{mn}(\Re s_{mn}+i\Im s_{mn})(\cos(m\vartheta+n\varphi)+i\sin(m\vartheta+n\varphi))\\
 & =\sum_{mn}(\Re s_{mn}\cos(m\vartheta+n\varphi)-\Im s_{mn}\sin(m\vartheta+n\varphi))\\
 & +i(\Re s_{mn}\sin(m\vartheta+n\varphi)+\Im s_{mn}\cos(m\vartheta+n\varphi))\\
\\
s_{n} & =s_{mn}e^{im\vartheta}=(\Re s_{mn}+i\Im s_{mn})(\cos m\vartheta+i\sin m\vartheta)\\
 & =\Re s_{mn}\cos m\vartheta-\Im s_{mn}\sin m\vartheta+i(\Re s_{mn}\sin m\vartheta+\Im s_{mn}\cos m\vartheta)
\end{align*}
Test:
\begin{align*}
s & =\Im s_{mn}(\cos(m\vartheta+n\varphi)-i\sin(m\vartheta+n\varphi))\\
s_{n} & =s_{mn}e^{im\vartheta}=\Im s_{mn}(-\sin m\vartheta+i\cos m\vartheta)\\
 & =\\
\\
\Re p_{mn} & =\Im s_{mn}/(mB^{\vartheta}+nB^{\varphi})\\
\\
p_{n} & =\Re p_{mn}(\cos m\vartheta+i\sin m\vartheta)
\end{align*}


\section*{Pseudotoroidal coordinates}

\begin{align*}
R & =R_{0}+r\cos\vartheta\\
Z & =r\sin\tht\\
\\
\boldsymbol{e}_{r} & =\frac{\partial R}{\partial r}\boldsymbol{e}_{R}+\frac{\partial Z}{\partial r}\boldsymbol{e}_{Z}\\
 & =\boldsymbol{e}_{R}\cos\vartheta+\boldsymbol{e}_{Z}\sin\tht\\
\\
\boldsymbol{e}_{\tht} & =\frac{\partial R}{\partial\tht}\boldsymbol{e}_{R}+\frac{\partial Z}{\partial\tht}\boldsymbol{e}_{Z}\\
 & =-\boldsymbol{e}_{R}r\sin\vartheta+\boldsymbol{e}_{Z}r\cos\tht
\end{align*}

\end{document}
