%% LyX 2.2.3 created this file.  For more info, see http://www.lyx.org/.
%% Do not edit unless you really know what you are doing.
\documentclass[12pt,british,english]{article}
\usepackage[T1]{fontenc}
\usepackage[latin9]{inputenc}
\usepackage{geometry}
\geometry{verbose,tmargin=2cm,bmargin=2cm,lmargin=2cm,rmargin=2cm}
\usepackage{amsmath}

\makeatletter
%%%%%%%%%%%%%%%%%%%%%%%%%%%%%% User specified LaTeX commands.
\usepackage{tikz}

\makeatother

\usepackage{babel}
\begin{document}
\selectlanguage{british}%
\global\long\def\tht{\vartheta}
\global\long\def\ph{\varphi}
\global\long\def\balpha{\boldsymbol{\alpha}}
\global\long\def\btheta{\boldsymbol{\theta}}
\global\long\def\bJ{\boldsymbol{J}}
\global\long\def\bGamma{\boldsymbol{\Gamma}}
\global\long\def\bOmega{\boldsymbol{\Omega}}
\global\long\def\d{\text{d}}
\global\long\def\t#1{\text{#1}}
\global\long\def\m{\text{m}}
\global\long\def\bm{\text{\textbf{m}}}
\global\long\def\k{\text{k}}
\global\long\def\i{\text{i}}
\global\long\def\v#1{\boldsymbol{#1}}
\selectlanguage{english}%

\title{Magnetic differential equations for stationary linear ideal MHD and
their numerical solution}

\title{\textemdash{} remaining input from Chris}
\maketitle

\subsection*{Coordinate conventions}

We use two different right-handed coordinate systems: Firstly cylindrical
coordinates $(R,\ph,Z)$ with $\ph$ running counterclockwise as seen
from above, so
\begin{align*}
x & =R\cos\ph,\quad y=R\sin\ph,\quad z=Z.
\end{align*}
Those coordinates are used in computations and for meshes. For derivations,
also symmetry flux coordinates $(\rho,\theta,\zeta)$ are used at
some point. Here, we use the convention of d'Haeseleer with radial
coordinate growing towards the outside of the torus. As a radial variable
we use $\rho=\sigma_{\mathrm{pol}}\psi$ with $\sigma_{\mathrm{pol}}=+1$
for counterclockwise poloidal field $\v B_{0}^{\mathrm{pol}}$ and
$\sigma_{\mathrm{pol}}=-1$ for clockwise field. The toroidal symmetry
angle is defined in the direction opposite to $\ph$ with
\begin{align*}
\zeta & =\frac{\pi}{2}-\ph.
\end{align*}
The flux angle $\theta$ has the same orientation as $\vartheta=\arctan\frac{Z}{R}$,
so the sign $\sigma_{\mathrm{pol}}=\mathrm{sgn}(B^{\tht})=\mathrm{sgn}(B^{\theta})$
is positive for counter-clockwise and negative for clockwise orientation
in the poloidal plane. The Jacobian of symmetry flux coordinates $(\rho,\theta,\zeta)=(\sigma_{\mathrm{pol}}\psi,\theta,\frac{\pi}{2}-\varphi)$
is
\begin{align}
\sqrt{g} & =\frac{1}{B_{0}^{\vartheta}}=\frac{q}{B_{0}^{\zeta}}\\
 & =\frac{q}{-B_{0}^{\varphi}}=\frac{qR^{2}}{-B_{0\varphi}}.
\end{align}
Note that in our configuration (ASDEX Upgrade) $\sigma_{\mathrm{pol}}=-1$.

\subsection*{Safety factor}

With the usual toroidal flux angle $\zeta=\pi/2-\ph$ (d'Haeseleer),
the toroidal flux $\psi_{\text{tor}}$ is given by
\begin{align}
\psi_{\text{tor}} & =\frac{1}{(2\pi)^{2}}\int\d V\,\boldsymbol{B}\cdot\nabla\zeta=-\frac{1}{(2\pi)^{2}}\int\d V\,\boldsymbol{B}\cdot\nabla\ph\nonumber \\
 & =-\frac{1}{2\pi}\int\d R\d Z\,RB^{\ph}=-\frac{1}{2\pi}\int\d R\d ZB_{(\ph)}.
\end{align}
The safety factor is then given by
\begin{align*}
q & =\frac{\psi_{\mathrm{tor}}^{\prime}(\rho)}{\psi_{\mathrm{pol}}^{\prime}(\rho)}=\sigma_{\mathrm{pol}}\frac{\d\psi_{\mathrm{tor}}}{\d\psi}\\
 & =\sigma_{\mathrm{pol}}\frac{1}{2\pi}\frac{\d}{\d\psi}\int\d R\d ZB_{(\ph)}.
\end{align*}
This quantity can be evaluated numerically by adding up $B_{(\ph)}$
inside the volume between two flux surfaces and dividing by the the
difference in $\psi$:
\begin{align*}
q & \approx\frac{\sigma_{\mathrm{pol}}}{2\pi\Delta\psi}\sum_{\triangle}B_{(\ph)}S_{\triangle}
\end{align*}
where the sum is taken over all triangles inside a triangle strip,
and $S_{\Delta}$ is the respective triangle surface area.

\subsection*{Generating a non-resonant test field}

We would like to have a completely non-resonant and only radial field
\begin{align*}
\delta\v B & =\v B_{n}e^{in\ph}
\end{align*}
 with $B_{n}^{\ph}=B_{n}^{\tht}=0$. As an ansatz we take as a radial
component of the field
\begin{equation}
B_{n}^{\psi}=\frac{B_{0\varphi}R_{0}}{qR^{2}}=-\frac{R_{0}}{\sqrt{g}}.
\end{equation}
Divergence-freeness of $\delta\boldsymbol{B}$ is now possible for
a field containing only the radial component $\delta B^{\rho}=\sigma_{\mathrm{pol}}\delta B^{\psi}$
with
\begin{align*}
\nabla\cdot\delta\v B & =\frac{1}{\sqrt{g}}\frac{\partial}{\partial x^{k}}\left(\sqrt{g}\delta B^{k}\right)\\
 & =\frac{1}{\sqrt{g}}\frac{\partial}{\partial x^{k}}\left(\sqrt{g}\delta B^{\rho}\right)\\
 & =\frac{1}{\sqrt{g}}\frac{\partial}{\partial x^{k}}\left(-\sigma_{\mathrm{pol}}R_{0}\right)=0.
\end{align*}
This means that
\begin{align*}
\v B_{n} & =(\v B_{n}\cdot\nabla\psi)\v e_{\psi}=B_{n}^{\psi}\v e_{\psi}.
\end{align*}
Fluxes through edges are
\begin{align*}
\Psi & \approx R\v B_{n}\cdot\v n\\
 & =RB_{n}^{\psi}(\v e_{\psi}\cdot\v n)\\
 & =RB_{n}^{\psi}n_{\psi}
\end{align*}
Fluxes through edges parallel to flux surfaces with $\v n_{3}=\sigma l_{3}\nabla\psi/|\nabla\psi|$
are
\begin{align*}
\Psi_{3} & \approx R_{3}B_{n}^{\psi}\v e_{\psi}\cdot\v n_{3}\\
 & =\sigma l_{3}R_{3}B_{n}^{\psi}/|\nabla\psi|\\
 & =R_{3}l_{3}\frac{B_{n}^{\psi}}{|\nabla\psi|^{2}}(\hat{\v n}_{3}\cdot\nabla\psi)\\
 & =R_{3}\frac{B_{n}^{\psi}}{|\nabla\psi|^{2}}(\v n_{3}\cdot\nabla\psi).
\end{align*}
TODO: other edges either by conservation law or by vector projection.

\subsection*{Reading input files from experiment}
\begin{itemize}
\item ASDEX Upgrade has two formats that are common: CLISTE and EFIT/EQDSK
(g-files)
\item Much data in /proj/plasma/RMP/DATA/ASDEX/ and /proj/plasma/RMP/DATA2017/ASDEX/
\end{itemize}

\subsection*{Reduced MHD (unfinished)}

We take
\[
B_{n\ph}=0
\]
so
\begin{align*}
\boldsymbol{B}_{n} & =\nabla\psi_{n}\times\nabla\ph.
\end{align*}
Amperes law becomes
\begin{align*}
\boldsymbol{B}_{n} & =\frac{1}{4\pi}\boldsymbol{J}_{n}.
\end{align*}

\end{document}
