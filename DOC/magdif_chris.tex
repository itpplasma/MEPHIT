%% LyX 2.2.3 created this file.  For more info, see http://www.lyx.org/.
%% Do not edit unless you really know what you are doing.
\documentclass[12pt,british,english]{article}
\usepackage[T1]{fontenc}
\usepackage[latin9]{inputenc}
\usepackage{geometry}
\geometry{verbose,tmargin=2cm,bmargin=2cm,lmargin=2cm,rmargin=2cm}
\usepackage{amsmath}

\makeatletter
%%%%%%%%%%%%%%%%%%%%%%%%%%%%%% User specified LaTeX commands.
\usepackage{tikz}

\makeatother

\usepackage{babel}
\begin{document}
\selectlanguage{british}%
\global\long\def\tht{\vartheta}
\global\long\def\ph{\varphi}
\global\long\def\balpha{\boldsymbol{\alpha}}
\global\long\def\btheta{\boldsymbol{\theta}}
\global\long\def\bJ{\boldsymbol{J}}
\global\long\def\bGamma{\boldsymbol{\Gamma}}
\global\long\def\bOmega{\boldsymbol{\Omega}}
\global\long\def\d{\text{d}}
\global\long\def\t#1{\text{#1}}
\global\long\def\m{\text{m}}
\global\long\def\bm{\text{\textbf{m}}}
\global\long\def\k{\text{k}}
\global\long\def\i{\text{i}}
\global\long\def\v#1{\boldsymbol{#1}}
\selectlanguage{english}%

\title{Magnetic differential equations for stationary linear ideal MHD and
their numerical solution}

\title{\textemdash{} remaining input from Chris}
\maketitle

\subsection*{Coordinate conventions}

We use two different right-handed coordinate systems: Firstly cylindrical
coordinates $(R,\ph,Z)$ with $\ph$ running counterclockwise as seen
from above, so
\begin{align*}
x & =R\cos\ph,\quad y=R\sin\ph,\quad z=Z.
\end{align*}
Those coordinates are used in computations and for meshes. For derivations,
also symmetry flux coordinates $(\rho,\theta,\zeta)$ are used at
some point. Here, we use the convention of d'Haeseleer with radial
coordinate growing towards the outside of the torus. We use $\rho=\sigma_{\mathrm{pol}}\psi$
with $\sigma_{\mathrm{pol}}=+1$ for counterclockwise poloidal field
$\v B_{0}^{\mathrm{pol}}$ and $\sigma_{\mathrm{pol}}=-1$ for clockwise
field. The toroidal symmetry angle is defined in the direction opposite
to $\ph$ with
\begin{align*}
\zeta & =\frac{\pi}{2}-\ph.
\end{align*}
The flux angle $\theta$ has the same orientation as $\vartheta=\arctan\frac{Z}{R}$,
so the sign $\sigma_{\mathrm{pol}}=\mathrm{sgn}(B^{\tht})=\mathrm{sgn}(B^{\theta})$,
so positive for counter-clockwise and negative for clockwise in the
poloidal plane.

\subsection*{Safety factor (TODO: check all signs with flux coordinates)}

With the usual toroidal flux angle $\zeta=\pi/2-\ph$ (d'Haeseleer),
the toroidal flux $\psi_{\text{tor}}$ is given by
\begin{align}
\psi_{\text{tor}} & =\frac{1}{(2\pi)^{2}}\int\d V\,\boldsymbol{B}\cdot\nabla\zeta=-\frac{1}{(2\pi)^{2}}\int\d V\,\boldsymbol{B}\cdot\nabla\ph\nonumber \\
 & =-\frac{1}{2\pi}\int\d R\d Z\,RB^{\ph}=-\frac{1}{2\pi}\int\d R\d ZB_{(\ph)}.
\end{align}
As a radial flux variable we use the (ribbon) poloidal flux $\psi_{\mathrm{pol}}=\psi$
which is the negative disk poloidal flux. The safety factor is then
given by
\begin{align*}
q & =\frac{\psi_{\mathrm{tor}}^{\prime}(\rho)}{\psi_{\mathrm{pol}}^{\prime}(\rho)}=\frac{\d\psi_{\mathrm{tor}}}{\d\psi_{\mathrm{pol}}}=\frac{\d\psi_{\mathrm{tor}}}{\d\psi}\\
 & =-\frac{1}{2\pi}\frac{\d}{\d\psi}\int\d R\d ZB_{(\ph)}.
\end{align*}
This quantity can be evaluated numerically by adding up $B_{(\ph)}$
inside the volume between two flux surfaces and dividing by the the
difference in $\psi$:
\begin{align*}
q & \approx-\frac{1}{2\pi\Delta\psi}\sum_{\triangle}B_{(\ph)}S_{\triangle}
\end{align*}
where the sum is taken over all triangles inside a triangle strip,
and $S_{\Delta}$ is the respective triangle surface area.

\subsection*{Generating a non-resonant test field (TODO: check all signs with
flux coordinates)}

For derivation we use symmetry flux coordinates $(\rho,\vartheta,\zeta)=(\sigma_{\mathrm{pol}}\psi,\vartheta,\frac{\pi}{2}-\varphi)$
with Jacobian
\begin{align}
\sqrt{g} & =\frac{1}{B_{0}^{\vartheta}}=\frac{q}{B_{0}^{\zeta}}\\
 & =\frac{q}{-B_{0}^{\varphi}}=\frac{qR^{2}}{-B_{0\varphi}}.
\end{align}
We would like to have a completely non-resonant field with $B_{n}^{\tht}=0$.
As an ansatz we take

\begin{equation}
B_{n}^{\zeta}=B_{n}^{\rho}F(\rho).
\end{equation}
and the normal component of the test field
\begin{equation}
B_{n}^{\psi}=\frac{B_{0\varphi}R_{0}}{R^{2}}.
\end{equation}
Thus
\begin{align*}
B_{n}^{\rho} & =-\frac{B_{0\varphi}R_{0}}{R^{2}}.
\end{align*}
Divergence-freeness of $\boldsymbol{B}_{n}$ yields
\begin{align*}
0 & =\frac{1}{\sqrt{g}}\frac{\partial}{\partial x^{k}}(\sqrt{g}B_{n}^{k})\\
0 & =\frac{\partial}{\partial\rho}(\sqrt{g}B_{n}^{\rho})+\frac{\partial}{\partial\rho}(\sqrt{g}B_{n}^{\tht})+\frac{\partial}{\partial\zeta}(\sqrt{g}B_{n}^{\zeta})\\
 & =\frac{\partial}{\partial\rho}(\sqrt{g}B_{n}^{\rho})+0-\frac{\partial}{\partial\ph}(\sqrt{g}B_{n}^{\zeta})\\
 & =\frac{\partial}{\partial\rho}(\sqrt{g}B_{n}^{\rho})-in\sqrt{g}B_{n}^{\rho}F(\rho)\\
 & =\frac{\partial}{\partial\rho}\left(+\frac{qR^{2}}{B_{0\varphi}}\frac{B_{0\varphi}R_{0}}{R^{2}}\right)+inF(\rho)\frac{qR^{2}}{B_{0\varphi}}\frac{B_{0\varphi}R_{0}}{R^{2}}\\
 & =R_{0}\left(\frac{\partial q}{\partial\rho}+inF(\rho)q\right).
\end{align*}
To fulfill this, we take
\begin{equation}
F(\rho)=-\frac{1}{inq}\frac{\partial q}{\partial\rho}=\frac{1}{inq}\frac{\partial q}{\partial\psi}.
\end{equation}
We obtain
\begin{align*}
B_{n}^{\psi}=-B_{n}^{\rho} & =\frac{B_{0\varphi}R_{0}}{R^{2}}\\
B_{n}^{\ph}=-B_{n}^{\zeta} & =-B_{n}^{\rho}F(\rho)=\frac{B_{0\varphi}R_{0}}{R^{2}}\frac{1}{inq}\frac{\partial q}{\partial\psi}
\end{align*}

TODO CONTINUE FROM HERE: With known $B_{n}^{\psi}$ and $B_{n}^{\ph}$
we now proceed to find fluxes through triangle edges. Divergence-freeness
in cylindrical coordinates gives
\begin{align}
\int_{1,2}\d l\,R\boldsymbol{B}_{n}^{\text{pol}}\cdot\boldsymbol{n} & =-\int_{3}\d l\,R\boldsymbol{B}_{n}^{\text{pol}}\cdot\boldsymbol{n}-in\int_{\Omega_{i}}\d R\d Z\,RB_{n}^{\ph}.\label{eq:integral law-1}
\end{align}
As for currents we use the notation for weighted magnetic fluxes through
edges,
\begin{equation}
\Psi_{k}=\int_{k}R\,\boldsymbol{B}_{n}^{\text{pol}}\cdot\boldsymbol{n}dl\approx R_{k}\,\boldsymbol{B}_{n}^{\text{pol}}\cdot\boldsymbol{n}_{k}.
\end{equation}
The term through edge $l_{3}$ orthogonal to $n_{3}\parallel\nabla\psi$
is
\begin{align*}
\Psi_{3} & \approx R_{3}\,\boldsymbol{B}_{n}^{\text{pol}}\cdot\boldsymbol{n}_{3}\\
 & =R_{3}\,(\boldsymbol{B}_{n}^{\text{pol}}\cdot\nabla\psi)/|\nabla\psi|^{2}\\
 & =R_{3}\frac{B_{n}^{\psi}}{|\nabla\psi|^{2}}(\boldsymbol{n}_{3}\cdot\nabla\psi)\\
 & =R_{3}l_{3}\frac{B_{n}^{\psi}}{|\nabla\psi|^{2}}(\hat{\boldsymbol{n}}_{3}\cdot\nabla\psi)
\end{align*}
with sign depending on edge orientation. The second term yields
\begin{align*}
-\Psi_{\ph}=-in\int_{\Omega_{i}}\d R\d Z\,RB_{n}^{\ph} & \approx-inS_{\Omega k}\,R_{k}B_{nk}^{\ph}.
\end{align*}
The equation to solve is
\begin{equation}
A_{jk}I^{(k)}=\boldsymbol{q},
\end{equation}
with
\begin{align}
A_{jk} & =\delta_{(j-1)k}-\delta_{jk},
\end{align}
and
\begin{align}
q^{k} & =\mp R_{3}l_{3}\frac{B_{n}^{\psi}}{|\nabla\psi|}-inS_{\Omega k}\,R_{k}B_{nk}^{\ph}.
\end{align}
Here, a compatibility condition needs to be fulfilled - the flux through
one flux surface shell, so the sum over all triangles $k$ must vanish
with
\begin{align}
\sum_{k}(\Psi_{\ph}^{(k)}+\Psi_{3}^{(k)}) & =0.
\end{align}

\subsection*{Reduced MHD (unfinished)}

We take
\[
B_{n\ph}=0
\]
so
\begin{align*}
\boldsymbol{B}_{n} & =\nabla\psi_{n}\times\nabla\ph.
\end{align*}
Amperes law becomes
\begin{align*}
\boldsymbol{B}_{n} & =\frac{1}{4\pi}\boldsymbol{J}_{n}.
\end{align*}

\subsection*{Reading input files from experiment}
\begin{itemize}
\item ASDEX Upgrade has two formats that are common: CLISTE and EFIT/EQDSK
(g-files)
\item Much data in /proj/plasma/RMP/DATA/ASDEX/ and /proj/plasma/RMP/DATA2017/ASDEX/
\end{itemize}

\end{document}
