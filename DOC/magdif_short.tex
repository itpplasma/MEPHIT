%% LyX 2.3.2 created this file.  For more info, see http://www.lyx.org/.
%% Do not edit unless you really know what you are doing.
\documentclass[12pt,british,english]{article}
\usepackage[T1]{fontenc}
\usepackage[latin9]{inputenc}
\usepackage{geometry}
\geometry{verbose,tmargin=2cm,bmargin=2cm,lmargin=2cm,rmargin=2cm}
\setlength{\parskip}{\smallskipamount}
\setlength{\parindent}{0pt}
\usepackage{amsmath}
\usepackage{babel}
\begin{document}
\selectlanguage{british}%
\global\long\def\tht{\vartheta}%
\global\long\def\ph{\varphi}%
\global\long\def\balpha{\boldsymbol{\alpha}}%
\global\long\def\btheta{\boldsymbol{\theta}}%
\global\long\def\bJ{\boldsymbol{J}}%
\global\long\def\bGamma{\boldsymbol{\Gamma}}%
\global\long\def\bOmega{\boldsymbol{\Omega}}%
\global\long\def\d{\text{d}}%
\global\long\def\t#1{\text{#1}}%
\global\long\def\m{\text{m}}%
\global\long\def\bm{\text{\textbf{m}}}%
\global\long\def\k{\text{k}}%
\global\long\def\i{\text{i}}%

\title{\selectlanguage{english}%
Short comments on magdif.pdf}
\maketitle

\section*{Section 2.1: Iteration scheme}

Iterations in the kinetic code and in \emph{test\_arnoldi.f90} proceed
in the following way in \emph{next\_iteration}:
\begin{enumerate}
\item Add vacuum field to right-hand side
\item Perform calculation including vacuum field
\end{enumerate}
So
\begin{align*}
B_{k+1} & =K(B_{k}+B_{v})\\
\\
B_{0} & =0\\
B_{1} & =K(B_{0}+B_{v})=KB_{v}\\
B_{2} & =K(B_{1}+B_{v})=K(KB_{v}+B_{v})=(K^{2}+K)B_{v}\\
B_{3} & =K(B_{2}+B_{v})=K((K^{2}+K)B_{v}+B_{v})=(K^{3}+K^{2}+K)B_{v}
\end{align*}
and so on.

The MHD code proceeds as follows:
\begin{enumerate}
\item Perform calculation excluding vacuum field
\item Add vacuum field to the right-hand side
\end{enumerate}
So
\begin{align*}
B_{k+1} & =KB_{k}+B_{v}\\
\\
B_{0} & =B_{v}\\
B_{1} & =KB_{0}+B_{v}=KB_{v}+B_{v}\\
B_{2} & =KB_{1}+B_{v}=K(KB_{v}+B_{v})+B_{v}=(K^{2}+K+I)B_{v}\\
B_{3} & =KB_{2}+B_{v}=K(K^{2}+K+I)B_{v}+B_{v}=(K^{3}+K^{2}+K+I)B_{v}
\end{align*}
and so on. This means in each iteration $B_{\mathrm{MHD}}=B_{\mathrm{kin}}+B_{v}$.
Since this includes the zeroeth iteration, one may not initialize
$B$ with zeroes inside Arnoldi, but rather initialize it with $B_{v}$.
Then again Arnoldi requires the iterations without the added $B_{v}$,
which would have to be subtracted in the result (only plasma response
field, no vacuum field). The easiest way around is to define another
routine \emph{next\_iteration\_arnoldi} that works in the kinetic
way, just to get eigenvalues and -vectors. The remaining calculation
can proceed in the MHD way.

\section*{Section 5.2: Current perturbation}

Start with (5.9)
\begin{equation}
\boldsymbol{J}_{n}\times\boldsymbol{B}_{0}=c(\nabla p_{n}+inp_{n}\nabla\ph)-\boldsymbol{J}_{0}\times\boldsymbol{B}_{n}.
\end{equation}

We have (5.14)

\begin{equation}
\boldsymbol{l}\cdot\nabla\psi=(\boldsymbol{n}\times R\nabla\ph)\cdot\nabla\psi=-R\boldsymbol{B}_{0}^{\mathrm{pol}}\cdot\boldsymbol{n}.
\end{equation}

Then follows (5.15) with
\begin{align}
\boldsymbol{l}\cdot(\boldsymbol{J}_{n}\times\boldsymbol{B}_{0}) & =(\boldsymbol{l}\cdot\nabla\psi)J_{n}^{\varphi}+RB_{0}^{\varphi}\boldsymbol{J}_{n}^{\mathrm{pol}}\cdot\boldsymbol{n}\\
 & =RB_{0}^{\varphi}\boldsymbol{J}_{n}^{\mathrm{pol}}\cdot\boldsymbol{n}-RJ_{n}^{\varphi}\boldsymbol{B}_{0}^{\mathrm{pol}}\cdot\boldsymbol{n}.
\end{align}

First use (5.18)
\begin{equation}
\boldsymbol{l}\cdot(\nabla\varphi\times\boldsymbol{B}_{n}^{\mathrm{pol}})=-\frac{1}{R}\boldsymbol{B}_{n}^{\mathrm{pol}}\cdot\boldsymbol{n}.
\end{equation}

The result in (5.20) is
\begin{align}
\boldsymbol{J}_{0}\times\nabla\varphi & =\frac{1}{B_{0\varphi}}\left(c\nabla p_{0}-\frac{J_{0\varphi}}{R^{2}}\right)\nabla\psi\\
 & =\frac{1}{B_{0\varphi}}\left(cp_{0}^{\prime}(\psi)-J_{0}^{\varphi}\right)\nabla\psi.
\end{align}
So (5.21) becomes
\begin{align}
\boldsymbol{l}\cdot(\boldsymbol{J}_{0}\times\boldsymbol{B}_{n}) & =\boldsymbol{l}\cdot(B_{n\varphi}\boldsymbol{J}_{0}^{\mathrm{pol}}\times\nabla\varphi+J_{0\varphi}\nabla\varphi\times\boldsymbol{B}_{n}^{\mathrm{pol}})\\
 & =\frac{B_{n\varphi}}{B_{0\varphi}}\left(cp_{0}^{\prime}(\psi)-J_{0}^{\varphi}\right)\boldsymbol{l}\cdot\nabla\psi+J_{0\varphi}\boldsymbol{l}\cdot(\nabla\varphi\times\boldsymbol{B}_{n}^{\mathrm{pol}})\\
 & =\frac{B_{n\varphi}}{B_{0\varphi}}\left(cp_{0}^{\prime}(\psi)-J_{0}^{\varphi}\right)\boldsymbol{l}\cdot\nabla\psi-RJ_{0}^{\varphi}\boldsymbol{B}_{n}^{\mathrm{pol}}\cdot\boldsymbol{n}.
\end{align}
With (5.14) we obtain (5.22)

\begin{align}
\boldsymbol{l}\cdot(\boldsymbol{J}_{0}\times\boldsymbol{B}_{n}) & =\frac{B_{n\varphi}}{B_{0\varphi}}c\nabla p_{0}(\psi)\cdot\boldsymbol{l}+RJ_{0}^{\varphi}\left(\frac{B_{n\varphi}}{B_{0\varphi}}\boldsymbol{B}_{0}^{\mathrm{pol}}\cdot\boldsymbol{n}-\boldsymbol{B}_{n}^{\mathrm{pol}}\cdot\boldsymbol{n}\right).
\end{align}
we can insert into multiplication of $\boldsymbol{l}\cdot$(5.9) to
obtain (5.23) with
\begin{equation}
RB_{0}^{\varphi}\boldsymbol{J}_{n}^{\mathrm{pol}}\cdot\boldsymbol{n}-RJ_{n}^{\varphi}\boldsymbol{B}_{0}^{\mathrm{pol}}\cdot\boldsymbol{n}=c\left(\nabla p_{n}-\frac{B_{n\varphi}}{B_{0\varphi}}\nabla p_{0}\right)\cdot\boldsymbol{l}-RJ_{0}^{\varphi}\left(\frac{B_{n\varphi}}{B_{0\varphi}}\boldsymbol{B}_{0}^{\mathrm{pol}}\cdot\boldsymbol{n}-\boldsymbol{B}_{n}^{\mathrm{pol}}\cdot\boldsymbol{n}\right).
\end{equation}
Dividing by $\boldsymbol{B}_{n}^{\mathrm{pol}}\cdot\boldsymbol{n}=-(\boldsymbol{l}\cdot\nabla\psi)/R$
yields (5.24) with
\begin{equation}
RJ_{n}^{\varphi}=RB_{0}^{\varphi}\frac{\boldsymbol{J}_{n}^{\mathrm{pol}}\cdot\boldsymbol{n}}{\boldsymbol{B}_{n}^{\mathrm{pol}}\cdot\boldsymbol{n}}+\frac{c}{\boldsymbol{B}_{n}^{\mathrm{pol}}\cdot\boldsymbol{n}}\left(\frac{B_{n\varphi}}{B_{0\varphi}}\nabla p_{0}-\nabla p_{n}\right)\cdot\boldsymbol{l}+RJ_{0}^{\varphi}\left(\frac{B_{n\varphi}}{B_{0\varphi}}-\frac{\boldsymbol{B}_{n}^{\mathrm{pol}}\cdot\boldsymbol{n}}{\boldsymbol{B}_{0}^{\mathrm{pol}}\cdot\boldsymbol{n}}\right).
\end{equation}
On edge f some terms vanish, as $\boldsymbol{B}_{0}^{\mathrm{pol}}\cdot\boldsymbol{n}=0$
and $\nabla p_{0}\cdot\boldsymbol{l}=0$, so from (5.23) we get directly
to (5.25) with
\begin{equation}
R\boldsymbol{J}_{n}^{\mathrm{pol}}\cdot\boldsymbol{n}_{\mathrm{f}}=\frac{c\nabla p_{n}\cdot\boldsymbol{l}_{\mathrm{f}}}{B_{0}^{\varphi}}+R\frac{J_{0}^{\varphi}}{B_{0}^{\varphi}}\boldsymbol{B}_{n}^{\mathrm{pol}}\cdot\boldsymbol{n}_{\mathrm{f}}.
\end{equation}

\end{document}
