%% LyX 2.3.2 created this file.  For more info, see http://www.lyx.org/.
%% Do not edit unless you really know what you are doing.
\documentclass[12pt,british,english]{article}
\usepackage[T1]{fontenc}
\usepackage[latin9]{inputenc}
\usepackage{geometry}
\geometry{verbose,tmargin=2cm,bmargin=2cm,lmargin=2cm,rmargin=2cm}
\setlength{\parskip}{\smallskipamount}
\setlength{\parindent}{0pt}
\usepackage{amsmath}
\usepackage{babel}
\begin{document}
\selectlanguage{british}%
\global\long\def\tht{\vartheta}%
\global\long\def\ph{\varphi}%
\global\long\def\balpha{\boldsymbol{\alpha}}%
\global\long\def\btheta{\boldsymbol{\theta}}%
\global\long\def\bJ{\boldsymbol{J}}%
\global\long\def\bGamma{\boldsymbol{\Gamma}}%
\global\long\def\bOmega{\boldsymbol{\Omega}}%
\global\long\def\d{\text{d}}%
\global\long\def\t#1{\text{#1}}%
\global\long\def\m{\text{m}}%
\global\long\def\bm{\text{\textbf{m}}}%
\global\long\def\k{\text{k}}%
\global\long\def\i{\text{i}}%

\title{\selectlanguage{english}%
Short comment on chapter 5 of linear ideal MHD by Patrick Lainer}

\maketitle
\begin{align}
\boldsymbol{J}_{0}\times\nabla\varphi & =\frac{1}{B_{0\varphi}}\left(c\nabla p_{0}-\frac{J_{0\varphi}}{R^{2}}\right)\nabla\psi\\
 & =\frac{1}{B_{0\varphi}}\left(cp_{0}^{\prime}(\psi)-J_{0}^{\varphi}\right)\nabla\psi.
\end{align}
We have (5.21)
\begin{equation}
\boldsymbol{l}\cdot(\nabla\varphi\times\boldsymbol{B}_{n}^{\mathrm{pol}})=-\frac{1}{R}\boldsymbol{B}_{n}^{\mathrm{pol}}\cdot\boldsymbol{n}.
\end{equation}
So (5.18) becomes
\begin{align}
\boldsymbol{l}\cdot(\boldsymbol{J}_{0}\times\boldsymbol{B}_{n}) & =\boldsymbol{l}\cdot(B_{n\varphi}\boldsymbol{J}_{0}^{\mathrm{pol}}\times\nabla\varphi+J_{0\varphi}\nabla\varphi\times\boldsymbol{B}_{n}^{\mathrm{pol}})\\
 & =\frac{B_{n\varphi}}{B_{0\varphi}}\left(cp_{0}^{\prime}(\psi)-J_{0}^{\varphi}\right)\boldsymbol{l}\cdot\nabla\psi+J_{0\varphi}\boldsymbol{l}\cdot(\nabla\varphi\times\boldsymbol{B}_{n}^{\mathrm{pol}})\\
 & =\frac{B_{n\varphi}}{B_{0\varphi}}\left(cp_{0}^{\prime}(\psi)-J_{0}^{\varphi}\right)\boldsymbol{l}\cdot\nabla\psi-RJ_{0}^{\varphi}\boldsymbol{B}_{n}^{\mathrm{pol}}\cdot\boldsymbol{n}.
\end{align}
According to (5.14) and (5.16)
\begin{align}
\boldsymbol{l}\cdot(\boldsymbol{J}_{n}\times\boldsymbol{B}_{0}) & =(\boldsymbol{l}\cdot\nabla\psi)J_{n}^{\varphi}+RB_{0}^{\varphi}\boldsymbol{J}_{n}^{\mathrm{pol}}\cdot\boldsymbol{n}\\
 & =RB_{0}^{\varphi}\boldsymbol{J}_{n}^{\mathrm{pol}}\cdot\boldsymbol{n}-RJ_{n}^{\varphi}\boldsymbol{B}_{0}^{\mathrm{pol}}\cdot\boldsymbol{n}.
\end{align}
we can insert into multiplication of $\boldsymbol{l}\cdot$(5.12)
to obtain (5.22)
\begin{equation}
(\boldsymbol{l}\cdot\nabla\psi)J_{n}^{\varphi}+RB_{0}^{\varphi}\boldsymbol{J}_{n}^{\mathrm{pol}}\cdot\boldsymbol{n}=c\boldsymbol{l}\cdot\nabla p_{n}-\frac{B_{n\varphi}}{B_{0\varphi}}\left(cp_{0}^{\prime}(\psi)-J_{0}^{\varphi}\right)\boldsymbol{l}\cdot\nabla\psi+RJ_{0}^{\varphi}\boldsymbol{B}_{n}^{\mathrm{pol}}\cdot\boldsymbol{n}.
\end{equation}
Multiplying by $R$ and dividing by $(\boldsymbol{l}\cdot\nabla\psi)$
yields
\begin{equation}
RJ_{n}^{\varphi}=-RB_{0}^{\varphi}\frac{\boldsymbol{J}_{n}^{\mathrm{pol}}\cdot\boldsymbol{n}}{\boldsymbol{B}_{0}^{\mathrm{pol}}\cdot\boldsymbol{n}}+\frac{c}{\boldsymbol{B}_{0}^{\mathrm{pol}}\cdot\boldsymbol{n}}\left(\boldsymbol{l}\cdot\nabla p_{n}-\frac{B_{n\varphi}}{B_{0\varphi}}\boldsymbol{l}\cdot\nabla p_{0}\right)+RJ_{0}^{\varphi}\left(\frac{B_{n\varphi}}{B_{0\varphi}}-\frac{\boldsymbol{B}_{n}^{\mathrm{pol}}\cdot\boldsymbol{n}}{\boldsymbol{B}_{0}^{\mathrm{pol}}\cdot\boldsymbol{n}}\right).
\end{equation}
On edge f some terms vanish, as $\boldsymbol{l}\cdot\nabla\psi=0$,
so from (5.22) we get directly to (5.24) with
\begin{equation}
RB_{0}^{\varphi}\boldsymbol{J}_{n}^{\mathrm{pol}}\cdot\boldsymbol{n}_{\mathrm{f}}=c\boldsymbol{l}_{\mathrm{f}}\cdot\nabla p_{n}+RJ_{0}^{\varphi}\boldsymbol{B}_{n}^{\mathrm{pol}}\cdot\boldsymbol{n}_{\mathrm{f}}.
\end{equation}

\end{document}
