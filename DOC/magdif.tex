\documentclass[a4paper, twoside, 10pt, english]{article}
\setlength\overfullrule{5pt}

% eTeX stuff
\usepackage{etoolbox}

% fonts and maths
\usepackage{fontspec}
%\setmainfont{TeX Gyre Pagella}
%\setsansfont{TeX Gyre Heros}
%\setmonofont[Scale=MatchLowercase]{Source Code Pro}
\usepackage{luatexbase} % needed to correct bug in microtype w.r.t. LuaTeX
\usepackage{microtype}
\usepackage[normalem]{ulem}
\usepackage{amsmath}  % [fleqn]
\allowdisplaybreaks[1]
\numberwithin{equation}{section}
\usepackage[math-style=ISO, bold-style=ISO]{unicode-math}
%\setmathfont{TeX Gyre Pagella Math}
%\setmathfont{Latin Modern Math}[range=\dagger]
\DeclareSymbolFont{AMSb}{U}{msb}{m}{n}
\protected\def\mathbb#1{{\mathchar\numexpr256*\symAMSb+`#1\relax}}
\AfterPreamble{%
  \let\temp\varrho
  \let\varrho\rho
  \let\rho\temp
  \let\temp\vartheta
  \let\vartheta\theta
  \let\theta\temp
  \let\temp\varphi
  \let\varphi\phi
  \let\phi\temp
  \let\vecarr\vec
  \let\vec\symbf
}

% language
\usepackage{polyglossia}
\setmainlanguage{english}
\usepackage{csquotes}
\MakeOuterQuote{"}

% tables
\usepackage{array}
\usepackage{ragged2e}
%\usepackage{tabularx}
\usepackage{longtable}
\usepackage{booktabs}
\usepackage{caption}
\usepackage{subcaption}

% graphics
\usepackage{xcolor}
\newcommand\comment[1]{\begingroup\color{red}[#1]\endgroup}
\usepackage{graphicx}
\usepackage{grffile}
\usepackage{tikz}
\usetikzlibrary{math, arrows.meta, calc}

% import and display data
\usepackage{siunitx}
\sisetup{
  group-digits = false,
  add-decimal-zero = false,
  retain-unity-mantissa = false,
  separate-uncertainty,
  round-mode = off,
  %round-precision = 2,
  table-number-alignment = left,
  detect-mode = true
}

% (pseudo)code
\usepackage{algorithm}
\usepackage{algpseudocode}

% page layout
\raggedbottom
\tolerance=2000
\usepackage{enumitem}
\setlist{align=left}
\usepackage{parskip}
%\usepackage{setspace}
%\setstretch{1.125}
\usepackage{lastpage}
\usepackage[margin=2cm, head=15pt, includehead, includefoot]{geometry}
\usepackage[bottom, stable, perpage]{footmisc}
\usepackage{fancyhdr}
\fancyhf{}
\fancyfoot[C]{\thepage/\pageref{LastPage}}
\fancyhead[RO]{\textsl{\rightmark}}
\fancyhead[LE]{\textsl{\leftmark}}
\pagestyle{fancy}
\renewcommand{\sectionmark}[1]{\markboth{\thesection~#1}{}}
\renewcommand{\subsectionmark}[1]{\markright{\thesubsection~#1}}

% front and back matter
\usepackage{eso-pic}
\usepackage[
  backend = biber,
  bibstyle = numeric,
  citestyle = authoryear,
  sorting = none,
  maxcitenames = 1,
  maxbibnames = 5
]{biblatex}
\renewcommand*\finalnamedelim{,\ }
\DefineBibliographyStrings{english}{%
  andothers        = {et\,al\adddot},
  andmore          = {et\,al\adddot}
}
\addbibresource{magdif.bib}
\usepackage[
% pdfa         = true,
  unicode      = true,
  bookmarks    = true,
  colorlinks   = true,
  linkcolor    = black,
  urlcolor     = black,
  citecolor    = black,
  pdfstartview = FitH
]{hyperref}
\hypersetup{
  pdftitle    = {Magnetic differential equations for stationary linear ideal MHD and their numerical solution},
  pdfauthor   = {Christoper Albert, Patrick Lainer},
  pdfsubject  = {write-up},
  pdfkeywords = {magnetohydrodynamics},
  pdflang     = {en}
}

% references
\usepackage{cleveref}

% debugging
%\usepackage[inline]{showlabels}
%\usepackage[notref]{showkeys}

\title{Magnetic differential equations for stationary linear ideal MHD and their numerical solution}
\author{Christopher Albert, Patrick Lainer}

% math shortcuts etc.
\DeclareMathOperator\Real{Re}
\DeclareMathOperator\Imag{Im}
\DeclareMathOperator\sgn{sgn}
\newcommand*\grad{\ensuremath{\nabla}}
\newcommand*\divg{\ensuremath{\nabla \cdot}}
\newcommand*\curl{\ensuremath{\nabla \times}}
\DeclareMathOperator\Grad{\mathbf{grad}}
\DeclareMathOperator\Divg{div}
\DeclareMathOperator\Curl{\mathbf{rot}}

\newcommand*\diff{\ensuremath{\symrm{d}}}  % differential
\newcommand*\e{\ensuremath{\symrm{e}}}  % Euler's constant
\newcommand*\im{\ensuremath{\symrm{i}}}  % imaginary unit
\newcommand*\pd[2][]{\ensuremath{\frac{\partial #1}{\partial #2}}}  % partial derivative
\newcommand*\td[2][]{\ensuremath{\frac{\diff #1}{\diff #2}}}  % total derivative
\newcommand*\norm[1]{\ensuremath{\left \lVert #1 \right \rVert}}  % norm
\newcommand*\pol{\ensuremath{\textrm{pol}}}  % poloidal
\newcommand*\tor{\ensuremath{\textrm{tor}}}  % toroidal
\newcommand*\polGrad{\ensuremath{\Grad_{\pol}}}  % transverse gradient
\newcommand*\polDivg{\ensuremath{\Divg_{\pol}}}  % transverse divergence
\newcommand*\polCurlvec{\ensuremath{\Curl_{\pol}}}  % transverse vector curl
\newcommand*\polCurl{\ensuremath{\operatorname{rot}_{\pol}}}  % transverse scalar curl

\newcommand*\Bvac{\ensuremath{\delta \vec{B}_{\text{v}}}}  % vacuum field
\newcommand*\Bplas{\ensuremath{\delta \vec{B}_{\text{p}}}}  % plasma response
\newcommand*\Bpert{\ensuremath{\delta \vec{B}}}  % full perturbation
\newcommand*\fs{\ensuremath{\textrm{f}}}  % flux surface edge label
\newcommand*\inw{\ensuremath{\textrm{i}}}  % inward current edge label
\newcommand*\out{\ensuremath{\textrm{o}}}  % outward current edge label
\newcommand*\vfs{\ensuremath{\textrm{F}}}  % label for vertex opposite flux surface edge
\newcommand*\vinw{\ensuremath{\textrm{I}}}  % label for vertex opposite inward current edge
\newcommand*\vout{\ensuremath{\textrm{O}}}  % label for vertex opposite outward current edge

\begin{document}

\maketitle
\tableofcontents

\newpage
\section{Stationary linear perturbation of ideal MHD equilibrium}

For the intended application on stationary (compared to MHD mode eigenfrequencies) non-axisymmetric magnetic perturbations by external coils, we consider a perturbed ideal MHD equilibrium for pressure $p$, currents $\vec{J}$ and magnetic field $\vec{B}$ fulfilling
\begin{align}
  \grad p &= \frac{1}{c} \vec{J} \times \vec{B}, \label{eq:mhd-gen} \\
  \curl \vec{B} &= \frac{4 \pi}{c} \vec{J}, \label{eq:ampere-gen} \\
  \divg \vec{B} &= 0. \label{eq:divfree-gen}
\end{align}
Starting with a given MHD equilibrium fulfilling \cref{eq:mhd-gen,eq:divfree-gen} denoted by subscripts "$0$", linear order equations for an external magnetic perturbation (denoted by $\delta$) split into a vacuum and a plasma part (subscript $\text{v}$ and $\text{p}$, respectively) are
\begin{align}
  \grad \delta p &= \frac{1}{c} \left( \vec{J}_{0} \times \Bpert + \delta \vec{J} \times \vec{B}_{0} \right), \label{eq:mhd} \\
  \Bpert &= \Bvac + \Bplas, \\
  \Bvac &= \frac{1}{c} \oint \frac{I_{\text{c}}(\vec{r}') \, \diff \vec{l}' \times \vec{r}}{\lvert \vec{r} - \vec{r}' \rvert^{3}}, \label{eq:biot-savart} \\
  \Bplas &= \curl \delta \vec{A}, \\
  \curl (\curl \delta \vec{A}) &= \frac{4 \pi}{c} \delta \vec{J}, \label{eq:ampere} \\
  \Rightarrow \divg \Bpert &= \divg \delta \vec{J} = 0. \label{eq:divfree}
\end{align}
Here the perturbation field in vacuum, $\Bvac$, is pre-evaluated by a \textsc{Biot}--\textsc{Savart} integral over external\footnote{i.e.\ entirely outside the plasma region} coil currents $I_{\text{c}} (\vec{r}')$. This induces a plasma response, resulting in the current density perturbation $\delta \vec{J}$. The perturbation field from currents within the plasma, $\Bplas$, is in turn computed from $\delta \vec{J}$, again giving rise to a plasma response current. Now, the linearized force balance \cref{eq:mhd} is used to compute $\delta \vec{J}$ for given $\Bpert$ whereas \cref{eq:ampere} yields $\Bplas$ for given $\delta \vec{J}$.

The solution of \cref{eq:mhd} can further be split into two steps: First the pressure perturbation $\delta p$ is found, and then the plasma current density $\delta \vec{J}$ is computed using the condition $\divg \delta \vec{J} = 0$. For an unperturbed equilibrium with nested flux surfaces, both steps can be performed in a radially local manner if a field-aligned computational grid is used, which will become clear in the following sections. Radial coupling happens by the combination of the two individual steps since their effective radial locations of computation are shifted by a half-step in radial grid distance.

\Cref{eq:mhd} and \cref{eq:ampere} are solved in an alternating way until convergence is reached, as described in \cref{sec:iteration}. In addition a preconditioner is used to enhance convergence, which is discussed in \cref{sec:Arnoldi}.

\subsection{Iteration scheme}
\label{sec:iteration}

With abstract operators $\hat{P}$ for the computation in \cref{eq:mhd} and $\hat{M}$ for the computation in \cref{eq:ampere}, this can be written in compact form:
\begin{align}
  \delta \vec{J} &= \hat{P} \Bpert = \hat{P} \left ( \Bvac + \Bplas \right ), \\
  \Bplas &= \hat{M} \delta \vec{J}.
\end{align}
Substituting for $\delta \vec{J}$ and using a shorthand $\hat{K} = \hat{M} \hat{P}$ gives
\begin{align}
  \hat{K} \left ( \Bvac + \Bplas \right ) &= \Bplas, \label{eq:K_fixed-point} \\
  \hat{K} \Bvac &= \left ( \hat{1} - \hat{K} \right ) \Bplas, \\
  \left ( \hat{1} - \hat{K} \right )^{-1} \hat{K} \Bvac &= \Bplas.
\end{align}
The first term can be rewritten in the form of a \textsc{Neumann} series, a generalisation of geometric series to operators, assuming the series converges:
\begin{gather}
  \left ( \hat{1} - \hat{K} \right )^{-1} = \sum_{k = 0}^{\infty} \hat{K}^{k}. \label{eq:Neumann_series}
\end{gather}
This way a consistent solution for $\Bplas$ can be computed from $\Bvac$ by repeated application of $\hat{K}$, given explicitly by the infinite series
\begin{gather}
  \Bplas = \left ( \hat{1} + \hat{K} + \hat{K}^{2} + \dotsb \right ) \hat{K} \Bvac = \sum_{k = 1}^{\infty} \hat{K}^{k} \Bvac = \sum_{k = 1}^{\infty} \Bpert^{(k)}. \label{eq:K_series}
\end{gather}
In \cref{eq:K_series} each term is given by the recurrence relation
\begin{gather}
  \Bpert^{(k+1)} = \hat{K} \Bpert^{(k)}.
\end{gather}
Adding the vacuum field as the initial value,
\begin{gather}
  \Bpert^{(0)} = \Bvac
\end{gather}
the series' terms are accumulated for the self-consistent solution:
\begin{gather}
  \Bpert = \sum_{k = 0}^{\infty} \Bpert^{(k)}.
\end{gather}
Alternatively, \cref{eq:K_fixed-point} can be expanded,
\begin{gather}
  \Bplas = \hat{K} \left ( \Bvac + \Bplas \right ) = \hat{K} \left ( \Bvac + \hat{K} \left ( \Bvac + \Bplas \right ) \right ) = \dotsb,
\end{gather}
yielding a fixed-point iteration for $\Bpert$:
\begin{gather}
  \Bpert^{[k+1]} = \hat{K} \Bpert^{[k]} + \Bvac.
\end{gather}
Compared to the previous approach, this one is cumulative, i.e. it immediately produces the next approximation of the full perturbation. In other words, it corresponds to the sequence of partial sums of the previous infinite series:
\begin{gather}
  \Bpert^{[n]} = \sum_{k = 0}^{n} \Bpert^{(k)}. \label{eq:K_partial_sum}
\end{gather}
For this to be consistent, the initial value is also given by the vacuum field,
\begin{gather}
  \Bpert^{[0]} = \Bvac.
\end{gather}
For illustration, both approaches are compared side-by-side in \cref{tab:comp_iter}.
\begin{table}[bth]
  \caption{Comparison of iteration with series (non-cumulative) and sequence (cumulative)}
  \label{tab:comp_iter}
  \begin{align*}
    \text{iteration step:} && \Bpert^{(k)} &= \hat{K} \Bpert^{(k-1)} & \Bpert^{[k]} &= \hat{K} \Bpert^{[k-1]} + \Bvac, \\
    \text{initial value:} && \Bpert^{(0)} &= \Bvac & \Bpert^{[0]} &= \Bvac, \\
    \text{step 1:} && \Bpert^{(1)} &= \hat{K} \Bpert^{(0)} = \hat{K} \Bvac & \Bpert^{[1]} &= \hat{K} \Bpert^{[0]} + \Bvac = \hat{K} \Bvac + \Bvac, \\
    \text{step 2:} && \Bpert^{(2)} &= \hat{K} \Bpert^{(1)} = \hat{K}^{2} \Bvac & \Bpert^{[2]} &= \hat{K} \Bpert^{[1]} + \Bvac = \left ( \hat{K}^{2} + \hat{K} + \hat{I} \right ) \Bvac, \\
    \text{explicit form:} && \Bpert^{(n)} &= \hat{K}^{n} \Bvac & \Bpert^{[n]} &= \sum_{k = 0}^{n} \hat{K}^{k} \Bvac, \\
    \text{full perturbation:} && \Bpert &= \sum_{k = 0}^{\infty} \hat{K}^{n} \Bvac & \Bpert &= \Bpert^{[\infty]}, \\
    \text{full plasma response:} && \Bplas &= \sum_{k = 1}^{\infty} \hat{K}^{n} \Bvac & \Bplas &= \Bpert^{[\infty]} - \Bvac.
  \end{align*}
\end{table}
For the implementation of preconditioned iterations (see \cref{sec:Arnoldi}), the cumulative approach is more convenient. To reproduce the intermediate summands, we use \cref{eq:K_partial_sum} and arrive at
\begin{gather}
  \Bpert^{(k)} = \Bpert^{[k]} - \Bpert^{[k-1]}.
\end{gather}

\subsection{Enhanced convergence with preconditioned iterations}
\label{sec:Arnoldi}

Both approaches outlined in \cref{sec:iteration} hinge on the convergence of the \textsc{Neumann} series in \cref{eq:Neumann_series}. The convergence criterion for the similar geometric series of scalars is not directly applicable to operators, but to their corresponding spectrum of eigenvalues. Thus we shall now consider a discretized equation of finite dimension $N$.

We start from the fixed-point iteration of the previous section which has the general form
\begin{gather}
  \vec{x} = \hat{K} \vec{x} + \vec{x}_{0}. \label{eq:Arnoldi_fixed_point}
\end{gather}
$\vec{x}$ serves as a shorthand for $\Bpert$ and a reminder that the derivations in this section are not just valid for the specific problem of calculating magnetic fields. Similarly, $\vec{x}_{0}$ stands in for $\Bvac$.  Assuming the linear operator $\hat{K}$ is non-singular, we can formally write down an eigendecomposition
\begin{gather}
  \hat{K} = \hat{V} \hat{\Lambda} \hat{V}^{-1},
\end{gather}
where $\hat{\Lambda}$ is a diagonal matrix with the eigenvalues,
\begin{gather}
  \hat{\Lambda} = \begin{pmatrix}
    \lambda_{1} & & & \\
    & \lambda_{2} & & \\
    & & \ddots & \\
    & & & \lambda_{N}
  \end{pmatrix} = \vec{\lambda} \hat{I},
\end{gather}
$\hat{V}$ contains the corresponding eigenvectors as its columns,
\begin{gather}
  \hat{V} = \left ( \vec{v}_{1}, \vec{v}_{2}, \dotsc, \vec{v}_{N} \right ),
\end{gather}
and $\hat{V}^{-1}$ is the inverse of $\hat{V}$. $\vec{x}$ can then be expressed in the eigenbasis with components $x_{k}'$,
\begin{gather}
  \vec{x} = \sum_{k = 1}^{N} x_{k}' \vec{v}_{k} = \hat{V} \vec{x}',
\end{gather}
and transformed back to the original basis by the inverse,
\begin{gather}
  \vec{x}' = \hat{V}^{-1} \vec{x}.
\end{gather}
Rearranging \cref{eq:Arnoldi_fixed_point} to
\begin{gather}
  \left ( \hat{I} - \hat{K} \right ) \vec{x} = \vec{x}_{0}, \label{eq:Arnoldi_direct}
\end{gather}
multiplying from the left with $\hat{V}^{-1}$ and expanding in the eigenbasis yields
\begin{gather}
  \Bigl ( \underbrace{\hat{V}^{-1} \hat{I} \hat{V}}_{\hat{I}} - \hat{\Lambda} \Bigr ) \vec{x}' = \vec{x}_{0}'.
\end{gather}
Solving for $\vec{x}'$ gives
\begin{gather}
  \vec{x}' = \left ( \hat{I} - \hat{\Lambda} \right )^{-1} \vec{x}_{0}' = \begin{pmatrix}
    \frac{1}{1 - \lambda_{1}} & & & \\
    & \frac{1}{1 - \lambda_{2}} & & \\
    & & \ddots & \\
    & & & \frac{1}{1 - \lambda_{N}}
  \end{pmatrix} \begin{pmatrix} x_{01}' \\ x_{02}' \\ \vdots \\ x_{0N}' \end{pmatrix}, \label{eq:direct_inversion}
\end{gather}
which can then be transformed back to the original basis. This approach yields a solution without resorting to series expansion and associated considerations of convergence, instead inverting the matrix directly. On the other hand, full diagonalization of $\hat{K}$ is computationally expensive, but partial diagonalization can be used to enhance convergence, or permit convergence at all, as will be seen below.

Applying the eigendecomposition to the operator series, we see that repeated application of $\hat{K}$ simplifies to
\begin{gather}
  \hat{K}^{n} \vec{x}_{0} = \underbrace{\left ( \hat{V} \hat{\Lambda} \hat{V}^{-1} \right ) \left ( \hat{V} \hat{\Lambda} \hat{V}^{-1} \right ) \dotsb \left ( \hat{V} \hat{\Lambda} \hat{V}^{-1} \right )}_{k} \vec{x}_{0}= \hat{V} \hat{\Lambda}^{n} \hat{V}^{-1} \vec{x}_{0} = \hat{V} \hat{\Lambda}^{n} \vec{x}_{0}' = \sum_{k = 1}^{N} \lambda_{k}^{n} x_{0 k}' \vec{v}_{k}.
\end{gather}
Comparing this to the solution in \cref{eq:direct_inversion}, it becomes apparent that convergence of the \textsc{Neumann} operator series is equivalent to the convergence of the geometric series of all eigenvalues. Since the geometric series only converges for $\lvert \lambda_{k} \rvert < 1$ (and only reasonably fast for $\lvert \lambda_{k} \rvert \ll 1$), we need direct inversion for the largest eigenvalues, i.e.\ partial diagonalization. To find the largest eigenvalues, we use the \textsc{Arnoldi} method summarized in \cref{app:Arnoldi}. This is a \textsc{Krylov} subspace method that reduces to the \textsc{Lanczos} method for Hermitian matrices and is also used as part of the generalized minimal residual method (GMRES). It does not involve matrix-matrix multiplication but only matrix-vector multiplication. Thus the matrix needs not be given explicitly, only its action on a given vector, which is fulfilled in our case for $\hat{K}$.

Using the \textsc{Arnoldi} method will give us the largest $r$ eigenvalues, denoted as $\vec{\lambda}_{r}$ or equivalently $\hat{\Lambda}_{r}$ and commonly called \textsc{Ritz} eigenvalues, as well as an orthonormal set of associated eigenvectors, this time arranged in an $N \times r$ matrix $\hat{V}_{r}$. The latter span the \textsc{Krylov} subspace of the full eigenspace. Instead of the eigenvalue equation of full rank,
\begin{gather}
  \hat{K} \vec{v}_{k} = \lambda_{k} \vec{v}_{k} \quad \forall k = 1, 2, \dotsc, N,
\end{gather}
or equivalently,
\begin{gather}
  \hat{K} \hat{V} = \vec{\lambda} \hat{V} = \hat{V} \hat{\Lambda},
\end{gather}
we can write down the reduced eigenvalue equation in the \textsc{Krylov} subspace,
\begin{gather}
  \hat{K} \hat{V}_{r} = \vec{\lambda}_{r} \hat{V}_{r} = \hat{V}_{r} \hat{\Lambda}_{r}. \label{eq:reduced_eigvals}
\end{gather}
Note that, compared to the eigenvalue equation in full space, the $r \times r$ matrix $\hat{\Lambda}_{r}$ has to be to the right of the $N \times r$ matrix $\hat{V}_{r}$.

With the largest $r$ eigenvalues now known, we want to find a preconditioner that modifies the direct iteration step in \cref{eq:Arnoldi_fixed_point} so that the largest eigenvalues don't contribute. Usually this is written by left-multiplying \cref{eq:Arnoldi_direct} by the inverse of a full-rank linear operator:
\begin{gather}
  \hat{\Pi}^{-1} \left ( \hat{I} - \hat{K} \right ) \vec{x} = \hat{\Pi}^{-1} \vec{x}_{0}. \label{eq:precon}
\end{gather}
We choose
\begin{gather}
  \hat{\Pi}^{-1} = \hat{I} - \hat{A}
\end{gather}
with some general matrix $\hat{A}$. \Cref{eq:precon} then becomes
\begin{gather}
  \left ( \hat{I} - \hat{A} - \left ( \hat{I} - \hat{A} \right ) \hat{K} \right ) \vec{x} = \left ( \hat{I} - \hat{A} \right ) \vec{x}_{0},
\end{gather}
which can be rearranged to resemble \cref{eq:Arnoldi_direct},
\begin{gather}
  \left ( \hat{I} - \hat{\bar{K}} \right ) \vec{x} = \bar{\vec{x}}_{0},
\end{gather}
with a modified iteration step
\begin{gather}
  \hat{\bar{K}} = \hat{A} + \left ( \hat{I} - \hat{A} \right ) \hat{K} \label{eq:Kbar}
\end{gather}
and a modified initial value
\begin{gather}
  \bar{\vec{x}}_{0} = \left ( \hat{I} - \hat{A} \right ) \vec{x}_{0}.
\end{gather}
By analogy, we can then write the explicit preconditioned iteration step as
\begin{gather}
  \bar{\vec{x}}^{[k+1]} = \hat{\bar{K}} \bar{\vec{x}}^{[k]} + \bar{\vec{x}}_{0}.
\end{gather}
Replacing the modified quantities on the right-hand side according to their definitions and rearranging gives
\begin{gather}
  \bar{\vec{x}}^{[k+1]} = \hat{A} \bar{\vec{x}}^{[k]} + \left ( \hat{I} - \hat{A} \right ) \left ( \hat{K} \bar{\vec{x}}^{[k]} + \vec{x}_{0} \right ).
\end{gather}
Now the last term in parentheses reproduces the direct, unmodified iteration yielding an unmodified $\vec{x}^{[k+1]}$,
\begin{gather}
  \bar{\vec{x}}^{[k+1]} = \hat{A} \bar{\vec{x}}^{[k]} + \left ( \hat{I} - \hat{A} \right ) \vec{x}^{[k+1]},
\end{gather}
which can again be rearranged for a final result,
\begin{gather}
  \bar{\vec{x}}^{[k+1]} = \vec{x}^{[k+1]} - \hat{A} \left ( \vec{x}^{[k+1]} - \bar{\vec{x}}^{[k]} \right ).
\end{gather}
Compared to the direct iterations, only one additional matrix-vector multiplication is necessary, but see below for details.

Now all we need is to compute $\hat{A}$. We required that the largest $r$ eigenvalues don't contribute to iterations, so we demand
\begin{gather}
  \hat{\bar{K}} \vec{v}_{k} \overset{!}{=} \vec{0} \quad \forall k \le r.
\end{gather}
This can be compactly rewritten and expanded via \cref{eq:Kbar,eq:reduced_eigvals} to give
\begin{gather}
  \hat{\bar{K}} \hat{V}_{r} = \hat{A} \hat{V}_{r} + \left ( \hat{I} - \hat{A} \right ) \hat{K} \hat{V}_{r} = \hat{A} \hat{V}_{r} + \left ( \hat{I} - \hat{A} \right ) \hat{V}_{r} \hat{\Lambda}_{r} \overset{!}{=} \hat{0}.
\end{gather}
This can in turn be rearranged to
\begin{gather}
  \hat{A} \hat{V}_{r} \left ( \hat{\Lambda}_{r} - \hat{I} \right ) \overset{!}{=} \hat{V}_{r} \hat{\Lambda}_{r}.
\end{gather}
Right-multiplying with the inverse of $\hat{\Lambda}_{r} - \hat{I}$ yields
\begin{gather}
  \hat{A} \hat{V}_{r} \overset{!}{=} \hat{V}_{r} \hat{\Lambda}_{r} \left ( \hat{\Lambda}_{r} - \hat{I} \right )^{-1}.
\end{gather}
Now $\hat{V}_{r}$ is not square and thus cannot be inverted, but we can add a unity term on the right-hand side:
\begin{gather}
  \hat{A} \hat{V}_{r} \overset{!}{=} \hat{V}_{r} \hat{\Lambda}_{r} \left ( \hat{\Lambda}_{r} - \hat{I} \right )^{-1} \left ( \hat{V}_{r}^{\dagger} \hat{V}_{r} \right )^{-1} \hat{V}_{r}^{\dagger} \hat{V}_{r}.
\end{gather}
By direct comparison it can be seen that a solution is given by
\begin{gather}
  \hat{A} \equiv \hat{V}_{r} \hat{\Lambda}_{r} \left ( \hat{\Lambda}_{r} - \hat{I} \right )^{-1} \left ( \hat{V}_{r}^{\dagger} \hat{V}_{r} \right )^{-1} \hat{V}_{r}^{\dagger}.
\end{gather}
The inner part can be grouped to the inverse of an $r \times r$ matrix $\hat{L}_{r}$,
\begin{gather}
  \hat{L}_{r} = \hat{V}_{r}^{\dagger} \hat{V}_{r} \left ( \hat{\Lambda}_{r} - \hat{I} \right ).
\end{gather}
This can then be conveniently inverted using the LAPACK routine \texttt{zgesv} to solve for $\hat{Y}$ in
\begin{gather}
  \hat{L}_{r} \hat{Y} = \hat{I}.
\end{gather}
Since $\hat{A}$ is constant during iterations, these computations would only need to be done once before preconditioned iterations start. Note that in practice though, $\hat{A}$ is not stored explicitly, since $r \ll N$ -- in test runs, $r \approx \num{1e1}, N \approx \num{1e5}$. Instead of keeping $N^2$ entries of the dense matrix $\hat{A}$ in storage, we keep $N r$ entries of $\hat{V}_{r}$ and $r^2$ entries of $\hat{\Lambda}_{r} \hat{L}_{r}^{-1}$. Applying the matrices on a given $\vec{x}$ from right to left then requires additional matrix-vector multiplications, but in the worst case, this involves only $2 N r + r^2$ floating-point operations compared to $N^2$ for one matrix-vector multiplication with full $\hat{A}$. For an overview on matrix sizes, see \cref{tab:matrix_dimensions}.
\begin{longtable}{ll}
  \caption{Dimensions of quantities used in derivation of preconditioned iterations.}
  \label{tab:matrix_dimensions} \\
  \toprule
  Quantity & Dimension \\
  \midrule
  \endfirsthead
  \toprule
  Quantity & Dimension \\
  \midrule
  \endhead
  $\vec{x}$ & $N$ \\
  $\vec{\lambda}_{r}$ & $r$ \\
  $\vec{v}_{k}$ & $N$ \\
  $\hat{\Lambda}_{r}$ & $r \times r$ \\
  $\hat{V}_{r}$ & $N \times r$ \\
  $\hat{V}_{r}^{\dagger}$ & $r \times N$ \\
  $\hat{A}$ & $N \times N$ \\
  $\hat{L}_{r}$ & $r \times r$ \\
  \bottomrule
\end{longtable}

\clearpage
\section{Geometrical considerations}

Here we limit the analysis to an axisymmetric unperturbed equilibrium and a single toroidal perturbation harmonic $\Bpert = \Real (\vec{B}_{n} \e^{\im n \phi})$ with a similar notation for other perturbed quantities. As all equations are linear, a superposition of multiple harmonics is easily possible. Note that $n \neq 0$; such a perturbation is necessarily small and considered part of the axisymmetric equilibrium.
% clarify why/whether $n = 0$ is in fact negligible

In axisymmetric coordinate systems, such as cylindrical $(R, \phi, Z)$, the equations to solve for harmonics in the toroidal angle $\phi$ are
\begin{align}
  \grad p_{n} + \im n p_{n} \grad \phi &= \frac{1}{c} \left( \vec{J}_{0} \times \vec{B}_{n} + \vec{J}_{n} \times \vec{B}_{0} \right), \label{eq:mhd-phi} \\
  \divg \vec{J}_{n}^{\pol}+ \im n J_{n}^{\phi} &= 0. \label{eq:divfree-phi}
\end{align}
now with a 2D $\grad$ operator acting in the poloidal ($RZ$) plane. The divergence operator is defined via
\begin{gather*}
  \divg \vec{u} = \frac{1}{R \sqrt{g_{\pol}}} \pd{x^{k}} (R \sqrt{g_{\pol}} u^{k}),
\end{gather*}
where $\sqrt{g_{\pol}}$ is the metric tensor of the coordinates in the poloidal plane, which is equal to $1$ for cylindrical coordinates.

The representation of equilibrium field $\vec{B}_{0}$ is given by
\begin{gather}
  \vec{B}_{0} = \vec{B}_{0}^{\pol} + \vec{B}_{0}^{\tor},
\end{gather}
where 
\begin{align}
  \vec{B}_{0}^{\pol} &= \grad \psi \times \grad \phi, \\
  \vec{B}_{0}^{\tor} &= B_{0 \phi} \grad \phi.
\end{align}
Here, $\psi$ is the disc poloidal flux $\Psi_{\pol}$ divided by $2 \pi$ and the sign chosen so that increases towards the magnetic axis -- $\psi_{\text{min}}$ is located at the outermost surface and $\psi_{\text{max}}$ is located at the magnetic axis. With $\phi$ increasing counter-clockwise when viewed from the top, this still constitutes a right-handed system.

This leads to the following contravariant components for $\vec{B}_{0}^{\pol}$ in cylindrical coordinates:
\begin{align}
  B_{0}^{R} &= (\grad \psi \times \grad \phi)^{R} = -\frac{1}{R} \pd[\psi]{Z}, \\
  B_{0}^{Z} &= (\grad \psi \times \grad \phi)^{Z} = \frac{1}{R} \pd[\psi]{R}.
\end{align}
Note that the toroidal field coils produce a magnetic field that is roughly proportional to $\frac{1}{R}$, so $B_{0 \phi}$ is assumed to be constant over the entire plasma volume.

\subsection{Coordinate conventions}
\label{sec:cocos}

We use two different right-handed coordinate systems: Firstly cylindrical coordinates $(R, \phi, Z)$ with $\phi$ running counterclockwise as seen from above, so
\begin{gather*}
  x = R \cos \phi, \quad y = R \sin \phi, \quad z = Z.
\end{gather*}
Those coordinates are used in computations and for meshes. Secondly, also symmetry flux coordinates $(\rho, \theta, \zeta)$ are used at some point for derivations. Here, we use the convention of \cite{dHaeseleer91} with radial coordinate growing towards the outside of the torus. An overview of possible conventions is given in \cite{Sauter13}, whose notation we will follow in this section. As a radial variable we use $\rho = \sigma_{\pol} \psi$ with $\sigma_{\pol} = +1$ for counterclockwise poloidal field $\vec{B}_{0}^{\pol}$ and $\sigma_{\pol} = -1$ for clockwise field. Here $\psi$ is the normalized ribbon poloidal flux also usually used in MHD for the \textsc{Grad}--\textsc{Shafranov} equation. The toroidal symmetry angle is defined in the direction opposite to $\phi$ with
\begin{gather*}
  \zeta = \frac{\pi}{2} - \phi.
\end{gather*}
The flux angle $\theta$ has the same orientation as $\vartheta = \arctan \frac{Z}{R}$, so the sign $\sigma_{\pol} = \sgn B^{\theta} = \sgn B^{\vartheta}$ is positive for counter-clockwise and negative for clockwise orientation in the poloidal plane. The Jacobian of symmetry flux coordinates $(\rho, \theta, \zeta) = (\sigma_{\pol} \psi, \theta, \frac{\pi}{2} - \phi)$ is
\begin{gather}
  \sqrt{g} = \frac{\sigma_{\pol}}{B_{0}^{\theta}} = \sigma_{\pol} \frac{q}{B_{0}^{\zeta}} = -\sigma_{\pol} \frac{q}{B_{0}^{\phi}} = -\sigma_{\pol} \frac{q R^{2}}{B_{0 \phi}}.
\end{gather}
If $B_{0}^{\theta}$ changes sign, both $\sigma_{\pol}$ and $q$ change sign, and a sign change in $B_{0}^{\phi}$ leads to a sign change also in $q$. Thus, $\sqrt{g}$ is always positive. The sign of $q$ describes the sign of the helicity in $(\theta, \zeta)$ and can be computed as described in \cref{sec:safety_factor}. Note that in our configuration (ASDEX Upgrade) $\sigma_{\pol} = -1$ and $B_{0 \phi}$ is positive, so $q$ is negative.

\subsection{Discretization and local coordinate system}
\label{sec:grid}

[\textsc{Raviart}--\textsc{Thomas} elements, \ldots]
\begin{gather*}
  \grad \psi \parallel \grad p_{0} \parallel \vec{n}_{\fs} \\
  \grad \phi \perp \left( \grad \psi, \grad p_{0}, \vec{n}, \vec{l}, \vec{B}_{\bullet}^{\pol}, \vec{h}_{0}^{\pol}, \vec{J}_{\bullet}^{\pol}, \grad p_{n} \right)
\end{gather*}

\begin{figure}[bth]
  \centering
  \begin{subfigure}[b]{0.33\textwidth}
    \centering
    \input{grid0.tpx}
    \caption{The innermost loop of the grid with the magnetic axis at its center. Edge \fs\ lies on the flux surface in the infinitesimal limit.}
    \label{fig:grid0}
  \end{subfigure}
  \quad
  \begin{subfigure}[b]{0.5\textwidth}
    \centering
    \input{grid1.tpx}
    \caption{One of the outer loops of the grid with two alternating kinds of triangles with edge \fs\ lying on the inner and outer flux surface respectively.}
    \label{fig:grid1}
  \end{subfigure}
  \caption{The 2D mesh is given by a triangulation of poloidal cross-sections of the nested flux surfaces, resulting in \enquote{loops}. The cross-sections are assumed to be circular for illustration purposes.}
  \label{fig:grid}
\end{figure}

For each edge (length $l$) we use a local orthogonal coordinate system on each triangle edge with $\vec{l}$ the vector of length $l$ along the edge in counter-clockwise orientation, $\vec{n}$ the outward normal of length $l$ and $\grad \phi$ pointing inside the plane. We obtain relations
\begin{align}
  \vec{l} \times \vec{n} &= l^{2} R \grad \phi, \\
  \vec{n} \times R \grad \phi &= \vec{l}, \\
  R \grad \phi \times \vec{l} &= \vec{n}.
\end{align}
This is illustrated for a small sector on one loop in \cref{fig:local_coordinates}.

\begin{figure}[bth]
  \centering
  \begin{tikzpicture}[> = Stealth, line join = round, shape = circle, scale = 1]  % , font = \small
  \tikzmath{
    real \ri, \ro;
    int \sector, \overshoot, \arcstart, \arcend;
    \ri = 10; % 6;
    \ro = 12.5; % 7.5;
    \sector = 30;
    \overshoot = 3;
    \arcstart = 90 - \sector - \overshoot;
    \arcend = 90 + \sector + \overshoot;
  };
  \draw (\arcstart:\ri) arc [start angle = \arcstart, end angle = \arcend, radius = \ri, very thin, color = gray];
  \draw (\arcstart:\ro) arc [start angle = \arcstart, end angle = \arcend, radius = \ro, very thin, color = gray];
  \coordinate (ri1) at (60:\ri); \coordinate (ro1) at (60:\ro);
  \coordinate (ri2) at (75:\ri); \coordinate (ro2) at (75:\ro);
  \coordinate (ri3) at (90:\ri); \coordinate (ro3) at (90:\ro);
  \coordinate (ri4) at (105:\ri); \coordinate (ro4) at (105:\ro);
  \coordinate (ri5) at (120:\ri); \coordinate (ro5) at (120:\ro);
  \draw (ri1) -- (ri2) -- (ri3) -- (ri4) -- (ri5);
  \draw (ro1) -- (ro2) -- (ro3) -- (ro4) -- (ro5);
  \draw (ri1) -- (ro1) -- (ri2) -- (ro2) -- (ri3) -- (ro3) -- (ri4) -- (ro4) -- (ri5) -- (ro5);
  % triangle 2
  \draw[thick, ->] (ri2) -- node [very near end, auto, swap] {$\vec{l}_{\inw}$} (ro1);
  \draw[thick, ->] (ro1) -- node [very near end, auto, swap] {$\vec{l}_{\fs}$} (ro2);
  \draw[thick, ->] (ro2) -- node [very near end, auto, swap] {$\vec{l}_{\out}$} (ri2);
  \coordinate (Gi2) at ($ (ri2) ! 0.5 ! (ro1) $);
  \coordinate (Gf2) at ($ (ro1) ! 0.5 ! (ro2) $);
  \coordinate (Go2) at ($ (ro2) ! 0.5 ! (ri2) $);
  \draw[thick, ->] (Gi2) -- node [near end, auto, swap] {$\vec{n}_{\inw}$} ($ (Gi2) ! 2 ! 90:(ri2) $);
  \draw[thick, ->] (Gf2) -- node [very near end, auto, swap] {$\vec{n}_{\fs}$} ($ (Gf2) ! 2 ! 90:(ro1) $);
  \draw[thick, ->] (Go2) -- node [very near end, auto, swap] {$\vec{n}_{\out}$} ($ (Go2) ! 2 ! 90:(ro2) $);
  % triangle 7
  \draw[thick, ->] (ro4) -- node [very near end, auto, swap] {$\vec{l}_{\out}$} (ri5);
  \draw[thick, ->] (ri5) -- node [very near end, auto, swap] {$\vec{l}_{\fs}$} (ri4);
  \draw[thick, ->] (ri4) -- node [very near end, auto, swap] {$\vec{l}_{\inw}$} (ro4);
  \coordinate (Go7) at ($ (ro4) ! 0.5 ! (ri5) $);
  \coordinate (Gf7) at ($ (ri5) ! 0.5 ! (ri4) $);
  \coordinate (Gi7) at ($ (ri4) ! 0.5 ! (ro4) $);
  \draw[thick, ->] (Go7) -- node [very near end, auto, swap] {$\vec{n}_{\out}$} ($ (Go7) ! 2 ! 90:(ro4) $);
  \draw[thick, ->] (Gf7) -- node [near end, auto, swap] {$\vec{n}_{\fs}$} ($ (Gf7) ! 2 ! 90:(ri5) $);
  \draw[thick, ->] (Gi7) -- node [very near end, auto] {$\vec{n}_{\inw}$} ($ (Gi7) ! 2 ! 90:(ri4) $);
  % toroidal basis vectors
  \node (O2) at ($ 0.5*(ri2) + 0.25*(ro1) + 0.25*(ro2) $) {$\otimes$};
  \node (O7) at ($ 0.5*(ro4) + 0.25*(ri5) + 0.25*(ri4) $) {$\otimes$};
  \node[circle, above right] at (O2) {$\nabla \phi$};
  \node[circle, below] at (O7) {$\nabla \phi$};
\end{tikzpicture}

  \caption{Schematic of the local coordinate system.}
  \label{fig:local_coordinates}
\end{figure}

\clearpage
\section{Pre-processing of input data}
\label{sec:inputs}

The equilibrium field $\vec{B}_{0}$ is assumed to be available in GEQDSK format, shortly summarized in \cite{Lao97}. It consists of data points given on a rectangular grid which are fitted from measurements to the \textsc{Grad}--\textsc{Shafranov} equation. Parts of NEO-2 code import such a file, generate a field-aligned grid as described in \cref{sec:grid} and supply values of $\vec{B}_{0}$ and its partial derivatives on arbitrary $R, Z$ coordinates by interpolation with splines of fifth order in $R$ and $Z$. In the implementation of NEO-EQ, the necessary values on edge midpoints are cached once before calculations starts as to avoid repeated function calls to spline interpolation.

[\Bvac \ldots]

[grid implementation \ldots]

\subsection{Toroidal unperturbed current}

Since $\vec{B}_{0}$ and $p_{0}$ are directly available as input data, but $\vec{J}_{0}$ is not, the latter will be derived below from \cref{eq:mhd-gen}, the condition of divergence-freeness and symmetry considerations.

We take a cross-product of \cref{eq:mhd-gen} by $\vec{B}_{0}$:
\begin{align}
  \vec{B}_{0} \times \left( \vec{J}_{0} \times \vec{B}_{0} \right) &= B_{0}^{2} \vec{J}_{0} - (\vec{B}_{0} \cdot \vec{J}_{0}) \vec{B}_{0} \nonumber \\
  &= B_{0}^{2} (\vec{J}_{0} - J_{0}^{\parallel} \vec{h}_{0}) \nonumber \\
  &= B_{0}^{2} \vec{J}_{0}^{\perp}.
\end{align}
Therefore
\begin{gather}
  \vec{J}_{0}^{\perp} = \frac{-c \grad p_{0} \times \vec{B}_{0}}{B_{0}^{2}},
\end{gather}
which is the diamagnetic current density. For the parallel current density we use
\begin{align}
 0 = \divg \vec{J}_{0} &= \divg \vec{J}_{0}^{\perp} + \divg (J_{0}^{\parallel} \vec{h}_{0}) \nonumber \\
 &= -c \divg \frac{\grad p_{0} \times \vec{B}_{0}}{B_{0}^{2}} + \vec{B}_{0} \cdot \grad \frac{J_{0}^{\parallel}}{B_{0}}.
\end{align}
In straight-field line magnetic flux coordinates $(\rho, \theta, \zeta) = (-\psi, \theta, \frac{\pi}{2} - \phi)$ and Jacobian $\sqrt{g}$, the divergence of the diamagnetic current is
\begin{align}
  \divg \vec{J}_{0}^{\perp} &= -\frac{c}{\sqrt{g}} \pd{x^{k}} \left[ \frac{\sqrt{g}}{B_{0}^{2}} \left( \grad p_{0} \times \vec{B}_{0} \right)^{k} \right] \nonumber \\
  &= -\frac{c}{\sqrt{g}} \pd{x^{k}} \left( \frac{\sqrt{g}}{B_{0}^{2}} \frac{\varepsilon^{ijk}}{\sqrt{g}} \pd[p_{0}]{x^{i}} B_{0 j} \right) \nonumber \\
  &= \frac{c p_{0}' (\rho) B_{0 \zeta}}{\sqrt{g}} \pd{\theta} \frac{1}{B_{0}^{2}},
\end{align}
since $p_{0}$ and $B_{0 \zeta}$ are constant on a flux surface, $\pd[p_{0}]{\theta} = 0$ and due to axisymmetry $\pd{\zeta} = 0$. The divergence of the parallel current is
\begin{gather}
  \divg(J_{0}^{\parallel} \vec{h}_{0}) = \vec{B}_{0} \cdot \grad \frac{J_{0}^{\parallel}}{B_{0}} = B_{0}^{\theta} \pd{\theta} \frac{J_{0}^{\parallel}}{B_{0}}.
\end{gather}
With $\sqrt{g} B_{0}^{\theta} = -\psi'(\rho)$ as a flux surface quantity, there are no dependencies of $\theta$ in front of the derivatives:
\begin{gather}
  c p_{0}' (\rho) B_{0 \zeta} \pd{\theta} \frac{1}{B_{0}^{2}} - \psi'(\rho) \pd{\theta} \frac{J_{0}^{\parallel}}{B_{0}} = 0.
\end{gather}
Direct integration and a change of variables and notation as in
\begin{gather}
  \frac{p_{0}'(\rho)}{\psi'(\rho)} = \frac{\pd[p_{0}]{\rho}}{\pd[\psi]{\rho}} = \pd[p_{0}]{\psi} = p_{0}'(\psi)
\end{gather}
yields
\begin{gather}
  \frac{-c p_{0}' (\psi) B_{0 \phi}}{B_{0}^{2}} - \frac{J_{0}^{\parallel}}{B_{0}} = C(\psi). \label{eq:j0parallel_general}
\end{gather}
With the extra condition of the flux-surface average $\left\langle J_{0}^{\parallel} B_{0} \right\rangle = 0$ for testing without bootstrap current and the assumption $\left\langle B_{0 \phi} \right\rangle \approx B_{0 \phi}$, we obtain
\begin{gather}
  -c p_{0}'(\psi) B_{0 \phi} = C(\psi) \left\langle B_{0}^{2} \right\rangle.
\end{gather}
In general, 
\begin{gather}
  C(\psi) = -\frac{c p_{0}'(\psi) B_{0 \phi}}{\left\langle B_{0}^{2} \right\rangle} D(\psi),
\end{gather}
with $D(\psi)$ set to 1 for now and modified for the more general case $\left\langle J_{0}^{\parallel} B_{0} \right\rangle \not\equiv 0$. Inserting this back into \cref{eq:j0parallel_general} yields
\begin{gather}
  J_{0}^{\parallel} = \frac{c p_{0}'(\psi) B_{0 \phi}}{B_{0}} \left( \frac{B_{0}^{2}}{\left\langle B_{0}^{2} \right\rangle} D(\psi) - 1 \right).
\end{gather}

For the unperturbed toroidal current density we have
\begin{gather}
  J_{0}^{\phi} = J_{0}^{\parallel} h_{0}^{\phi} + \vec{J}_{0}^{\perp} \cdot \grad \phi,
\end{gather}
where
\begin{align}
  J_{0}^{\parallel} h_{0}^{\phi} &= \frac{1}{B_{0}} c p_{0}'(\psi) B_{0 \phi} \frac{B_{0}^{\phi}}{B_{0}} \left( \frac{B_{0}^{2}}{\left\langle B_{0}^{2} \right\rangle} D(\psi) - 1 \right) \nonumber \\
  &= c p_{0}'(\psi) \frac{\left( B_{0}^{\tor} \right)^{2}}{B_0^2} \left( \frac{B_{0}^{2}}{\left\langle B_{0}^{2} \right\rangle} D(\psi) - 1 \right) \nonumber \\
  &= c p_{0}'(\psi) \left( B_{0}^{\tor} \right)^{2} \left( \frac{D(\psi)}{\left\langle B_{0}^{2} \right\rangle} - \frac{1}{B_{0}^{2}} \right).
\end{align}
and
\begin{align}
  \vec{J}_{0}^{\perp} \cdot \grad \phi &= \frac{-c p_{0}'(\psi) \grad \phi \cdot (\grad \psi \times \vec{B}_{0})}{B_{0}^{2}} \nonumber \\
  &= \frac{-c p_{0}'(\psi) \vec{B}_{0} \cdot (\grad \phi \times \grad \psi)}{B_{0}^{2}} \\
  &= \frac{c p_{0}'(\psi)}{B_{0}^{2}} \vec{B}_{0}^{\pol} \cdot \vec{B}_{0} \nonumber \\
  &= c p_{0}'(\psi) \frac{\left( B_{0}^{\pol} \right)^{2}}{B_{0}^{2}}.
\end{align}
It follows that
\begin{gather}
  J_{0}^{\phi} = c p_{0}'(\psi) \left( \frac{\left( B_{0}^{\tor} \right)^{2}}{\left\langle B_{0}^{2} \right\rangle} D(\psi) + \frac{\left( B_{0}^{\pol} \right)^2 - \left( B_{0}^{\tor} \right)^2}{B_{0}^{2}} \right).
\end{gather}

\subsection{Safety factor}
\label{sec:safety_factor}

With the usual toroidal flux angle $\zeta = \frac{\pi}{2} - \phi$ used by \cite{dHaeseleer91}, the toroidal flux $\psi_{\tor}$ is given by
\begin{gather}
  \psi_{\tor} = \frac{1}{(2 \pi)^{2}} \int \diff V \, \vec{B} \cdot \grad \zeta = -\frac{1}{(2 \pi)^{2}} \int \diff V \, \vec{B} \cdot \grad \phi = -\frac{1}{2 \pi} \int \diff R \, \diff Z \, R B^{\phi} = -\frac{1}{2 \pi} \int \diff R \, \diff Z \, B_{(\phi)}.
\end{gather}
The safety factor is then given by
\begin{gather*}
  q = \frac{\psi_{\tor}' (\rho)}{\psi_{\pol}' (\rho)} = \frac{\psi_{\tor}' (\sigma_{\pol} \psi)}{\psi_{\pol}' (\sigma_{\pol} \psi)} = \td[\psi_{\tor}]{\psi} = -\frac{1}{2 \pi} \td{\psi} \int \diff R \, \diff Z \, B_{(\phi)}.
\end{gather*}
This quantity can be evaluated numerically by adding up $B_{(\phi)}$ inside the volume between two flux surfaces and dividing by the the difference in $\psi$:
\begin{gather}
  q \approx -\frac{1}{2 \pi \symup{\Delta} \psi} \sum_{k} B_{(\phi)} (\Omega^{(k)}) S_{\Omega^{(k)}},
\end{gather}
where the sum is taken over all triangles $\Omega$ inside a triangle strip, and $S_{\Omega}$ is the respective triangle surface area.

\clearpage
\section{Linearized MHD force balance}

[\ldots]

\subsection{Pressure perturbation}
\label{sec:compute_presn}

Multiplying \cref{eq:mhd-phi} by $\vec{B}_{0}$ yields
\begin{align}
  c \vec{B}_{0} \cdot \grad p_{n} + \im n c p_{n} \vec{B}_{0} \cdot \grad \phi &= -\vec{B}_{n} \cdot (\vec{J}_{0} \times \vec{B}_{0}) = -c \vec{B}_{n} \cdot \grad p_{0} \nonumber \\
  \vec{B}_{0}^{\pol} \cdot \grad p_{n} + \im n p_{n} B_{0}^{\phi} &= -B_{n}^{\psi} p_{0}'(\psi). \label{eq:pn}
\end{align}
To solve this equation on one flux surface, we use a lowest-order finite difference method. Nodes are indexed by superscript $(k)$ and $\vec{r}^{(k)}$ is the position of node $(k)$, whereas $\vec{l}_{\fs}^{(k)} = \vec{r}^{(k+1)} - \vec{r}^{(k)}$ is the counter-clockwise vector between nodes on edge $\Gamma_{\fs}^{(k)}$. $\grad p_{n}$ is approximated at the midpoint of edge $\Gamma_{\fs}^{(k)}$ as finite difference of $p_{n}$ at nodes $(k)$ and $(k+1)$. $p_{n}$ is accordingly approximated at the midpoint as the arithmetic mean of the values at these nodes. With a shorthand $p_{n}^{(k)} = p_{n} (\vec{r}^{(k)})$ for the degrees of freedom we get
\begin{gather}
  \vec{B}_{0}^{\pol} (\Gamma_{\fs}^{(k)}) \cdot \frac{\vec{l}_{\fs}^{(k)}}{l_{\fs}^{(k)}} \frac{p_{n}^{(k+1)} - p_{n}^{(k)}}{l_{\fs}^{(k)}} + \im n B_{0}^{\phi} (\Gamma_{\fs}^{(k)}) \frac{p_{n}^{(k+1)} + p_{n}^{(k)}}{2} = -\pd[p_{0}]{\psi} (\Gamma_{\fs}^{(k)}) B_{n}^{\psi} (\Gamma_{\fs}^{(k)}),
\end{gather}
where a unit vector along the edge is used to get the correct sign for the gradient in the direction of the poloidal magnetic field. Reordering in terms of the unknowns yields
\begin{gather}
  (b_{k} + a_{k}) p_{n}^{(k+1)} + (b_{k} - a_{k}) p_{n}^{(k)} = s_{k}
\end{gather}
with
\begin{align}
  a_{k} &= \vec{B}_{0}^{\pol} (\Gamma_{\fs}^{(k)}) \cdot \frac{\hat{\vec{l}}_{\fs}^{(k)}}{l_{\fs}^{(k)}}, \\
  b_{k} &= \frac{\im n B_{0}^{\phi} (\Gamma_{\fs}^{(k)})}{2}, \\
  s_{k} &= -\pd[p_{0}]{\psi} (\Gamma_{\fs}^{(k)}) B_{n}^{\psi} (\Gamma_{\fs}^{(k)}).
\end{align}
In matrix form this scheme is written as
\begin{gather}
  K_{jk} p_{n}^{(k)} = s_{j},
\end{gather}
where the elements of the stiffness matrix $\hat{K}$ are
\begin{gather}
  K_{jk} = (b_{j} + a_{j}) \delta_{j-1, k} + (b_{j} - a_{j}) \delta_{jk}.
\end{gather}
Note that for $N$ nodes with periodic boundary conditions $p_{n}^{(0)} = p_{n}^{(N)}$, indices "wrap around", resulting in the following shape for the stiffness matrix:
\begin{gather*}
  \hat{K} = \begin{pmatrix}
    b_{1} - a_{1} &  b_{1} + a_{1} &        0       & \hdots &    0   \\
           0       & b_{2} - a_{2} &  b_{2} + a_{2} & \hdots &    0   \\
           0       &        0       & b_{3} - a_{3} & \hdots &    0   \\
        \vdots     &     \vdots     &     \vdots     & \ddots & \vdots \\
     b_{N} + a_{N} &        0       &        0       & \hdots & b_{N} - a_{N}
  \end{pmatrix}.
\end{gather*}

\subsection{Current perturbation}
\label{sec:compute_currn}

\begin{figure}[bth]
  \centering
  \begin{tikzpicture}[> = Stealth, line join = round, shape = circle, scale = 1] % , font = \small
  \tikzmath{
    real \ri, \ro;
    int \sector, \overshoot, \arcstart, \arcend;
    \ri = 10; % 6;
    \ro = 12.5; % 7.5;
    \sector = 30;
    \overshoot = 3;
    \arcstart = 90 - \sector - \overshoot;
    \arcend = 90 + \sector + \overshoot;
  };
  \draw (\arcstart:\ri) arc [start angle = \arcstart, end angle = \arcend, radius = \ri, very thin, color = gray];
  \draw (\arcstart:\ro) arc [start angle = \arcstart, end angle = \arcend, radius = \ro, very thin, color = gray];
  \coordinate (ri1) at (60:\ri); \coordinate (ro1) at (60:\ro);
  \coordinate (ri2) at (75:\ri); \coordinate (ro2) at (75:\ro);
  \coordinate (ri3) at (90:\ri); \coordinate (ro3) at (90:\ro);
  \coordinate (ri4) at (105:\ri); \coordinate (ro4) at (105:\ro);
  \coordinate (ri5) at (120:\ri); \coordinate (ro5) at (120:\ro);
  \draw (ri1) -- (ri2) -- (ri3) -- (ri4) -- (ri5);
  \draw (ro1) -- (ro2) -- (ro3) -- (ro4) -- (ro5);
  \draw (ri1) -- (ro1) -- (ri2) -- (ro2) -- (ri3) -- (ro3) -- (ri4) -- (ro4) -- (ri5) -- (ro5);
  % triangle 2
  \coordinate (Gi2) at ($ (ri2) ! 0.5 ! (ro1) $);
  \coordinate (Gf2) at ($ (ro1) ! 0.5 ! (ro2) $);
  \coordinate (Go2) at ($ (ro2) ! 0.5 ! (ri2) $);
  \draw[thick, ->] (Gi2) -- node [very near end, right, swap] {$I^{(2)}_{\inw}$} ($ (Gi2) ! 0.8 ! 90:(ri2) $);
  \draw[thick, ->] (Gf2) -- node [near end, right, swap] {$I^{(2)}_{\fs}$} ($ (Gf2) ! 0.8 ! 90:(ro1) $);
  \draw[thick, ->] (Go2) -- node [very near end, auto] {$I^{(2)}_{\out}$} ($ (Go2) ! 0.8 ! 90:(ro2) $);
  % triangle 7
  \coordinate (Go7) at ($ (ro4) ! 0.5 ! (ri5) $);
  \coordinate (Gf7) at ($ (ri5) ! 0.5 ! (ri4) $);
  \coordinate (Gi7) at ($ (ri4) ! 0.5 ! (ro4) $);
  \draw[thick, ->] (Go7) -- node [near end, below] {$I^{(7)}_{\out}$} ($ (Go7) ! 0.8 ! 90:(ro4) $);
  \draw[thick, ->] (Gf7) -- node [very near end, auto] {$I^{(7)}_{\fs}$} ($ (Gf7) ! 0.8 ! 90:(ri5) $);
  \draw[thick, ->] (Gi7) -- node [near end, above right] {$I^{(7)}_{\inw}$} ($ (Gi7) ! 0.8 ! 90:(ri4) $);
  % toroidal currents
  \node[label={[label distance=-5mm]90:$I^{(2)}_{\phi}$}] (O2) at ($ 0.5*(ri2) + 0.25*(ro1) + 0.25*(ro2) $) {$\otimes$};
  \node[label={[label distance=-5mm]270:$I^{(7)}_{\phi}$}] (O7) at ($ 0.5*(ro4) + 0.25*(ri5) + 0.25*(ri4) $) {$\otimes$};
  % degree of freedom
  \coordinate (Gi5) at ($ (ri3) ! 0.5 ! (ro3) $);
  \draw[red, ->] ($ (Gi5) + (0.5, 0) $) -- node [at start, above, red] {$I^{(5)}$} ($ (Gi5) - (0.5, 0) $);
\end{tikzpicture}

  \caption{Schematic of the degrees of freedom and unknowns in the calculation of currents.}
  \label{fig:current_perturbation}
\end{figure}

Multiplying \cref{eq:divfree-phi} by $R$ yields
\begin{gather}
  \pd{x^{k}} (R J_{n}^{k}) + \im n R J_{n}^{\phi} = 0.
\end{gather}
Using the divergence theorem this can also be written in integral form in a specific triangular mesh element $\Omega^{(k)}$ as
\begin{gather}
  \oint_{\partial \Omega^{(k)}} \diff l \, R \vec{J}_{n} \cdot \hat{\vec{n}} + \im n \int_{\Omega^{(k)}} \diff R \, \diff Z \, R J_{n}^{\phi} = 0.
\end{gather}
Here the first integral is performed over the 1-dimensional element boundary $\partial \Omega^{(k)} = \Gamma^{(k)}$. The first term is split into three contributions,
\begin{gather}
  \oint_{\partial \Omega^{(k)}} \diff l \, R \vec{J}_{n} \cdot \hat{\vec{n}} = \int_{\Gamma_{\inw}^{(k)}, \Gamma_{\out}^{(k)}} \diff l \, R \vec{J}_{n} \cdot \hat{\vec{n}} + \int_{\Gamma_{\fs}^{(k)}} \diff l \, R \vec{J}_{n} \cdot \hat{\vec{n}},
\end{gather}
where edge \fs\ is tangential to an adjacent flux surface and edges \inw\ and \out\ are not. For the grid it is important to distinguish two types of triangles, starting from the second loop outside the magnetic axis, see \cref{fig:grid}.

For the computation of toroidal $J_{n}^{\phi}$ in the element volume we start from \cref{eq:mhd-phi} with
\begin{gather}
  \vec{J}_{n} \times \vec{B}_{0} = c (\grad p_{n} + \im n p_{n} \grad \phi) - \vec{J}_{0} \times \vec{B}_{n} \label{eq:jnxB0}
\end{gather}
and
\begin{gather}
  \vec{B}_{0} = \grad \psi \times \grad \phi + B_{0 \phi} \grad \phi.
\end{gather}
Taking a scalar product of $\vec{l}$ with \cref{eq:jnxB0} on some edge yields
\begin{align}
  \vec{l} \cdot (\vec{J}_{n} \times \vec{B}_{0}) &= \vec{l} \cdot (\vec{J}_{n} \times (\grad \psi \times \grad \phi + B_{0 \phi} \grad \phi)) \nonumber \\
  &= \vec{J}_{n} \cdot ((\grad \psi \times \grad \phi) \times \vec{l} + B_{0 \phi} \grad \phi \times \vec{l}) \nonumber \\
  &= \vec{J}_{n} \cdot ((\grad \psi \times \grad \phi) \times \vec{l} + B_{0 \phi} \vec{n} / R) \nonumber \\
  &= \vec{J}_{n} \cdot \left( (\vec{l} \cdot \grad \psi) \grad \phi + B_{0 (\phi)} \vec{n} \right) \nonumber \\
  &= (\vec{l} \cdot \grad \psi) J_{n}^{\phi} + B_{0 (\phi)} \vec{J}_{n}^{\pol} \cdot \vec{n}.
\end{align}
Further,
\begin{gather}
  \vec{l} \cdot \grad \psi = R \grad \psi \cdot \left( \vec{n} \times \grad \phi \right) = -R \vec{n} \cdot \vec{B}_{0}^{\pol}.
\end{gather}
Finally,
\begin{gather}
  \vec{l} \cdot (\vec{J}_{n} \times \vec{B}_{0}) = B_{0 (\phi)} \vec{J}_{n}^{\pol} \cdot \vec{n} - R J_{n}^{\phi} \vec{n} \cdot \vec{B}_{0}^{\pol}.
\end{gather}
The right-hand side of \cref{eq:jnxB0} yields:
\begin{align}
  \vec{l} \cdot (\grad p_{n} + \im n p_{n} \grad \phi) &= \vec{l} \cdot \grad p_{n} \\
  \vec{l} \cdot (\vec{J}_{0} \times \vec{B}_{n}) &= \vec{l} \cdot (B_{n \phi} \vec{J}_{0}^{\pol} \times \grad \phi + J_{0 \phi} \grad \phi \times \vec{B}_{n}^{\pol}).
\end{align}
We use the fact that $\grad p_{0}$ is parallel to $\grad \psi$, so the cross product in the equilibrium is purely radial,
\begin{align}
  \vec{J}_{0} \times \vec{B}_{0} &= c \grad p_{0} \nonumber \\
  &= \vec{J}_{0}^{\pol} \times (B_{0 \phi} \grad \phi) + J_{0 \phi} \grad \phi \times (\grad \psi \times \grad \phi) \nonumber \\
  &= \vec{J}_{0}^{\pol} \times (B_{0 \phi} \grad \phi) + \frac{J_{0 \phi}}{R^{2}} \grad \psi.
\end{align}
Thus
\begin{gather}
  \vec{J}_{0}^{\pol} \times \grad \phi = \frac{1}{B_{0 \phi}} \left( c \grad p_{0} - \frac{J_{0 \phi}}{R^{2}} \grad \psi \right).
\end{gather}
Also
\begin{align}
  \vec{l} \cdot (\grad \phi \times \vec{B}_{n}^{\pol}) &= \vec{B}_{n}^{\pol} \cdot (\vec{l} \times \grad \phi) \nonumber \\
  &= -\frac{1}{R} \vec{B}_{n}^{\pol} \cdot \vec{n}.
\end{align}
Finally
\begin{gather}
  (\vec{l} \cdot \grad \psi) J_{n}^{\phi} + B_{0 (\phi)} \vec{J}_{n}^{\pol} \cdot \vec{n} = c \vec{l} \cdot \grad p_{n} - \frac{B_{n \phi}}{B_{0 \phi}} \left( c \vec{l} \cdot \grad p_{0} - \frac{J_{0 \phi}}{R^{2}} \vec{l} \cdot \grad \psi \right) + \frac{J_{0 \phi}}{R} \vec{B}_{n}^{\pol} \cdot \vec{n} \label{eq:jnphi-jnpol}
\end{gather}
This term is only meaningful to obtain $J_{n}^{\phi}$ on edges where $\vec{l} \cdot \grad \psi \neq 0$ (\inw\ and \out\ in \cref{fig:grid}) and we obtain
\begin{gather}
  R J_{n}^{\phi} = J_{n (\phi)} = -\frac{B_{0 \phi}}{\vec{l} \cdot \grad \psi} \vec{J}_{n}^{\pol} \cdot \vec{n} + \frac{c R}{\vec{l} \cdot \grad \psi} \left( \vec{l} \cdot \grad p_{n} - \frac{B_{n \phi}}{B_{0 \phi}} \vec{l} \cdot \grad p_{0} \right) + J_{0 \phi} \left( \frac{B_{n \phi}}{R B_{0 \phi}} + \frac{\vec{B}_{n}^{\pol} \cdot \vec{n}}{\vec{l} \cdot \grad \psi} \right). \label{eq:jnphi}
\end{gather}
In the implementation, an average over edge \inw\ and \out\ will be taken for this quantity.

On edge \fs, no connection between $J_{n}^{\phi}$ and $\vec{J}_{n}^{\pol} \cdot \vec{n}$ can be made, but the latter expression can be given in terms of already knwown quantities:
\begin{gather}
  R \vec{J}_{n}^{\pol} \cdot \vec{n} = \frac{c R}{B_{0 (\phi)}} \vec{l} \cdot \grad p_{n} + \frac{J_{0 \phi}}{B_{0 (\phi)}} \vec{B}_{n}^{\pol} \cdot \vec{n}. \label{eq:If}
\end{gather}

We start with
\begin{gather}
  I_{\inw} + I_{\out} + \im n \int_{\Omega} R J_{n}^{\phi} \diff S = -I_{\fs},
\end{gather}
where the notation for currents through edges, weighted by $R$, is
\begin{gather}
  I_{k} = \int_{\Gamma_{k}} R \vec{J}_{n}^{\pol} \cdot \hat{\vec{n}} \, \diff l \approx R (\Gamma_{k}) \vec{J}_{n}^{\pol} (\Gamma_{k}) \cdot \vec{n}_{k},
\end{gather}
where values are taken at the midpoint of edge $\Gamma_{k}$ and assumed to be constant along that edge. $I_{\fs}$ is already known from \cref{eq:If} and therefore acts as a source on the right-hand side. The remaining currents $I_{\inw}$ and $I_{\out}$ are taken as unknowns and appear also in the last term of the left-hand side as
\begin{align}
  \im n \int_{\Omega} R J_{n}^{\phi} \, \diff S & \approx \im n S_{\Omega} \frac{R (\Gamma_{\inw}) J_{n}^{\phi} (\Gamma_{\inw}) + R (\Gamma_{\out}) J_{n}^{\phi} (\Gamma_{\out})}{2} \nonumber \\
  &= -\frac{\im n S_{\Omega}}{2} \left( \frac{B_{0 (\phi)} (\Gamma_{\inw})}{\vec{l}_{\inw} \cdot \grad \psi (\Gamma_{\inw})} I_{\inw} + \frac{B_{0 (\phi)} (\Gamma_{\out})}{\vec{l}_{\out} \cdot \grad \psi (\Gamma_{\out})} I_{\out} + \dotsb \right),
\end{align}
where $S_{\Omega}$ is the triangle surface area. In the approximation above, a term $R (\Gamma_{\fs}) J_{n}^{\phi} (\Gamma_{\fs})$ is neglected because within one loop of triangles in \cref{fig:grid1}, edge \fs\ alternates between the inner and outer flux surface and this "oscillation" of sample points would carry over to the approximation values. Furthermore, $R (\Gamma_{\fs}) J_{n}^{\phi} (\Gamma_{\fs})$ can't be reformulated in terms of $R (\Gamma_{\fs}) \vec{J}_{n}^{\pol} (\Gamma_{\fs})$, introducing another unknown. This would lead to an overdetermined set of equations, possibly violating divergence-freeness. Instead, $I_{(\phi)}$ is approximated for each triangle by
\begin{gather}
  I_{(\phi)} = \int_{\Omega} R J_{n}^{\phi} \, \diff S \approx S_{\Omega} R (\Omega) J_{n}^{\phi} (\Omega).
\end{gather}
The remaining terms are moved to the right-hand-side as sources $s$, so the discretized equation in each triangle $\Omega$ is
\begin{gather}
  \left( 1 - \frac{\im n S_{\Omega}}{2} \frac{B_{0 (\phi)} (\Gamma_{\inw})}{\vec{l}_{\inw} \cdot \grad \psi (\Gamma_{\inw})} \right) I_{\inw} + \left( 1 - \frac{\im n S_{\Omega}}{2} \frac{B_{0 (\phi)} (\Gamma_{\out})}{\vec{l}_{\out} \cdot \grad \psi (\Gamma_{\out})} \right) I_{\out} = s. \label{eq:Ii-Io}
\end{gather}
The source term is given by
\begin{gather}
  s = -I_{\fs} - \frac{\im n S_{\Omega}}{2} \sum_{k = \inw, \out} \frac{c R}{\vec{l}_{k} \cdot \grad \psi} \left( \vec{l}_{k} \cdot \grad p_{n} - \frac{B_{n \phi}}{B_{0 \phi}} \vec{l}_{k} \cdot \grad p_{0} \right) + J_{0 \phi} \left( \frac{B_{n \phi}}{R B_{0 \phi}} + \frac{\vec{B}_{n}^{\pol} \cdot \vec{n}}{\vec{l}_{k} \cdot \grad \psi} \right).
\end{gather}
The directional derivatives $\vec{l}_{k} \cdot \grad \psi$ are approximated by a difference quotient with values taken at the nodes (for indexing see \cref{fig:grid}),
\begin{align}
  \vec{l}_{\inw} \cdot \grad \psi = l_{\inw} \hat{\vec{l}}_{\inw} \cdot \grad \psi = l_{\inw} \pd[\psi]{\vec{l}_{\inw}} & \approx l_{\inw} \frac{-\symup{\Delta} \psi}{l_{\inw}} = -\symup{\Delta} \psi, \\
  \vec{l}_{\out} \cdot \grad \psi = l_{\out} \hat{\vec{l}}_{\out} \cdot \grad \psi = l_{\out} \pd[\psi]{\vec{l}_{\out}} & \approx l_{\out} \frac{\symup{\Delta} \psi}{l_{\out}} = \symup{\Delta} \psi.
\end{align}
Note that $\psi \vert_{\vfs} \neq \psi \vert_{\vinw} = \psi \vert_{\vout}$ for each triangle because values are taken from the nodes at adjacent flux surfaces. Thus $\symup{\Delta} \psi$ is identical for the entire loop, but signs change depending on the direction of $\vec{l}_{k}$, alternating between triangles. It takes a positive value for $\vec{l}_{k}$ pointing in the radially inward direction. The same logic applies to $\vec{l} \cdot \grad p_{0}$ terms.

For the global indexing scheme, we call the ingoing current into triangle $(k)$ counted in clockwise direction $I^{(k)}$. In triangle $(k)$, this is equal to $I_{\inw} = -I^{(k)}$ and $I_{\out} = I^{(k+1)}$. The matrix form of \cref{eq:Ii-Io} is then
\begin{gather}
  K_{jk} I^{(k)} = s_{j},
\end{gather}
where the elements of the stiffness matrix $\hat{K}$ are
\begin{gather}
  K_{jk} = \left( -1 - \frac{\im n S_{\Omega^{(j)}}}{2} \frac{B_{0 (\phi)} (\Gamma_{\inw}^{(j)})}{\symup{\Delta} \psi} \right) \delta_{jk} + \left( 1 - \frac{\im n S_{\Omega^{(j)}}}{2} \frac{B_{0 (\phi)} (\Gamma_{\out}^{(j)})}{\symup{\Delta} \psi} \right) \delta_{j+1, k}.
\end{gather}

\subsection{Magnetic field perturbation}
\label{sec:compute_Bn}

[\ldots]

The weak form used in \texttt{FreeFEM++} is given by  % rework!
\begin{gather}
  \int_{\Omega} \left ( \polCurl \vec{w} R \polCurl \vec{a} + \frac{1}{R} n^{2} w_{k} a_{k} \right ) \, \diff \Omega + \frac{4 \pi}{c} \int_{\Omega} w_{k} j_{k} = 0.
\end{gather}

[For details see \cite{Albert19,Biro15,Hecht12,Jin02}.]

\clearpage
\section{Construction of test cases and analysis of results}

Along with comparison to results from NEO-2, special test cases altering different components of the calculation are considered. In \cref{sec:nonres}, $\Bvac$ is not taken from \textsc{Biot}--\textsc{Savart} calculations of currents in external coils as per \cref{eq:biot-savart,sec:inputs}, but constructed directly to avoid resonances, at least in the first iteration step. \Cref{sec:analytical} instead looks at the analytical solution in the cylindrical limit on a circular cross-section obtained from KilCa, effectively changing the grid geometry.

\subsection{Generating a non-resonant test field}
\label{sec:nonres}

The axisymmetric equilibrium field $\vec{B}_{0}$ lies on nested flux surfaces $\psi = \text{const.}$, meaning
\begin{gather*}
  \vec{B}_{0} \cdot \grad \psi = B_{0}^{\psi} = 0.
\end{gather*}
If the perturbed field shall still lie on distorted, but not broken, flux surfaces (non-resonant, without magnetic islands), a new flux surface label $\psi + \delta \psi$ must exist fulfilling
\begin{gather*}
  (\vec{B}_{0} + \Bpert) \cdot \grad (\psi + \delta \psi) = 0
\end{gather*}
for the perturbed magnetic field $\vec{B}_{0} + \Bpert$, where $\delta \psi$ remains small and continuous within the plasma. In that case we can use a linear order expansion
\begin{gather*}
  \underbrace{\vec{B}_{0} \cdot \grad \psi}_{= 0} + \vec{B}_{0} \cdot \grad \delta \psi + \Bpert \cdot \grad \psi + \mathcal{O}(\varepsilon^{2}) = 0,
\end{gather*}
or in coordinate form with symmetry flux coordinates $(\rho, \theta, \zeta)$,
\begin{gather*}
  B_{0}^{\theta} \pd[\delta \psi]{\theta} + B_{0}^{\zeta} \pd[\delta \psi]{\zeta} + \delta B^{\rho} \psi' (\rho) = 0.
\end{gather*}
Here we have used $\grad \psi = \psi' (\rho) \grad \rho$. Further, $\psi' (\rho) = \sigma_{\pol}$. With safety factor
\begin{gather*}
  q = \frac{B_{0}^{\zeta}}{B_{0}^{\theta}}
\end{gather*}
defined for clockwise toroidal direction $\diff \zeta = -\diff \phi$ (see \cref{sec:cocos}) by dividing through $B_{0}^{\theta}$, we obtain
\begin{align*}
  \left( \pd{\theta} + q \pd{\zeta} \right) \delta \psi = \left( \pd{\theta} - q \pd{\phi} \right) \delta \psi &= -\sigma_{\pol} \frac{\delta B^{\rho}}{B_{0}^{\theta}} = -\sigma_{\pol} \frac{\sqrt{g} \delta B^{\rho}}{\psi_{\tor}' (\rho)} = -\sigma_{\pol} \frac{\sqrt{g} \delta B^{\rho}}{\sigma_{\pol} \psi_{\tor}' (\psi)} = \\
  &= -\frac{\sqrt{g} \delta B^{\rho}}{q} = -\sigma_{\pol} \frac{\sqrt{g} \delta B^{\psi}}{q}.
\end{align*}
Written in terms of toroidal Fourier harmonics $m, n$ in $\theta, \phi$ the equation becomes
\begin{gather*}
  \psi_{m n} = -\sigma_{\pol} \frac{\left( \sqrt{g} \delta B^{\psi} \right)_{m n}}{q (m - n q)} = -\sigma_{\pol} \frac{\left( \sqrt{g} B_{n}^{\psi} \right)_{m}}{q (m - n q)}.
\end{gather*}
The safety factor $q$ is a flux function, but $\sqrt{g}$ depends also on $\theta$, so poloidal harmonics $m$ have to be taken outside the bracket here. In order to fulfill the original requirement not to break flux surfaces, $\psi_{m n}$ must never diverge, so $(m - n q)$ must not become zero. We can avoid such resonant surfaces if
\begin{gather*}
  \left( \sqrt{g} B_{n}^{\psi} \right)_{m} = 0
\end{gather*}
for all possible $m$ that could lead to a resonance. The simplest way to do so is to make $\sqrt{g} B_{n}^{\psi}$ a flux function, possessing only the poloidally symmetric harmonic $m = 0$. More generally, also $m < n q_{\text{min}}$ and $m > n q_{\text{max}}$ are possible, but for nonzero $m$ a transformation to the flux angle $\theta$ has to be computed explicitly.

Using the Jacobian of symmetry flux coordinates we obtain
\begin{gather*}
  \psi_{m n} = \frac{1}{m - n q} \left( \frac{R^{2}}{B_{0 \phi}} B_{n}^{\psi} \right)_{m}
\end{gather*}
and so, to generate a non-resonant field, we are allowed to use any value
\begin{gather*}
  B_{n}^{\psi} = C(\psi) \frac{B_{0 \phi}}{R^{2}},
\end{gather*}
where $C(\psi)$ is a flux function. The other components of $\Bpert$ are not relevant for the resonance condition and just need to fulfill divergence-freeness. Thus we proceed as in \cref{sec:compute_currn} and find fluxes through triangle edges:
\begin{gather}
  \int_{\Gamma_{\inw}, \Gamma_{\out}} \diff l \, R \vec{B}_{n}^{\pol} \cdot \vec{n} = -\int_{\Gamma_{\fs}} \diff l \, R \vec{B}_{n}^{\pol} \cdot \vec{n} - \im n \int_{\Omega} \diff R \, \diff Z \, R B_{n}^{\phi}. \label{eq:Bn_divfree}
\end{gather}
As for currents we use the notation for weighted magnetic fluxes through edges,
\begin{gather}
  \Psi_{k} = \int_{\Gamma_{k}} R \vec{B}_{n}^{\pol} \cdot \vec{n} \, \diff l \approx R (\Gamma_{k}) \vec{B}_{n}^{\pol} (\Gamma_{k}) \cdot \vec{n}_{k},
\end{gather}
and the same for the weighted toroidal magnetic flux,
\begin{gather}
  \Psi_{\phi} = \int_{\Omega} \diff R \, \diff Z \, R B_{n}^{\phi} \approx S_{\Omega} \, R (\Omega) B_{n}^{\phi} (\Omega),
\end{gather}
resulting in a shortened notation for \cref{eq:Bn_divfree} where the system of equations to be assembled is more apparent:
\begin{gather}
  \Psi_{\inw} + \Psi_{\out} = -\Psi_{\fs} - \im n \Psi_{\phi}.
\end{gather}
The term through edge \fs\ orthogonal to $\vec{n}_{\fs} \parallel \grad \psi$ is
\begin{gather}
  \Psi_{\fs} \approx B_{n}^{\psi} (\Gamma_{\fs}) \frac{R (\Gamma_{\fs}) l_{\fs}}{\vec{n}_{\fs} \cdot \grad \psi (\Gamma_{\fs})} \label{eq:Psi_f}
\end{gather}
with the sign of the directional derivative depending on edge orientation.

The system of linear equations to solve is
\begin{gather}
  K_{jk} \Psi^{(k)} = s_{j} \quad \forall j, k = 1, 2, \dotsc, N,
\end{gather}
with
\begin{gather}
  K_{jk} = \delta_{j-1, k} - \delta_{jk} \rightarrow \hat{K} = \begin{pmatrix}
    -1   &    1   &    0   & \hdots &    0   \\
     0   &   -1   &    1   & \hdots &    0   \\
     0   &    0   &   -1   & \hdots &    0   \\
  \vdots & \vdots & \vdots & \ddots & \vdots \\
     1   &    0   &    0   & \hdots &   -1
  \end{pmatrix}
\end{gather}
and
\begin{gather}
  s_{j} = -\Psi_{\fs}^{(j)} - \im n \Psi_{\phi}^{(j)}.
\end{gather}
Since any one column in $\hat{K}$ is a linear combination of all other columns, $\hat{K}$ is of rank $N - 1$ and thus singular. To construct a solution, we first consider the homogenous case with $\vec{s} = \vec{0}$, which also determines the toroidal flux. The non-zero solution for the remaining degrees of freedom then assumes the simple form
\begin{gather}
  \Psi^{(k)} = C_{0} \quad \forall k = 1, 2, \dotsc, N,
\end{gather}
with an arbitrary constant $C_{0}$. With local indices, the degrees of freedom are written as 
\begin{align}
  \Psi_{\inw}^{(k)} &= -C_{0}, & \Psi_{\out}^{(k)} &= C_{0} & \Psi_{\phi}^{(k)} &= \frac{\im}{n} \Psi_{\fs}^{(k)} \quad \forall k = 1, 2, \dotsc, N. \label{eq:Psi_hom}
\end{align}
This means that flux conservation, i.e.\ divergence-freeness, is fulfilled by balancing $\Psi_{\phi}$ with $\Psi_{\fs}$ on each individual triangle and assigning a constant flux $C_{0}$ in poloidal direction to $\Psi_{\inw}$ and $\Psi_{\out}$ over the entire triangle strip.

A consistent solution according to the \textsc{Rouché}-\textsc{Capelli} theorem can also be constructed for the inhomogeneous case, i.e.\ non-zero $\vec{s}$, as long as $\vec{s}$ is a linear combination of columns of $K$. Starting from the trivial (zero) solution to the homogeneous case and for a fixed $k$, we add to $\vec{s}$ the $(k+1)$-th column of $\hat{K}$, multiplied by another arbitrary constant $C_{k}$, resulting in
\begin{align}
  s_{k} &= -\Psi_{\fs}^{(k)} - \im n \Psi_{\phi}^{(k)} = C_{k}, \\
  s_{k+1} &= -\Psi_{\fs}^{(k+1)} - \im n \Psi_{\phi}^{(k+1)} = -C_{k}.
\end{align}
Considering the two equations affected compared to the homogeneous case,
\begin{align}
  -\Psi^{(k-1)} + \Psi^{(k)} &= C_{k}, \\
  -\Psi^{(k)} + \Psi^{(k+1)} &= -C_{k},
\end{align}
we arrive at a particular solution
\begin{gather}
  \Psi^{(k)} = C_{k}, \quad \Psi^{(j)} = 0 \quad \forall j \neq k.
\end{gather}
The degrees of freedom differing from the solution to the zero solution (\cref{eq:Psi_hom} with $C_{0} = 0$) are, given with local indices:
\begin{align}
  \Psi_{\inw}^{(k)} &= 0, & \Psi_{\out}^{(k)} &= C_{k}, & \Psi_{\phi}^{(k)} &= \frac{\im}{n} \Psi_{\fs}^{(k)} - \frac{\im}{n} C_{k}, \\
  \Psi_{\inw}^{(k+1)} &= -C_{k}, & \Psi_{\out}^{(k+1)} &= 0, & \Psi_{\phi}^{(k+1)} &= \frac{\im}{n} \Psi_{\fs}^{(k+1)} + \frac{\im}{n} C_{k}.
\end{align}
This means that, compared to the zero solution, a change $C_{k}$ in the flux across the associated edge of adjacent triangles in one triangle strip is accomdated by a change in the toroidal flux of these triangles.

The approach outlined above can be repeated for different values of $k$. Linear superposition of the solutions to the homogeneous and inhomogeneous cases then yields the most general solution,
\begin{align}
  \Psi_{\inw}^{(k)} &= -C_{0} - C_{k-1}, & \Psi_{\out}^{(k)} &= C_{0} + C_{k}, & \Psi_{\phi}^{(k)} &= \frac{\im}{n} \Psi_{\fs}^{(k)} + \frac{\im}{n} C_{k-1} - \frac{\im}{n} C_{k} \quad \forall k = 1, 2, \dotsc, N.
\end{align}
Now, to assign sensible values to the arbitrary constants, consider the solution to the homogeneous sytem of equations. Since edge \fs\ alternates between inner and outer flux surface for all but the innermost triangle strip, $\vec{n}_{\fs}$ will also alternate between pointing inwards and outwards. Thus $\Psi_{\fs}$ will alternate signs too, but it is consistent along one flux surface. Since $\Psi_{\phi}$ depends linearly on $\Psi_{\fs}$, on any one triangle the sign of $B_{n (\phi)}$ will differ from all three adjacent triangles, except for the innermost triangle strip. The resulting field is then clearly dependent on the choice of the grid. The procedure described before can be used to alleviate this problem. On every pair of triangles $\Omega^{(2k)}$ and $\Omega^{(2k+1)}$ -- which originally result from diagonally dividing a more regular quadrilateral -- we average $\Psi_{\phi}$ and set $C_{2k}$ to the deviation to counterbalance the change:
\begin{align}
  C_{2k} &= \frac{\im}{2 n} \left( \Psi_{\fs}^{(2k)} - \Psi_{\fs}^{(2k+1)} \right) \quad \forall k = 0, 1, \dotsc, \frac{N}{2} - 1.
\end{align}

\subsection{Analytical solution for very large aspect ratios}
\label{sec:analytical}

Using flux coordinates $(r, \theta, \phi)$, a term in \cref{eq:pn} can be simplified:
\begin{gather}
  \vec{B}_{0}^{\pol} \cdot \grad p_{n} = B_{0}^{\theta} \vec{e}_{\theta} \cdot \grad p_{n} = B_{0}^{\theta} \pd[p_{n}]{\theta}.
\end{gather}
Now when we expand the perturbed quantities as series in $\theta$ as well as $\phi$ so that
\begin{gather}
  \delta p = \sum_{n} p_n(r, \theta) \e^{\im n \phi} = \sum_{m, n} p_{mn}(r) \e^{\im m \theta + \im n \phi},
\end{gather}
\cref{eq:pn} reduces to
\begin{gather}
  \im m p_{mn} B_{0}^{\theta} + \im n p_{mn} B_{0}^{\phi} = -B_{mn}^{r} p_{0}'(r) =: -s_{mn}
\end{gather}
with a source term $s_{mn}$. This allows a simple algebraic solution where all quantities depend only on $r$:
\begin{gather}
  p_{mn} =  \frac{\im s_{mn}}{m B_{0}^{\theta} + n B_{0}^{\phi}}.
\end{gather}
Now an approximation can be made by switching from contravariant components to covariant components which are (for some reason) assumed to be constant. The factor $R$ appearing in the metric components can also be considered constant over the domain in question -- the toroidal geometry is effectively transformed to cylindrical geometry with a very thin and elongated cylinder and periodic boundary conditions.

[\ldots]

\clearpage
\appendix
\section{List of symbols}

\begin{longtable}{l >{\RaggedRight}p{0.8\textwidth}}
  \caption{Base symbols} \\
  \toprule
  \textbf{Notation} & \textbf{Description} \\
  \midrule
  \endhead
  $\vec{B}$ & magnetic field strength, measured in Gauss \\
  $\Bplas$ & perturbation field from \emph{p}lasma current \\
  $\Bvac$ & perturbation field in \emph{v}acuum (from external coils) \\
  $C$ & arbitrary constant (of integration) \\
  $c$ & speed of light in centimetres per second \\
  $\e$ & \textsc{Euler}'s constant, base of natural logarithm \\
  $g$ & metric determinant \\
  $\vec{h}_{0}$ & unit vector of $\vec{B}_{0}$ \\
  $\hat{I}$ & unit matrix \\
  $I$ & electric current \\
  $I_{\text{c}}$ & electric current produced by RMP \emph{c}oils \\
  $\im$ & imaginary unit, $\im^{2} = -1$ \\
  $\vec{J}$ & electric current density, measured in statampere \\
  $\hat{K}$ & combined linear operator $\hat{M} \hat{P}$ / stiffness matrix \\
  $\vec{l}$ & edge vector in counter-clockwise direction \\
  $l$ & length of edge \\
  $\hat{M}$ & linear operator representing computation of the magnetic field from the currents via \textsc{Ampère}'s equation \\
  $m$ & poloidal mode number \\
  $n$ & toroidal mode number \\
  $\vec{n}$ & outward pointing normal vector \\
  $N$ & dimension of system of linear equations \\
  $\hat{P}$ & linear operator representing computation of the currents from the magnetic field via MHD equations \\
  $p$ & pressure, measured in dyne per square centimetre \\
  $q$ & safety factor \\
  $R$ & radial coordinate in cylindrical coordinates \\
  $R_{0}$ & major radius of the tokamak \\
  $S$ & area in poloidal cross-section \\
  $Z$ & axial coordinate in cylindrical coordinates \\
  $\delta_{ij}$ & \textsc{Kronecker} delta \\
  $\theta$ & poloidal angle \\
  $\phi$ & toroidal angle \\
  $\Psi$ & magnetic flux \\
  $\psi$ & flux surface label \\
  \bottomrule
\end{longtable}

\begin{longtable}{l >{\RaggedRight}p{0.8\textwidth}}
  \caption{Symbol decorations} \\
  \toprule
  \textbf{Notation} & \textbf{Description} \\
  \midrule
  \endhead
  $p_{0}$ & equilibrium $p$ \\
  $\delta p$ & perturbation of $p$ \\
  $p_{n}$ & \textsc{Fourier} coefficient of perturbation of $p$ with toroidal mode number $n$ \\
  $p_{m n}$ & \textsc{Fourier} coefficient of perturbation of $p$ with poloidal mode number $m$ and toroidal mode number $n$ \\
  $p^{(k)}$ & $k$-th summand in series expansion of perturbed $p$ / $k$-th degree of freedom of $p$ in a particular loop \\
  $p^{[k]}$ & $k$-th partial sum in series expansion of perturbed $p$ \\
  \midrule
  $\vec{J}^{\pol}$ & poloidal component of $\vec{J}$ \\
  $\vec{J}^{\tor}$ & toroidal component of $\vec{J}$ \\
  $\vec{J}^{\parallel}$ & component of $\vec{J}$ parallel to $\vec{B}_{0}$ \\
  $\vec{J}^{\perp}$ & component of $\vec{J}$ perpendicular to $\vec{B}_{0}$ \\
  $J^{u}$ & contravariant component of $\vec{J}$ w.r.t.\ coordinate $u$ \\
  $J_{u}$ & covariant component of $\vec{J}$ w.r.t.\ coordinate $u$ \\
  $J_{(u)}$ & physical component of $\vec{J}$ w.r.t.\ coordinate $u$ \\
  $\hat{\vec{n}}$ & unit vector in direction of $\vec{n}$ \\
  \midrule
  $\vec{B} (\vec{r}^{(k)})$ & $\vec{B}$ evaluated at node with index $k$ \\
  $\vec{B} (\Gamma_{e}^{(k)})$ & $\vec{B}$ evaluated at midpoint of edge $e$ of triangle $k$ \\
  $\vec{B} (\Omega^{(k)})$ & $\vec{B}$ evaluated at weighted centroid of triangle $k$ \\
  \bottomrule
\end{longtable}

\section{\textsc{Arnoldi} iterations}
\label{app:Arnoldi}

For an $N \times N$ matrix $\hat{K}$, the \textsc{Arnoldi} algorithm gives the largest $n_{r}$ \textsc{Ritz} eigenvalues $\lambda_{k}$  and the associated eigenvectors $\vec{v}_{k}$ which span the \textsc{Krylov} subspace of $\hat{K}$. To accomplish this, $\hat{K}$ is repeatedly applied to an arbitrary initial vector $\vec{q}_{1}$. In each iteration step, the next \textsc{Arnoldi} vector $\vec{q}_{k}$ is generated and orthonormalized with respect to the previously generated vectors by \textsc{Gram}-\textsc{Schmidt} orthogonalization. Also, entries of an upper \textsc{Hessenberg} matrix $\hat{H}$ are constructed, as outlined in \cref{alg:arnoldi}.
\begin{algorithm}
  \caption{\textsc{Arnoldi} iterations}
  \label{alg:arnoldi}
  \begin{algorithmic}[1]
    \State{$\vec{q}_{1} \gets \frac{\vec{q}_{1}}{\norm{\vec{q}_{1}}}$}
    \For{$k = 2, 3, \dotsc, n_{r}$}
      \State{$\vec{q}_{k} \gets \hat{K} \vec{q}_{k-1}$}
      \For{$j = 1, 2, \dotsc, k-1$}
        \State{$H_{j, k-1} \gets \vec{q}_{j}^{\dagger} \vec{q}_{k}$}
        \State{$\vec{q}_{k} \gets \vec{q}_{k} - H_{j, k-1} \vec{q}_{j}$}
      \EndFor
      \State{$H_{k, k-1} \gets \norm{\vec{q}_{k}}$}
      \State{$\vec{q}_{k} \gets \frac{\vec{q}_{k}}{H_{k, k-1}}$}
    \EndFor
  \end{algorithmic}
\end{algorithm}
Grouping together the $\vec{q}_{k}$ vectors in an $N \times n_{r}$ matrix $\hat{Q}$, an approximation of $\hat{K}$ can be written out as
\begin{gather}
  \hat{K} \approx \hat{Q} \hat{H} \hat{Q}^{\dagger}.
\end{gather}
Diagonalization of $\hat{H}$ via the LAPACK routine \texttt{zhseqr} yields the \textsc{Ritz} eigenvalues $\lambda_{k}$ and application of $\hat{Q}$ on the eigenvectors of $\hat{H}$ obtained via \texttt{zhsein} gives the eigenvectors $\vec{v}_{k}$ associated with $\hat{K}$.

\section{Fourier transform of vector quantitites}

$\Bpert$ in symmetry flux coordinates $(\rho, \theta, \phi)$ is given as the \textsc{Fourier} series
\begin{gather}
  \delta B^{k} (\rho, \theta, \phi) = B_{m n}^{k}(\rho) \e^{-\im n \phi} \e^{-\im m \theta}.
\end{gather}
$\Bpert$ is also divergence-free and assuming $g = g(\rho, \theta)$, we get
\begin{align}
  0 &= \divg \Bpert = \frac{1}{\sqrt{g}} \partial_{k} \left( \sqrt{g} \delta B^{k} \right) = \frac{1}{\sqrt{g}} \partial_{k} \left( \sqrt{g} B_{m n}^{k} \e^{-\im n \phi} \e^{-\im m \theta} \right) = \\
    &= \frac{1}{\sqrt{g}} \partial_{\rho} \left( \sqrt{g} B_{m n}^{\rho} \right) \e^{-\im n \phi} \e^{-\im m \theta} + B_{m n}^{\phi} \partial_{\phi} \left( \e^{-\im n \phi} \right) \e^{-\im m \theta} + \frac{1}{\sqrt{g}} \partial_{\theta} \left( \sqrt{g} \e^{-\im m \theta} \right) B_{m n}^{\theta} \e^{-\im n \phi} = \\
    &= \underbrace{\e^{-\im n \phi} \e^{-\im m \theta}}_{\neq 0} \left( \frac{\partial_{\rho} \sqrt{g}}{\sqrt{g}} B_{m n}^{\rho} + \partial_{\rho} B_{m n}^{\rho} - \im n B_{m n}^{\phi}  + \frac{\partial_{\theta} \sqrt{g}}{\sqrt{g}} B_{m n}^{\theta} - \im m B_{m n}^{\theta} \right).
\end{align}

\printbibliography

\end{document}

%%% Local Variables: 
%%% mode: latex
%%% TeX-master: t
%%% End: 
