\iffalse
% glossary
\usepackage[
  xindy,
  numberedsection,
  toc,
  nonumberlist,
  nopostdot
]{glossaries}
\usepackage{glossary-longbooktabs}
\setglossarystyle{longragged-booktabs}
\makeglossaries
\AfterPreamble{%
  \newglossaryentry{Aprecon}{
% type = {base},
  name = {\ensuremath{\hat{A}}},
  sort = {AAm},
  description = {matrix applied to precondition iterations}
}
\newglossaryentry{vecA}{
% type = {base},
  name = {\ensuremath{\vec{A}}},
  sort = {AAv},
  description = {magnetic vector potential}
}
\newglossaryentry{vecB}{
% type = {base},
  name = {\ensuremath{\vec{B}}},
  sort = {ABv},
  description = {magnetic field strength, measured in Gauss}
}
\newglossaryentry{Bplas}{
% type = {base},
  name = {\ensuremath{\Bplas}},
  sort = {ABvp},
  description = {perturbation field from \emph{p}lasma current}
}
\newglossaryentry{Bvac}{
% type = {base},
  name = {\ensuremath{\Bvac}},
  sort = {ABvv},
  description = {perturbation field in \emph{v}acuum (from external coils)}
}
\newglossaryentry{arbconst}{
% type = {base},
  name = {\ensuremath{C}},
  sort = {AC},
  description = {arbitrary constant (of integration)}
}
\newglossaryentry{lightspeed}{
% type = {base},
  name = {\ensuremath{c}},
  sort = {Ac},
  description = {speed of light in centimetres per second}
}
\newglossaryentry{Euler}{
% type = {base},
  name = {\ensuremath{\e}},
  sort = {Ae},
  description = {\textsc{Euler}'s constant, base of natural logarithm}
}
\newglossaryentry{basis}{
% type = {base},
  name = {\ensuremath{\vec{e}}},
  sort = {Aev},
  description = {basis vector}
}
\newglossaryentry{metric}{
% type = {base},
  name = {\ensuremath{g}},
  sort = {Ag},
  description = {metric determinant}
}
\newglossaryentry{vech}{
% type = {base},
  name = {\ensuremath{\vec{h}_{0}}},
  sort = {Ah},
  description = {unit vector of \ensuremath{\vec{B}_{0}}}
}
\newglossaryentry{unitmat}{
% type = {base},
  name = {\ensuremath{\hat{I}}},
  sort = {AIm},
  description = {unit matrix}
}
\newglossaryentry{current}{
% type = {base},
  name = {\ensuremath{I}},
  sort = {AIs},
  description = {electric current}
}
\newglossaryentry{coilcurrent}{
% type = {base},
  name = {\ensuremath{I_{\text{c}}}},
  sort = {AIsc},
  description = {electric current produced by RMP \emph{c}oils}
}
\newglossaryentry{imun}{
% type = {base},
  name = {\ensuremath{\im}},
  sort = {Ai},
  description = {imaginary unit, \ensuremath{\im^{2} = -1}}
}
\newglossaryentry{vecJ}{
% type = {base},
  name = {\ensuremath{\vec{J}}},
  sort = {AJ},
  description = {electric current density, measured in statampere}
}
\newglossaryentry{opK}{
% type = {base},
  name = {\ensuremath{\hat{K}}},
  sort = {AKm1},
  description = {combined linear operator \ensuremath{\hat{M} \hat{P}}}
}
\newglossaryentry{stiff}{
% type = {base},
  name = {\ensuremath{\hat{K}}},
  sort = {AKm2},
  description = {stiffness matrix}
}
\newglossaryentry{boltzmann}{
% type = {base},
  name = {\ensuremath{k_{\text{B}}}},
  sort = {Ak},
  description = {\textsc{Boltzmann} constant}
}
\newglossaryentry{vecl}{
% type = {base},
  name = {\ensuremath{\vec{l}}},
  sort = {Alv},
  description = {edge vector in counter-clockwise direction}
}
\newglossaryentry{l}{
% type = {base},
  name = {\ensuremath{l}},
  sort = {Als},
  description = {length of edge}
}
\newglossaryentry{opM}{
% type = {base},
  name = {\ensuremath{\hat{M}}},
  sort = {AM},
  description = {linear operator representing computation of the magnetic field from the currents via \textsc{Ampère}'s equation}
}
\newglossaryentry{m}{
% type = {base},
  name = {\ensuremath{m}},
  sort = {Am},
  description = {poloidal mode number}
}
\newglossaryentry{n}{
% type = {base},
  name = {\ensuremath{n}},
  sort = {An1},
  description = {toroidal mode number}
}
\newglossaryentry{dens}{
% type = {base},
  name = {\ensuremath{n}},
  sort = {An2},
  description = {density of particles}
}
\newglossaryentry{Ndim}{
% type = {base},
  name = {\ensuremath{N}},
  sort = {AN},
  description = {dimension of system of linear equations}
}
\newglossaryentry{normal}{
% type = {base},
  name = {\ensuremath{\vec{n}}},
  sort = {Anv},
  description = {outward pointing normal vector}
}
\newglossaryentry{opP}{
% type = {base},
  name = {\ensuremath{\hat{P}}},
  sort = {AP},
  description = {linear operator representing computation of the currents from the magnetic field}
}
\newglossaryentry{pres}{
% type = {base},
  name = {\ensuremath{p}},
  sort = {Ap},
  description = {pressure, measured in dyne per square centimetre}
}
\newglossaryentry{q}{
% type = {base},
  name = {\ensuremath{q}},
  sort = {Aq},
  description = {safety factor}
}
\newglossaryentry{R}{
% type = {base},
  name = {\ensuremath{R}},
  sort = {AR},
  description = {radial coordinate in cylindrical coordinates}
}
\newglossaryentry{R0}{
% type = {base},
  name = {\ensuremath{R_{0}}},
  sort = {AR0},
  description = {major radius of the tokamak}
}
\newglossaryentry{r}{
% type = {base},
  name = {\ensuremath{r}},
  sort = {Ar},
  description = {arbitrary point in a domain \ensuremath{\Omega}}
}
\newglossaryentry{area}{
% type = {base},
  name = {\ensuremath{S}},
  sort = {AS},
  description = {area in poloidal cross-section}
}
\newglossaryentry{source}{
% type = {base},
  name = {\ensuremath{\vec{s}}},
  sort = {As},
  description = {source term in systems of linear equations}
}
\newglossaryentry{temp}{
% type = {base},
  name = {\ensuremath{T}},
  sort = {AT},
  description = {temperature}
}
\newglossaryentry{Z}{
% type = {base},
  name = {\ensuremath{Z}},
  sort = {AZ},
  description = {axial coordinate in cylindrical coordinates}
}
\newglossaryentry{Gamma}{
% type = {base},
  name = {\ensuremath{\Gamma}},
  sort = {G03},
  description = {border of a domain \ensuremath{\Omega}, e.g. edge of a triangle}
}
\newglossaryentry{kronecker}{
% type = {base},
  name = {\ensuremath{\delta_{ij}}},
  sort = {G04},
  description = {\textsc{Kronecker} delta}
}
\newglossaryentry{theta}{
% type = {base},
  name = {\ensuremath{\theta}},
  sort = {G08},
  description = {poloidal angle}
}
\newglossaryentry{eigval}{
% type = {base},
  name = {\ensuremath{\lambda}},
  sort = {G11},
  description = {eigenvalue}
}
\newglossaryentry{rho}{
% type = {base},
  name = {\ensuremath{\rho}},
  sort = {G17},
  description = {flux surface label}
}
\newglossaryentry{phi}{
% type = {base},
  name = {\ensuremath{\phi}},
  sort = {G21},
  description = {toroidal angle}
}
\newglossaryentry{flux}{
% type = {base},
  name = {\ensuremath{\Psi}},
  sort = {G23A},
  description = {magnetic flux}
}
\newglossaryentry{psi}{
% type = {base},
  name = {\ensuremath{\psi}},
  sort = {G23a},
  description = {disc poloidal flux, used as flux surface label}
}
\newglossaryentry{Omega}{
% type = {base},
  name = {\ensuremath{\Omega}},
  sort = {G24},
  description = {domain of computation, e.g. a triangle}
}

}
\fi


\begin{align}
  \vec{j}_{0 \perp} &= \frac{-c \nabla p_{0} \times \vec{B}_{0}}{B_{0}^{2}} \\
  &= \frac{-c p_{0}' (\psi) \nabla \psi \times (\nabla \psi \times \nabla \phi + B_{0 \phi} \nabla \phi)}{B_{0}^{2}} \nonumber \\
  &= \frac{-c p_{0}' (\psi)}{B_{0}^{2}} \left( B_{0 \phi} \vec{B}_{0}^{\pol} - \lvert \nabla \psi \rvert^{2} \nabla \phi \right).
\end{align}

\begin{gather}
  \left( B_{0}^{\pol} \right)^{2} + \left( B_{0}^{\tor} \right)^{2} = B_{0}^{2}
\end{gather}

\subsection{Maxwell's equations}

\textsc{Maxwell}'s equations in their differential and microscopic formulation:
\begin{align}
  \divg \vec{E} &= 4 \pi \rho_{\text{e}}, \label{eq:gauss_field} \\
  \divg \vec{B} &= 0, \label{eq:no_monopoles_field} \\
  \curl \vec{E} &= -\frac{1}{c_{0}} \pd[\vec{B}]{t}, \label{eq:faraday_field} \\
  \curl \vec{B} &= \frac{4 \pi}{c_{0}} \vec{J} + \frac{1}{c_{0}} \pd[\vec{E}]{t}. \label{eq:ampere_field}
\end{align}
The fields $\vec{E}$ and $\vec{B}$ can be expressed via electrical potential $\phi_{\text{e}}$ and magnetic vector potential $\vec{A}$:
\begin{align}
  \vec{E} &= -\grad \phi_{\text{e}} - \frac{1}{c_{0}} \pd[\vec{A}]{t}, \label{eq:electric_field} \\
  \vec{B} &= \curl \vec{A}. \label{eq:magnetic_field}
\end{align}
With these definitions, the equations are invariant under gauge transformations involving a gauge potential $\chi$:
\begin{align*}
  \phi_{\text{e}} &\to \phi_{\text{e}}' = \phi_{\text{e}} - \frac{1}{c_{0}} \pd[\chi]{t}, \\
  \vec{A} &\to \vec{A}' = \vec{A} + \grad \chi.
\end{align*}
Common gauges are the \textsc{Coulomb} gauge with
\begin{gather*}
  \divg \vec{A}' = 0
\end{gather*}
and the \textsc{Lorenz} gauge with
\begin{gather*}
  \divg \vec{A}' + \frac{1}{c_{0}} \pd[\phi_{\text{e}}']{t} = 0.
\end{gather*}
Since we neglect time derivatives, these gauges are equal in our case. The potential equations are then given by
\begin{align}
  \divg \nabla \phi_{\text{e}} &= -4 \pi \rho_{\text{e}}, \label{eq:gauss_potential} \\
  \divg (\curl \vec{A}) &= 0, \label{eq:no_monopoles_potential} \\
  \curl \nabla \phi_{\text{e}} &= 0, \label{eq:faraday_potential} \\
  \curl (\curl \vec{A}) &= \frac{4 \pi}{c_{0}} \vec{J}, \label{eq:ampere_vector}
\end{align}
where the homogeneous \cref{eq:no_monopoles_potential,eq:faraday_potential} are trivially true. Since $\vec{A}$ can be gauged independently, we can continue with \cref{eq:ampere_vector} alone.

\subsection{Regularization (by Sergei Kasilov)}

Both equations (for pressure and for current) are magnetic differential equations of the form
\begin{gather*}
  \vec{B}_{0} \cdot \nabla f = q
\end{gather*}
where $q$ is the source and $f$ is the unknown. We are solving for a single Fourier mode over the toroidal angle to transform it to
\begin{gather*}
  \vec{B}_{0}^{\pol} \cdot \nabla f + \im n {B}_{0}^{\phi} f = s
\end{gather*}
If this equation is written in symmetry flux coordinates it becomes
\begin{gather*}
  B_{0}^{\theta} \pd[f]{\theta} + \im n B_{0}^{\phi} f = s
\end{gather*}
which is better re-written as
\begin{gather*}
  \pd[f]{\theta} + \im n q f = \frac{s}{B_{0}^{\theta}}.
\end{gather*}
Fourier analysis over $\theta$ makes it algebraic and shows resonant denominators
\begin{gather*}
  \im (m + n q) f_{m} = \left [ \frac{s}{B_{0}^{\theta}} \right ]_{m}.
\end{gather*}
The suggestion is to add a small imaginary part to the resonant denominator (the sign does not matter). This can be done if one replaces the toroidal wavenumber $n$ in the right-hand side with the product $n (1 + \im \epsilon)$ where $\epsilon \ll 1$. This should be done in two places:
\begin{enumerate}
\item in the equation for the perturbed pressure, $n$ should be modified in the quantity $b_{k}$ given by eq.~(61);
\item in the equation for the current, $n$ should be modified in the matrix $K_{jk}$ given by eq.~(90).
\end{enumerate}
The value of the regularization parameter $\epsilon$, on one hand, should be small, but on the other hand it should be large enough in order that the resonant factor $1 / (m + n q + \im \epsilon n q)$ does not change much for neighbourig radial grid points. If we denote the distance between radial grid points with $\symup{\Delta} r$, the criteria for $\epsilon$ look as follows,
\begin{gather*}
  \symup{\Delta} r \frac{q'}{q} \ll \epsilon \ll 1,
\end{gather*}
where $q'$ is the radial derivative of the safety factor $q$.
Obvioulsy, the better the refinement of the grid around the resonant suface, the smaller are $\epsilon$ values needed to satisfy the first criterion (and the smaller is the error introduced by the regularization).

\subsection{Sequence of partial sums}

\begin{align}
  \delta \vec{B}_{\text{p}}^{[n]} &= \hat{K} \left ( \delta \vec{B}_{\text{v}} + \delta \vec{B}_{\text{p}}^{[n-1]} \right ) \nonumber \\
  &= \hat{K} \left ( \delta \vec{B}_{\text{v}} + \hat{K} \left ( \delta \vec{B}_{\text{v}} + \delta \vec{B}_{\text{p}}^{[n-2]} \right ) \right ) \nonumber \\
  &= \hat{K} \left ( \delta \vec{B}_{\text{v}} + \hat{K} \left ( \delta \vec{B}_{\text{v}} + \hat{K} \left ( \delta \vec{B}_{\text{v}} + \dotsb \right ) \right ) \right ) \nonumber \\
  &= \hat{K} \delta \vec{B}_{\text{v}} + \hat{K}^{2} \delta \vec{B}_{\text{v}} + \hat{K}^{3} \delta \vec{B}_{\text{v}} + \dotsb + \hat{K}^{n+1} \delta \vec{B}_{\text{v}} \nonumber \\
  &= \left ( \hat{I} + \hat{K} + \hat{K}^{2} + \dotsb + \hat{K}^{n} \right ) \hat{K} \delta \vec{B}_{\text{v}} \nonumber \\
  &= \sum_{k = 1}^{n} \hat{K}^{k} \delta \vec{B}_{\text{v}}.
\end{align}


\subsubsection{Alternative formulation for the current through flux surface edges}

As an alternative to \cref{eq:If} it can be shown (see writeup on gyrokinetics) that
\begin{align}
  \vec{j}_{n}^{\perp} &= \vec{j}_{n} - (\vec{j}_{n} \cdot \vec{h}_{0}) \vec{h}_{0} \nonumber \\
  &= j_{0}^{\parallel} \frac{\vec{B}_{n}^{\perp}}{B_{0}} - \frac{c B_{n}^{\parallel}}{B_{0}^{2}} \vec{h}_{0} \times \nabla p_{0} + \frac{c}{B_{0}} \vec{h}_{0} \times (\nabla p_{n} + \im n p_{n} \nabla \phi).
\end{align}
Scalar multiplication with $\vec{n}_{\fs} \parallel \nabla p_{0} \parallel \nabla \psi \perp \vec{h}_{0}$ -- the first relation is true for infinitesimally small triangles -- yields
\begin{align}
  \vec{j}_{n}^{\perp} \cdot \vec{n}_{\fs} = \vec{j}_{n}^{\perp \pol} \cdot \vec{n}_{\fs} &= j_{0}^{\parallel} \frac{\vec{B}_{n} \cdot \vec{n}_{\fs}}{B_{0}} + \frac{c}{B_{0}} \vec{n}_{\fs} \cdot (\vec{h}_{0} \times (\nabla p_{n} + \im n p_{n} \nabla \phi)) \nonumber \\
  &= j_{0}^{\parallel} \frac{\vec{B}_{n} \cdot \vec{n}_{\fs}}{B_{0}} + \frac{c}{B_{0}} \vec{n}_{\fs} \cdot (h_{0 \phi} \nabla \phi \times \nabla p_{n} + \im n p_{n} \vec{h}_{0}^{\pol} \times \nabla \phi) \nonumber \\
  &= j_{0}^{\parallel} \frac{\vec{B}_{n} \cdot \vec{n}_{\fs}}{B_{0}} + \frac{c}{B_{0}} (h_{0 \phi} \nabla p_{n} \cdot (\vec{n}_{\fs} \times \nabla \phi) + \im n p_{n} \vec{h}_{0}^{\pol} \cdot (\nabla \phi \times \vec{n}_{\fs})) \nonumber \\
  &= j_{0}^{\parallel} \frac{\vec{B}_{n} \cdot \vec{n}_{\fs}}{B_{0}} + \frac{c}{R B_{0}} \vec{l}_{\fs} \cdot (h_{0 \phi} \nabla p_{n} - \im n p_{n} \vec{h}_{0}^{\pol}) \nonumber \\
  &= j_{0}^{\parallel} \frac{\vec{B}_{n} \cdot \vec{n}_{\fs}}{B_{0}} + \frac{c}{B_{0}^{2}} \vec{l}_{\fs} \cdot \left( B_{0 (\phi)} \nabla p_{n} - \frac{\im n}{R} p_{n} \vec{B}_{0}^{\pol} \right). \label{eq:I3_alt}
\end{align}

\section{Current perturbation -- alternative}

We split the current into parallel and perpendicular current:
\begin{gather}
  \vec{j}_{n} = j_{n \parallel} \vec{h}_{0} + \vec{j}_{n \perp}.
\end{gather}
We write the divergence-freeness condition as
\begin{gather}
  \nabla \cdot (\vec{h}_{0}^{\pol} j_{n \parallel}) + \im n h_{0}^{\phi} j_{n \parallel} = -\nabla \cdot \vec{j}_{n \perp}^{\pol} - \im n j_{n \perp}^{\phi}.
\end{gather}
In integral form, multiplied by $R$:
\begin{gather}
  \oint R j_{n \parallel} \vec{h}_{0}^{\pol} \cdot \vec{n} \, \diff l + \im n \int R h_{0}^{\phi} j_{n \parallel} \, \diff S = -\oint R \vec{j}_{n \perp}^{\pol} \cdot \vec{n} \, \diff l - \im n \int R j_{n \perp}^{\phi} \, \diff S.
\end{gather}
Approximate
\begin{gather}
  R^{1} j_{n \parallel}^{1} \vec{h}_{0}^{\pol} \cdot \vec{n}^{1} + R^{2} j_{n \parallel}^{2} \vec{h}_{0}^{\pol} \cdot \vec{n}^{2} + \im n S_{\Omega} \frac{(h_{0 (\phi)}^{1} j_{n \parallel}^{1} + h_{0 (\phi)}^{2} j_{n \parallel}^{2})}{2} = \nonumber \\
  -R^{1} \vec{j}_{n \perp}^{\pol, 1} \cdot \vec{n}^{1} - R^{2} \vec{j}_{n \perp}^{\pol, 2} \cdot \vec{n}^{2} - R^{3} \vec{j}_{n \perp}^{\pol, 3} \cdot \vec{n}^{3} - \im n S_{\Omega} R j_{n \perp}^{\phi}.
\end{gather}
We use
\begin{align}
  j_{n \perp}^{\phi} &= \text{TODO} - \frac{c B_{n \parallel}}{B_{0}^{2}} \nabla \phi \cdot (\vec{h}_{0} \times \nabla p_{0}) + \frac{c}{B_{0}} \nabla \phi \cdot (\vec{h}_{0} \times \nabla p_{n}) \\
  &= -\frac{c B_{n \parallel}}{B_{0}^{3}} \nabla p_{0} \cdot ((\nabla \psi \times \nabla \phi) \times \nabla \phi) + \frac{c}{B_{0}^{2}} \nabla p_{n} \cdot ((\nabla \psi \times \nabla \phi) \times \nabla \phi)\\
  &= \frac{c B_{n \parallel}}{R^{2} B_{0}^{3}} \nabla p_{0} \cdot \nabla \psi - \frac{c}{R^{2} B_{0}^{2}} \nabla p_{n}\cdot \nabla \psi \\
  &= \frac{c B_{n \parallel}}{R^{2} B_{0}^{3}} p_{0}'(\psi) \lvert \nabla \psi \rvert^{2} - \frac{c}{R^{2} B_{0}^{2}} \nabla p_{n} \cdot \nabla \psi.
\end{align}
We would like to represent
\begin{gather}
  \nabla \psi = (\nabla \psi)_{1} \vec{l}^{1} + (\nabla \psi)_{2} \vec{l}^{2}.
\end{gather}
We construct the system
\begin{align*}
  \vec{l}^{1} \cdot \nabla \psi &= \lvert l^{1} \rvert^{2} (\nabla \psi)_{1} + \vec{l}^{1} \cdot \vec{l}^{2} (\nabla \psi)_{2} \\
  \vec{l}^{2} \cdot \nabla \psi = -(\vec{l}^{1} \cdot \nabla \psi) &= \vec{l}^{1} \cdot \vec{l}^{2} (\nabla \psi)_{1} + \lvert l^{2} \rvert^{2} (\nabla \psi)_{2}
\end{align*}
or generally
\begin{align*}
  v_{1} &= \frac{\lvert l^{2} \rvert^{2} + (\vec{l}^{1} \cdot \vec{l}^{2})}{\lvert l^{1} \rvert^{2} \lvert l^{2} \rvert^{2} - (\vec{l}^{1} \cdot \vec{l}^{2})^{2}} (\vec{l}^{1} \cdot \vec{v}) \\
  v_{2} &= \frac{\lvert l^{1} \rvert^{2} + (\vec{l}^{1} \cdot \vec{l}^{2})}{\lvert l^{1} \rvert^{2} \lvert l^{2} \rvert^{2} - (\vec{l}^{1} \cdot \vec{l}^{2})^{2}} (\vec{l}^{2} \cdot \vec{v}).
\end{align*}
Finally
\begin{align}
  \vec{v} &= \frac{\lvert l^{2} \rvert^{2} + (\vec{l}^{1} \cdot \vec{l}^{2})}{\lvert l^{1} \rvert^{2} \lvert l^{2} \rvert^{2} - (\vec{l}^{1} \cdot \vec{l}^{2})^{2}}(\vec{l}^{1} \cdot \vec{v}) \vec{l}^{1} + \frac{\lvert l^{1} \rvert^{2} + (\vec{l}^{1} \cdot \vec{l}^{2})}{\lvert l^{1} \rvert^{2} \lvert l^{2} \rvert^{2} - (\vec{l}^{1} \cdot \vec{l}^{2})^{2}}(\vec{l}^{2} \cdot \vec{v}) \vec{l}^{2} \\
  &= c_{1} (\vec{l}^{1} \cdot \vec{v}) \vec{l}^{1} + c_{2} (\vec{l}^{2} \cdot \vec{v}) \vec{l}^{2}.
\end{align}
Flux
\begin{align}
  B_{n \parallel} &= \vec{B}_{n} \cdot \vec{h} = \vec{B}_{n}^{\pol} \cdot \vec{h}^{\pol} + B_{n \phi} h^{\phi} \nonumber \\
  &= c_{1} (\vec{n}^{1} \cdot \vec{B}_{n}^{\pol}) \vec{n}^{1} \cdot \vec{h}^{\pol} + c_{2} (\vec{n}^{2} \cdot \vec{B}_{n}^{\pol}) \vec{n}^{2} \cdot \vec{h}^{\pol} + B_{n (\phi)} h_{(\phi)}.
\end{align}

\section{Reduced MHD (unfinished)}

We take
\begin{gather*}
  B_{n \phi} = 0
\end{gather*}
so
\begin{gather*}
  \vec{B}_{n} = \nabla \psi_{n} \times \nabla \phi.
\end{gather*}
Ampere's law becomes
\begin{gather*}
  \nabla \times \vec{B}_{n} = \frac{1}{4 \pi} \vec{J}_{n}.
\end{gather*}

\subsubsection{Non-resonant test field: alternative version including safety factor}

We would like to have a completely non-resonant field
\begin{gather*}
  \delta \vec{B} = \vec{B}_{n} \e^{\im n \phi}
\end{gather*}
with $B_{n}^{\theta} = 0$. As an ansatz we take as a radial component of the field
\begin{gather}
  B_{n}^{\psi} = \frac{B_{0 \phi} R_{0}}{R^{2}} = -\sigma_{\pol} \frac{q R_{0}}{\sqrt{g}}
\end{gather}
so
\begin{gather*}
  B_{n}^{\rho} = -\frac{q R_{0}}{\sqrt{g}}
\end{gather*}
and
\begin{gather*}
  B_{n}^{\zeta} = -B_{n}^{\psi} F(\psi) = -\sigma_{\pol} B_{n}^{\rho} F(\rho).
\end{gather*}
Divergence-freeness of $\delta \vec{B}$ is now possible with
\begin{align*}
  0 = \nabla \cdot \delta \vec{B} &= \frac{1}{\sqrt{g}} \pd{x^{k}} \left( \sqrt{g} \delta B^{k} \right) \\
  &= \frac{1}{\sqrt{g}} \left( \pd{\rho} \left( \sqrt{g} \delta B^{\rho} \right) + \pd{\zeta} \left( -\sigma_{\pol} \sqrt{g} \delta B^{\zeta} \right) \right) \\
  &= \frac{1}{\sqrt{g}} \left(\pd{\rho} \left( -q R_{0} \right) - \im n \left( \sigma_{\pol} q R_{0} F(\rho) \right) \right).
\end{align*}
We should thus have 
\begin{gather*}
  q' (\rho) R_{0} = -\im n \sigma_{\pol} q R_{0} F(\rho)
\end{gather*}
or
\begin{gather*}
  F = -\sigma_{\pol} \frac{q' (\rho)}{\im n q} = -\frac{q' (\psi)}{\im n q}.
\end{gather*}
We obtain
\begin{gather*}
  \vec{B}_{n} = B_{n}^{\psi} \vec{e}_{\psi} + B_{n}^{\phi} \vec{e}_{\phi}
\end{gather*}
with
\begin{align*}
  B_{n}^{\psi} &= \vec{B}_{n} \cdot \nabla \psi = \frac{B_{0 \phi} R_{0}}{R^{2}}, \\
  B_{n}^{\phi} &= \vec{B}_{n} \cdot \nabla \phi = -\frac{B_{0 \phi} R_{0}}{R^{2}} \frac{q' (\psi)}{\im n q}.
\end{align*}
Divergence-freeness in cylindrical coordinates gives
\begin{gather*}
  \int_{\Gamma_{\inw}, \Gamma_{\out}} \diff l \, R \vec{B}_{n}^{\pol} \cdot \vec{n} = -\int_{\Gamma_{\fs}} \diff l \, R \vec{B}_{n}^{\pol} \cdot \vec{n} - \im n \int_{\Omega} \diff R \, \diff Z \, R B_{n}^{\phi}.
\end{gather*}
