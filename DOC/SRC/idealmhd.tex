\section{Basic Plasma Physics}
\label{sec:plasma}

In this chapter, we give a short overview of plasma physics concepts that form the basis of this thesis. Since plasma physics is a wide field and derivations can be longwinded, we restrict ourselves to sketches and refer to literature instead for the details.

Plasma in the physical sense may defined as matter made up of free ions and electrons that is nonetheless electrically neutral on a macroscopic scale. Besides this condition of \emph{quasineutrality}, the \emph{weak coupling} condition demands that the electrostatic \textsc{Coulomb} energy between the particles is lower than ther kinetic energy, so that \textsc{Debye} screening limits the influence of the local electric field to within the \textsc{Debye} length. This conditions can be fulfilled by a large set of values for density and temperature, ranging from rarefied astrophysical plasmas like the interstellar medium to the dense electron gas in metals. Typical values for thermonuclear plasmas in fusion devices are $n \approx \SI{1e15}{\per\centi\meter}$ and $k_{\text{B}} T \approx \SI{1e4}{\electronvolt}$.

The collective behavior arising from the interactions of the moving charged particles mediated by electric and magnetic fields becomes quite complex -- no single theory can describe all plasma phenomena and is computationally tractable at the same time. The present thesis considers the theory of ideal \emph{magnetohydrodynamics} (MHD) which is useful to determine the macroscopic stability of plasma. To see how it relates to other models, we sketch its derivation via \emph{kinetic theory}. In the non-relativistic limit, \textsc{Newton}'s equations of motion describe the single particle motion under the influence of the \textsc{Lorentz} force originating from external fields generated by other particles in the plasma. This picture already allows a description of particle drifts and confinement in toroidal geometry by consideration of their trajectories for a given configuration of magnetic and electric fields. While it is theoretically possible to set up the system of equations to fully describe the interaction of all particles among each other and with external fields, this approach is neither computationally feasible nor \comment{desirable -- knowing the exact positions and velocities of each particle is not helpful in understanding their large-scale collective behavior.} Instead, we opt for the same level of abstraction as in statistical mechanics: we take ensemble averages of the electric and magnetic fields and desribe the particles' position and velocities by a \emph{distribution function} $f_{s} (\vec{r}, \vec{v}, t)$. Here, $\vec{r}$ is position, $\vec{v}$ is velocity, and $t$ is time, while $s$ designates the particle species (ions and electrons). The distribution function is normalized to
\begin{gather}
  \int_{\Omega} \diff^{3} r \int_{\symbb{R}^{3}} \diff^{3} v \, f_{s} (\vec{r}, \vec{v}, t) = 1
\end{gather}
for any given time $t$. The time evolution of $f_{s}$ is given by the \textsc{Vlasov}--\textsc{Boltzmann} equation
\begin{gather}
  \pd[f_{s} (\vec{r}, \vec{v}, t)]{t} + \vec{v} \cdot \grad_{\vec{r}} f_{s} (\vec{r}, \vec{v}, t) + \frac{Q_{s}}{m_{s}} \left ( \vec{E} + \frac{1}{c} \vec{v} \times \vec{B} \right ) \cdot \grad_{\vec{v}} f_{s} (\vec{r}, \vec{v}, t) = \left ( \pd[f_{s} (\vec{r}, \vec{v}, t)]{t} \right )_{\text{coll}},
\end{gather}
where the term on the right-hand side is the \emph{collision term} modelling \textsc{Coulomb} collisions. $Q_{s}$ and $m_{s}$ are the charge and mass of the particle species $s$, respectively. The plasma kinetic equations forms the basis for kinetic theory which is able to account for a wide array of particle effects within the plasma. To derive the fluid description of magnetohydrodynamics, we have to take moments of the distribution function, i.e., we integrate over velocity space with appropriate weights.

\subsection{Ideal Magnetohydrodynamics}
\label{sec:ideal-mhd}

With the total number $n_{0 s}$ of particles of species $s$, we derive the number density as a zeroth moment:
\begin{gather}
  n_{s} (\vec{r}, t) = n_{0 s} \int_{\symbb{R}^{3}} f_{s} (\vec{r}, \vec{v}, t) \, \diff^{3} v.
\end{gather}
The charge density and mass density simply follow as $\varrho_{m s} = m_{s} n_{s}$ and $\varrho_{q s} = q_{s} n_{s}$, respectively. As a first moment, we get the particle flux density
\begin{gather}
  \vec{\Gamma}_{s} (\vec{r}, t) = n_{0 s} \int_{\symbb{R}^{3}} f_{s} (\vec{r}, \vec{v}, t) \vec{v} \, \diff^{3} v,
\end{gather}
from which the particle velocity
\begin{gather}
  \vec{V}_{s} (\vec{r}, t) = \frac{\vec{\Gamma}_{s} (\vec{r}, t)}{n_{s} (\vec{r}, t)}
\end{gather}
and current density
\begin{gather}
  \vec{J}_{s} (\vec{r}, t) = q_{s} \vec{\Gamma}_{s} = q_{s} n_{s} \vec{V}_{s}
\end{gather}
follow. Taking the second moment gives the pressure tensor as
\begin{gather}
  \symsf{p}_{s} (\vec{r}, t) = n_{0 s} m_{s} \int_{\symbb{R}^{3}} f_{s} (\vec{r}, \vec{v}, t) (\vec{v} - \vec{V}_{s}) (\vec{v} - \vec{V}_{s}) \, \diff^{3} v = n_{0 s} m_{s} \int_{\symbb{R}^{3}} f_{s} (\vec{r}, \vec{v}, t) \vec{v} \vec{v} \, \diff^{3} v - m_{s} n_{s} \vec{V}_{s} \vec{V}_{s},
\end{gather}
which reduces to $\symsf{p}_{s} = p_{s} \symsf{I}_{3}$ in an isotropic medium. The latter is assumed when the plasma is dominated by collisions so that the distribution function takes the form of the \textsc{Maxwell} distribution in velocity space. While most plasmas of interest are not in thermodynamic equilibrium, the approximation is good enough for our needs and ideal MHD also gives reasonable results when the assumption does not apply. Now, taking moments of the \textsc{Vlasov}--\textsc{Boltzmann} equation yields the time evolution of the fluid quantities derived above. Taking the zeroth moment gives
\begin{gather}
  \pd{t} \int_{\symbb{R}^{3}} f_{s} \, \diff^{3} v + \nabla_{\vec{r}} \cdot \int_{\symbb{R}^{3}} \vec{v} f_{s} \, \diff^{3} v + \frac{q_{s}}{m_{s}} \int_{\symbb{R}^{3}} \nabla_{\vec{v}} \cdot \left ( \vec{E} + \frac{1}{c} \vec{v} \times \vec{B} \right ) f_{s} \, \diff^{3} v = \int_{\symbb{R}^{3}} \left ( \pd[f_{s}]{t} \right )_{\text{coll}} \, \diff^{3} v.
\end{gather}
The third integral term on the left-hand side vanishes after application of the divergence theorem because the distribution function vanishes at infinite velocity. Likewise, the integral of the collision term vanishes due to particle conservation. Expressing the remaining to integrals by fluid quantities yields
\begin{gather}
  \pd[n_{s}]{t} + \divg (n_{s} \vec{V}_{s}) = 0, \label{eq:continuity}
\end{gather}
which we identify as continuity equation. Taking the first moment results in
\begin{gather}
  \pd{t} \int_{\symbb{R}^{3}} \vec{v} f_{s} \, \diff^{3} v + \nabla_{\vec{r}} \cdot \int_{\symbb{R}^{3}} \vec{v} \vec{v} f_{s} \, \diff^{3} v + \frac{q_{s}}{m_{s}} \int_{\symbb{R}^{3}} \vec{v} \nabla_{\vec{v}} \cdot \left ( \vec{E} + \frac{1}{c} \vec{v} \times \vec{B} \right ) f_{s} \, \diff^{3} v = \int_{\symbb{R}^{3}} \vec{v} \left ( \pd[f_{s}]{t} \right )_{\text{coll}} \, \diff^{3} v = \vec{R}_{s}.
\end{gather}
Here, $\vec{R}_{s}$ is constrained by $\sum_{s} \vec{R}_{s} = \vec{0}$. Using integration by parts for the third integral on the left-hand side and some more rearrangement, we get the momentum equation
\begin{gather}
  m_{s} n_{s} \left ( \pd[\vec{V}_{s}]{t} + \vec{V}_{s} \cdot \grad \vec{V}_{s} \right ) + \divg \symsf{p}_{s} - q_{s} n_{s} \left ( \vec{E} + \frac{1}{c} \vec{V}_{s} \times \vec{B} \right ) = \vec{R}_{s}. \label{eq:momentum}
\end{gather}
Now, the continuity equation connects a zeroth moment ($n_{s}$) to a first moment ($\vec{V}_{s}$), the momentum equation connects both to a second moment ($\symsf{p}_{s}$), and so on. This hierarchy of equations has to be truncated at some point to give a closed set of equations, which will be discussed later. In addition to the equations discussed up to now, we have to consider \textsc{Maxwell}'s equations:
\begin{align}
  \divg \vec{E} &= 4 \pi \varrho_{q} = 4 \pi e (n_{i} - n_{e}), \label{eq:gauss_field} \\
  \divg \vec{B} &= 0, \label{eq:no_monopoles_field} \\
  \curl \vec{E} &= -\frac{1}{c} \pd[\vec{B}]{t}, \label{eq:faraday_field} \\
  \curl \vec{B} &= \frac{4 \pi}{c} \vec{J} + \frac{1}{c} \pd[\vec{E}]{t} = \frac{4 \pi}{c} e (n_{i} \vec{V}_{i} - n_{e} \vec{V}_{e}) + \frac{1}{c} \pd[\vec{E}]{t}. \label{eq:ampere_field}
\end{align}
This is the full two-fluid picture of magnetohydrodynamics. This will now be simplified in a few steps. [\ldots]

\subsection{Flux Surfaces}

[flux surfaces and \textsc{Grad}--\textsc{Shafranov} equation, flux surface quantities, \ldots]

\textcite{dHaeseleer91} defines the flux-surface average of an arbitrary quantity $\Phi$ in a toroidal system as
\begin{gather}
  \langle \Phi \rangle = \frac{\int_{0}^{2 \pi} \int_{0}^{2 \pi} \sqrt{g} \Phi \, \diff \phi \, \diff \theta}{\int_{0}^{2 \pi} \int_{0}^{2 \pi} \sqrt{g} \, \diff \phi \, \diff \theta}. \label{eq:flux_surface_avg}
\end{gather}

[\ldots]

%%% Local Variables: 
%%% mode: latex
%%% TeX-master: "../magdif"
%%% End: 
