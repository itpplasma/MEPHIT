\section{Basic Plasma Physics}
\label{sec:plasma}

This chapter contains a short overview of plasma physics concepts that form the basis of this thesis. Since plasma physics is a broad field and derivations can be longwinded, we restrict ourselves to sketches and refer to literature instead of giving details here. As such, this chapter is based mostly on the monograph by \textcite{Freidberg14}, albeit written in Gaussian units instead of SI units.

Plasma in the physical sense may be defined as matter that is made up of free ions and electrons that is nonetheless electrically neutral on a macroscopic scale. Besides this condition of \emph{quasineutrality}, the \emph{weak coupling} condition demands that the electrostatic \textsc{Coulomb} energy between the particles is lower than their kinetic energy so that \textsc{Debye} screening limits the influence of the local electric field to within the \textsc{Debye} length. This conditions can be fulfilled by a large set of values for density and temperature, ranging from rarefied astrophysical plasmas like the interstellar medium to the dense electron gas in metals. Typical values for thermonuclear plasmas in fusion devices are $n \approx \SI{1e14}{\per\centi\meter\cubed}$ and $k_{\text{B}} T \approx \SI{1e4}{\electronvolt}$.

The collective behavior arising from the interactions of the moving charged particles mediated by electric and magnetic fields becomes quite complex -- no single theory can describe all plasma phenomena and is computationally tractable at the same time. The present thesis considers the theory of ideal \emph{magnetohydrodynamics} (MHD), which is useful to determine the macroscopic stability of a plasma. To see how it relates to other models, we sketch its derivation via \emph{kinetic theory}. In the non-relativistic limit, \textsc{Newton}'s equations of motion describe the single-particle motion under the influence of the \textsc{Lorentz} force originating from external fields generated by other particles in the plasma. This picture already allows a description of particle drifts and confinement in toroidal geometry by consideration of their trajectories for a given configuration of magnetic and electric fields. While it is theoretically possible to set up the system of equations to fully describe the interaction of all particles among each other and with external fields, this approach is neither computationally feasible nor desirable -- knowing the exact positions and velocities of each particle is not helpful in understanding their large-scale collective behavior. Instead, we opt for the same level of abstraction as in statistical mechanics: We take ensemble averages of the electric and magnetic fields and describe the particles' positions and velocities by a \emph{distribution function} $f_{s} (\vec{r}, \vec{v}, t)$. Here, $\vec{r}$ is position, $\vec{v}$ is velocity, and $t$ is time, while $s$ designates the particle species (ions and electrons). The distribution function is normalized to
\begin{gather}
  \int_{\Omega} \diff^{3} r \int_{\symbb{R}^{3}} \diff^{3} v \, f_{s} (\vec{r}, \vec{v}, t) = 1
\end{gather}
for any given time $t$. The time evolution of $f_{s}$ is given by the kinetic equation
\begin{gather}
  \pd[f_{s} (\vec{r}, \vec{v}, t)]{t} + \vec{v} \cdot \grad_{\vec{r}} f_{s} (\vec{r}, \vec{v}, t) + \frac{q_{s}}{m_{s}} \left ( \vec{E} + \frac{1}{c} \vec{v} \times \vec{B} \right ) \cdot \grad_{\vec{v}} f_{s} (\vec{r}, \vec{v}, t) = \left ( \pd[f_{s} (\vec{r}, \vec{v}, t)]{t} \right )_{\text{coll}},
\end{gather}
where the term on the right-hand side is the \emph{collision term} modelling \textsc{Coulomb} collisions. $q_{s}$ and $m_{s}$ are the charge and mass of the particle species $s$, respectively. The plasma kinetic equations forms the basis for kinetic theory which is able to account for a wide array of particle effects within the plasma. To derive the fluid description of magnetohydrodynamics, we have to take moments of the distribution function, i.e., we integrate over velocity space with appropriate weights.

\subsection{Ideal Magnetohydrodynamics}
\label{sec:ideal-mhd}

With the total number $n_{0 s}$ of particles of species $s$, we derive the number density as a zeroth moment:
\begin{gather}
  n_{s} (\vec{r}, t) = n_{0 s} \int_{\symbb{R}^{3}} f_{s} (\vec{r}, \vec{v}, t) \, \diff^{3} v.
\end{gather}
The charge density and mass density simply follow as $\varrho_{m s} = m_{s} n_{s}$ and $\varrho_{q s} = q_{s} n_{s}$, respectively. As a first moment, we get the particle flux density
\begin{gather}
  \vec{\Gamma}_{s} (\vec{r}, t) = n_{0 s} \int_{\symbb{R}^{3}} f_{s} (\vec{r}, \vec{v}, t) \vec{v} \, \diff^{3} v,
\end{gather}
from which the particle velocity
\begin{gather}
  \vec{V}_{s} (\vec{r}, t) = \frac{\vec{\Gamma}_{s} (\vec{r}, t)}{n_{s} (\vec{r}, t)}
\end{gather}
and current density
\begin{gather}
  \vec{J}_{s} (\vec{r}, t) = q_{s} \vec{\Gamma}_{s} = q_{s} n_{s} \vec{V}_{s}
\end{gather}
follow. Taking the second moment gives the pressure tensor as
\begin{gather}
  \symsf{p}_{s} (\vec{r}, t) = n_{0 s} m_{s} \int_{\symbb{R}^{3}} f_{s} (\vec{r}, \vec{v}, t) (\vec{v} - \vec{V}_{s}) (\vec{v} - \vec{V}_{s}) \, \diff^{3} v = n_{0 s} m_{s} \int_{\symbb{R}^{3}} f_{s} (\vec{r}, \vec{v}, t) \vec{v} \vec{v} \, \diff^{3} v - m_{s} n_{s} \vec{V}_{s} \vec{V}_{s},
\end{gather}
which reduces to $\symsf{p}_{s} = p_{s} \symsf{I}$ in an isotropic medium, where $\symsf{I}$ is the unit tensor. The latter is assumed when the plasma is dominated by collisions so that the distribution function takes the form of the \textsc{Maxwell} distribution in velocity space. While most plasmas of interest are not in thermodynamic equilibrium, the approximation is good enough for our needs and ideal MHD also gives reasonable results when the assumption does not apply. Now, taking moments of the kinetic equation yields the time evolution of the fluid quantities derived above. Taking the zeroth moment gives
\begin{gather}
  \pd{t} \int_{\symbb{R}^{3}} f_{s} \, \diff^{3} v + \nabla_{\vec{r}} \cdot \int_{\symbb{R}^{3}} \vec{v} f_{s} \, \diff^{3} v + \frac{q_{s}}{m_{s}} \int_{\symbb{R}^{3}} \nabla_{\vec{v}} \cdot \left ( \vec{E} + \frac{1}{c} \vec{v} \times \vec{B} \right ) f_{s} \, \diff^{3} v = \int_{\symbb{R}^{3}} \left ( \pd[f_{s}]{t} \right )_{\text{coll}} \, \diff^{3} v.
\end{gather}
The third integral term on the left-hand side vanishes after application of the divergence theorem because the distribution function vanishes at infinite velocity. Likewise, the integral of the collision term vanishes due to particle conservation. Expressing the remaining two integrals by fluid quantities yields
\begin{gather}
  \pd[n_{s}]{t} + \divg (n_{s} \vec{V}_{s}) = 0, \label{eq:continuity}
\end{gather}
which we identify as continuity equation. Taking the first moment results in
\begin{multline}
  \pd{t} \int_{\symbb{R}^{3}} \vec{v} f_{s} \, \diff^{3} v + \nabla_{\vec{r}} \cdot \int_{\symbb{R}^{3}} \vec{v} \vec{v} f_{s} \, \diff^{3} v + \frac{q_{s}}{m_{s}} \int_{\symbb{R}^{3}} \vec{v} \nabla_{\vec{v}} \cdot \left ( \vec{E} + \frac{1}{c} \vec{v} \times \vec{B} \right ) f_{s} \, \diff^{3} v = \\ = \int_{\symbb{R}^{3}} \vec{v} \left ( \pd[f_{s}]{t} \right )_{\text{coll}} \, \diff^{3} v = \vec{R}_{s}.
\end{multline}
Here, $\vec{R}_{s}$ is constrained by $\sum_{s} \vec{R}_{s} = \vec{0}$. Using integration by parts for the third integral on the left-hand side and some more rearrangement, we get the momentum equation
\begin{gather}
  m_{s} n_{s} \left ( \pd[\vec{V}_{s}]{t} + \vec{V}_{s} \cdot \grad \vec{V}_{s} \right ) + \divg \symsf{p}_{s} - q_{s} n_{s} \left ( \vec{E} + \frac{1}{c} \vec{V}_{s} \times \vec{B} \right ) = \vec{R}_{s}. \label{eq:momentum}
\end{gather}
Now, the continuity equation connects a zeroth moment ($n_{s}$) to a first moment ($\vec{V}_{s}$), the momentum equation connects both to a second moment ($\symsf{p}_{s}$), and so on. This hierarchy of equations has to be truncated at some point to give a closed set of equations, which will be discussed later. To complete the picture of two-fluid MHD, we have to consider \textsc{Maxwell}'s equations:
\begin{align}
  \divg \vec{E} &= 4 \pi \varrho_{q} = 4 \pi e (n_{i} - n_{e}), \label{eq:gauss_field} \\
  \divg \vec{B} &= 0, \label{eq:no_monopoles_field} \\
  \curl \vec{E} &= -\frac{1}{c} \pd[\vec{B}]{t}, \label{eq:faraday_field} \\
  \curl \vec{B} &= \frac{4 \pi}{c} \vec{J} + \frac{1}{c} \pd[\vec{E}]{t} = \frac{4 \pi}{c} e (n_{i} \vec{V}_{i} - n_{e} \vec{V}_{e}) + \frac{1}{c} \pd[\vec{E}]{t}. \label{eq:ampere_field}
\end{align}
Here, we already used fluid quantities with indices $i$ and $e$ for ions (with an implicit atomic number of $Z = 1$) and electrons, respectively. Now, we can make some simplifications. First, quasineutrality implies $n_{e} = n_{i} = n$, so the right-hand side of \cref{eq:gauss_field} vanishes. Second, on the time scale considered in MHD, the fields are assumed to be static -- such simulations typically recalculate fields from kinetic theory after a given time increment --, so we can neglect the time derivatives in \textsc{Maxwell}'s equations. As a consequence, the current density is also divergence-free: $\divg \vec{J} = 0$. Third, we reduce the two-fluid picture to a single-fluid picture:
\begin{align}
  \varrho_{m} &= m_{e} n_{e} + m_{i} n_{i}, \\
  \vec{v} &= \frac{m_{e} \vec{V}_{e} + m_{i} \vec{V}_{i}}{m_{e} + m_{i}}, \\
  \vec{J} &= \vec{J}_{e} + \vec{J}_{i} \\
  T &= \frac{T_{e} + T_{i}}{2}, \\
  p &= p_{e} + p_{i} = 2 n T.
\end{align}
This also leads to a version of \textsc{Ohm}'s law,
\begin{gather}
  \vec{E} + \frac{1}{c} \vec{v} \times \vec{B} = \vec{0}.
\end{gather}
Fourth, we close the moment equations by the adiabatic condition
\begin{gather}
  \frac{\diff}{\diff t} \frac{p}{\varrho_{m}^{\gamma}} = 0,
\end{gather}
where $\gamma$ is the adiabatic index. For a monoatomic gas, as is considered in fusion experiments, we have $\gamma = \frac{5}{3}$. Fifth, assuming electrons redistribute much faster than the timescale considered, no essential electric fields can build up in the plasma, and resistivity is effectively zero\footnote{This is where the designation \emph{ideal} comes from -- non-ideal MHD considers effects caused by finite resistivity.}. When we furthermore neglect the inertial term (the first term in \cref{eq:momentum}), i.e., we assume a static solution, we finally arrive at the ideal MHD \emph{force balance} equation:
\begin{gather}
  c \grad p = \vec{J} \times \vec{B}. \label{eq:ideal-mhd}
\end{gather}
The interpretation is that the force resulting from the thermodynamic pressure gradient is balanced by the \textsc{Lorentz} force. This equation is the basis of the calculations in \cref{sec:linmhd}.

\subsection{Flux Surfaces}

Scalar multiplication of \cref{eq:ideal-mhd} with the magnetic field or current density yields the important relations
\begin{gather}
  \vec{B} \cdot \grad p = \vec{J} \cdot \grad p = 0,
\end{gather}
which means that the current density and magnetic field lines lie on surfaces of constant pressure. They are in general not parallel to each other, but neither do they have a component perpendicular to these surfaces. By casting the magnetic field in \textsc{Clebsch} form given by \cref{eq:B_pol}, we see that $\vec{B} \cdot \grad \psi = 0$, so they are also surfaces of constant magnetic flux\footnote{For a detailed discussion, see \cref{sec:cocos}.} $\psi$, which is why they are commonly called \emph{flux surfaces}. Any quantity that is constant on a flux surface is called a \emph{flux surface quantity} accordingly. Application of Hamiltonian theory to the differential equation determining magnetic field lines shows that \emph{nested} flux surfaces appear in systems with axisymmetry, such as tokamaks. In the poloidal plane, these curves are described by the \textsc{Grad}--\textsc{Shafranov} equation
\begin{gather}
  \Delta^{*} \psi = -B_{\phi} \frac{\diff B_{\phi}}{\diff \psi} - R^{2} \frac{\diff p}{\diff \psi}, \label{eq:grad-shafranov}
\end{gather}
where the differential operator $\Delta^{*}$ is defined as
\begin{gather}
  \Delta^{*} \psi = R \pd{R} \left ( \frac{1}{R} \pd[\psi]{R} \right ) + \frac{\partial^{2} \psi}{\partial R^{2}},
\end{gather}
making \cref{eq:grad-shafranov} a nonlinear partial differential equation of second order. Note that due to the \textsc{Shafranov} shift, the nested flux surfaces are not concentric, but the \emph{magnetic axis} -- the innermost flux surface, degenerated to a point in the poloidal plane -- is shifted outwards from the geometric center of the torus. Furthermore, usual tokamak configurations also have an \emph{X point}\footnote{The magnetic axis is also called \emph{O point} by analogy.}, where the magnetic field lines cross, so that the flux surface is not closed. This flux surface is called the \emph{separatrix}, and in our calculations, we only consider the plasma volume contained within the separatrix. We also speak of the \emph{last closed flux surface} (LCFS) because we count the nested flux surfaces starting from the magnetic axis, going outward.

Now we shall consider an important flux surface quantity, the \emph{safety factor} $q$. For every toroidal transit ($\symup{\Delta} \phi = 2 \pi$), the magnetic field line traverses the poloidal angle $\symup{\Delta} \theta$. The latter is averaged over a number of toroidal transits $k$ to give the average rotational transform $\iota$ (lowercase iota):
\begin{gather}
  \iota = \lim_{N \to \infty} \frac{1}{N} \sum_{k = 1}^{N} \symup{\Delta} \theta_{k}.
\end{gather}
The safety factor $q$ is then defined by
\begin{gather}
  q = \frac{2 \pi}{\iota}
\end{gather}
When the magnetic field line closes on itself after a finite number of toroidal transits, $q$ is rational and the flux surface on which it attains that value is called a rational flux surface. In this case, more than one magnetic field line is necessary to trace out the flux surface. If $q$ is irrational, the magnetic field line never closes on itself and will trace out the entire flux surface (given infinite time and ergodicity). Since the rational numbers are a dense subset of the real numbers, an irrational flux surface may be approximated arbitrarily close by a rational flux surface and vice versa. The actual values of $q$ are relevant to the macroscopic stability of the plasma: $q$ has to be larger than one over the entire plasma volume, and it has to be larger than two on the plasma edge. Otherwise, kink instabilities will appear. This shows that despite all the simplifications and assumptions, ideal MHD is useful to estimate macroscopic stability of the plasma.

Of particular concerns are the \emph{edge-localized modes} (ELMs) that can appear in high-confinement mode or \emph{H-mode}, the most common mode of tokamak operation. The radial pressure profile will steepen, and when it becomes too steep, the pressure gradient force ejects particles which are then lost from confinement and can damage the reactor wall. One approach to handle these instabilities is to utilize ELM mitigation coils which induce \emph{resonant magnetic perturbations} (RMPs). These are non-axisymmetric perturbations to an otherwise axisymmetric MHD equilibrium configuration. A resonance appears at a flux surface with rational $q$ corresponding to the ratio of the poloidal and toroidal mode numbers of the perturbation; for details see \cref{sec:nonres}. At the resonance, \emph{tearing modes} and chains of narrow \emph{magnetic islands} appear, across which particle transport is enhanced. This avoids the build-up of pressure and could be used to control ELMs. Ideal MHD cannot describe tearing modes and magnetic islands, but the inclusion of \emph{sheet currents} from tearing mode theory, which we touch upon in \cref{sec:sheets}, alleviates this shortcoming somewhat.

%%% Local Variables: 
%%% mode: latex
%%% TeX-master: "../magdif"
%%% End: 
