\section{Construction of Test Cases}

Along with comparison to results from kinetic computations, special test cases altering different components of the calculation are considered. In \cref{sec:nonres}, $\Bvac$ is not taken from \textsc{Biot}--\textsc{Savart} calculations of currents in external coils as per \cref{eq:biot-savart,sec:inputs}, but constructed directly to avoid resonances, at least in the first iteration step. \Cref{sec:analytical} instead looks at the analytical solution in the cylindrical limit on a circular cross-section obtained from KiLCA, effectively changing the grid geometry.

\subsection{Generating a Non-Resonant Vacuum Perturbation}
\label{sec:nonres}

The axisymmetric equilibrium field $\vec{B}_{0}$ lies on nested flux surfaces $\psi = \text{const.}$, meaning
\begin{gather*}
  \vec{B}_{0} \cdot \grad \psi = B_{0}^{\psi} = 0.
\end{gather*}
If the perturbed field shall still lie on distorted, but not broken, flux surfaces (non-resonant, without magnetic islands), a new flux surface label $\psi + \delta \psi$ must exist fulfilling
\begin{gather*}
  (\vec{B}_{0} + \Bpert) \cdot \grad (\psi + \delta \psi) = 0
\end{gather*}
for the perturbed magnetic field $\vec{B}_{0} + \Bpert$, where $\delta \psi$ remains small and continuous within the plasma. In that case we can use a linear order expansion
\begin{gather}
  \underbrace{\vec{B}_{0} \cdot \grad \psi}_{= 0} + \vec{B}_{0} \cdot \grad \delta \psi + \Bpert \cdot \grad \psi + \mathcal{O}(\delta^{2}) = 0,
\end{gather}
or in coordinate form with symmetry flux coordinates $(\rho, \phi, \theta)$ outlined in \cref{sec:cocos},
\begin{gather}
  B_{0}^{\theta} \pd{\theta} \delta \psi + B_{0}^{\phi} \pd{\phi} \delta \psi + \delta B^{\rho} \psi' (\rho) = 0.
\end{gather}
Here we have used $\grad \psi = \psi' (\rho) \grad \rho$. Further, $\psi' (\rho) = -1$. With safety factor
\begin{gather*}
  q = \frac{B_{0}^{\phi}}{B_{0}^{\theta}}
\end{gather*}
and dividing by $B_{0}^{\theta}$, we obtain
\begin{gather}
  \left ( \pd{\theta} + q \pd{\phi} \right ) \delta \psi = \frac{\delta B^{\rho}}{B_{0}^{\theta}} = -\sqrt{g} \delta B^{\rho} = \sqrt{g} \delta B^{\psi}.
\end{gather}
Written in terms of toroidal \textsc{Fourier} harmonics $m, n$ in $\theta, \phi$ the equation becomes
\begin{gather*}
  \psi_{m n} = \frac{\left [ \sqrt{g} \delta B^{\psi} \right ]_{m n}}{\im (m + n q)} = \frac{\left [ \sqrt{g} B_{n}^{\psi} \right ]_{m}}{\im (m + n q)}.
\end{gather*}
$\sqrt{g}$ depends also on $\theta$, so poloidal harmonics $m$ have to be taken outside the bracket here. In order to fulfill the original requirement not to break flux surfaces, $\psi_{m n}$ must never diverge, so $m + n q$ must not become zero. We can avoid such resonant surfaces if
\begin{gather}
  \left [ \sqrt{g} B_{n}^{\psi} \right ]_{m} \overset{!}{=} 0
\end{gather}
for all possible $m$ that could lead to a resonance. The simplest way to do so is to make $\sqrt{g} B_{n}^{\psi}$ a flux function, possessing only the poloidally symmetric harmonic $m = 0$. More generally, also $-m < n q_{\text{min}}$ and $-m > n q_{\text{max}}$ are possible, but for nonzero $m$ a transformation to the flux angle $\theta$ has to be computed explicitly.

Using the Jacobian of symmetry flux coordinates from \cref{eq:flux_metric}, we obtain
\begin{gather}
  \psi_{m n} = \frac{q}{\im (m + n q)} \left [ \frac{R^{2}}{B_{0 \phi}} B_{n}^{\psi} \right ]_{m} \label{eq:psi_mn}
\end{gather}
and so, to generate a non-resonant field, we are allowed to use any value
\begin{gather}
  B_{n}^{\psi} = C(\psi) \frac{B_{0 \phi}}{R^{2}}, \label{eq:Bnpsi_vac}
\end{gather}
where $C(\psi)$ is a flux function. The other components of $\Bpert$ are not relevant for the resonance condition and just need to fulfill divergence-freeness. Thus we proceed as in \cref{sec:compute_currn} and find fluxes through triangle edges:
\begin{gather}
  \int_{\Gamma_{\inw}, \Gamma_{\out}} R \vec{B}_{n}^{\pol} \cdot \vec{n} \, \diff l = -\int_{\Gamma_{\fs}} R \vec{B}_{n}^{\pol} \cdot \vec{n} \, \diff l - \im n \int_{\Omega} R B_{n}^{\phi} \, \diff R \, \diff Z. \label{eq:Bn_divfree}
\end{gather}
As for currents we use the notation $\Psi_{k}$ for weighted magnetic fluxes through edges as in \cref{eq:Psi_k},
\begin{gather}
  \Psi_{k} = \int_{\Gamma_{k}} R \vec{B}_{n}^{\pol} \cdot \vec{n} \, \diff l \approx R (\Gamma_{k}) \vec{B}_{n}^{\pol} (\Gamma_{k}) \cdot \vec{n}_{k},
\end{gather}
and, differing from the procedure in \cref{sec:compute_currn}, the same for the weighted toroidal magnetic flux $\Psi_{\phi}$ as per \cref{eq:Psi_phi},
\begin{gather}
  \Psi_{\phi} = \int_{\Omega} R B_{n}^{\phi} \, \diff R \, \diff Z \approx S_{\Omega} \, R (\Omega) B_{n}^{\phi} (\Omega),
\end{gather}
resulting in a shortened notation for \cref{eq:Bn_divfree} where the system of equations to be assembled is more apparent:
\begin{gather}
  \Psi_{\inw} + \Psi_{\out} = -\Psi_{\fs} - \im n \Psi_{\phi}.
\end{gather}
Again, the flux through edge \fs\ is fixed, so we rearrange \cref{eq:Bnpsi} and get
\begin{gather}
  \Psi_{\fs} \approx 2 R (\Gamma_{\fs}) B_{n}^{\psi} (\Gamma_{\fs}) \frac{S_{\Omega^{(+1)}} + S_{\Omega^{(-1)}}}{\psi^{(+1)} - \psi^{(-1)}}, \label{eq:Psi_f}
\end{gather}
where $B_{n}^{\psi}$ is given by \cref{eq:Bnpsi_vac}.

The system of linear equations to solve is
\begin{gather}
  K_{jk} \Psi^{(k)} = s_{j} \quad \forall j, k = 1, 2, \dotsc, N,
\end{gather}
with
\begin{gather}
  K_{jk} = \delta_{j-1, k} - \delta_{jk} \rightarrow \hat{K} = \begin{pmatrix}
    -1   &    1   &    0   & \hdots &    0   \\
     0   &   -1   &    1   & \hdots &    0   \\
     0   &    0   &   -1   & \hdots &    0   \\
  \vdots & \vdots & \vdots & \ddots & \vdots \\
     1   &    0   &    0   & \hdots &   -1
  \end{pmatrix}
\end{gather}
and
\begin{gather}
  s_{j} = -\Psi_{\fs}^{(j)} - \im n \Psi_{\phi}^{(j)}.
\end{gather}
Since any one column in $\hat{K}$ is a linear combination of all other columns, $\hat{K}$ is of rank $N - 1$ and thus singular. To construct a solution, we first consider the homogenous case with $\vec{s} = \vec{0}$, which also determines the toroidal flux. The non-zero solution for the remaining degrees of freedom then assumes the simple form
\begin{gather}
  \Psi^{(k)} = C_{0} \quad \forall k = 1, 2, \dotsc, N,
\end{gather}
with an arbitrary constant $C_{0}$. With local indices, the degrees of freedom are written as 
\begin{align}
  \Psi_{\inw}^{(k)} &= -C_{0}, & \Psi_{\out}^{(k)} &= C_{0} & \Psi_{\phi}^{(k)} &= \frac{\im}{n} \Psi_{\fs}^{(k)} \quad \forall k = 1, 2, \dotsc, N. \label{eq:Psi_hom}
\end{align}
This means that flux conservation, i.e.\ divergence-freeness, is fulfilled by balancing $\Psi_{\phi}$ with $\Psi_{\fs}$ on each individual triangle and assigning a constant flux $C_{0}$ in poloidal direction to $\Psi_{\inw}$ and $\Psi_{\out}$ over the entire triangle strip.

A consistent solution according to the \textsc{Rouché}-\textsc{Capelli} theorem can also be constructed for the inhomogeneous case, i.e.\ non-zero $\vec{s}$, as long as $\vec{s}$ is a linear combination of columns of $K$. Starting from the trivial (zero) solution to the homogeneous case and for a fixed $k$, we add to $\vec{s}$ the $(k+1)$-th column of $\hat{K}$, multiplied by another arbitrary constant $C_{k}$, resulting in
\begin{align}
  s_{k} &= -\Psi_{\fs}^{(k)} - \im n \Psi_{\phi}^{(k)} = C_{k}, \\
  s_{k+1} &= -\Psi_{\fs}^{(k+1)} - \im n \Psi_{\phi}^{(k+1)} = -C_{k}.
\end{align}
Considering the two equations affected compared to the homogeneous case,
\begin{align}
  -\Psi^{(k-1)} + \Psi^{(k)} &= C_{k}, \\
  -\Psi^{(k)} + \Psi^{(k+1)} &= -C_{k},
\end{align}
we arrive at a particular solution
\begin{gather}
  \Psi^{(k)} = C_{k}, \quad \Psi^{(j)} = 0 \quad \forall j \neq k.
\end{gather}
The degrees of freedom differing from the solution to the zero solution (\cref{eq:Psi_hom} with $C_{0} = 0$) are, given with local indices:
\begin{align}
  \Psi_{\inw}^{(k)} &= 0, & \Psi_{\out}^{(k)} &= C_{k}, & \Psi_{\phi}^{(k)} &= \frac{\im}{n} \Psi_{\fs}^{(k)} - \frac{\im}{n} C_{k}, \\
  \Psi_{\inw}^{(k+1)} &= -C_{k}, & \Psi_{\out}^{(k+1)} &= 0, & \Psi_{\phi}^{(k+1)} &= \frac{\im}{n} \Psi_{\fs}^{(k+1)} + \frac{\im}{n} C_{k}.
\end{align}
This means that, compared to the zero solution, a change $C_{k}$ in the flux across the associated edge of adjacent triangles in one triangle strip is accomodated by a change in the toroidal flux of these triangles.

The approach outlined above can be repeated for different values of $k$. Linear superposition of the solutions to the homogeneous and inhomogeneous cases then yields the most general solution,
\begin{align}
  \Psi_{\inw}^{(k)} &= -C_{0} - C_{k-1}, & \Psi_{\out}^{(k)} &= C_{0} + C_{k}, & \Psi_{\phi}^{(k)} &= \frac{\im}{n} \Psi_{\fs}^{(k)} + \frac{\im}{n} C_{k-1} - \frac{\im}{n} C_{k} \quad \forall k = 1, 2, \dotsc, N.
\end{align}
Now, to assign sensible values to the arbitrary constants, consider the solution to the homogeneous sytem of equations. Since edge \fs\ alternates between inner and outer flux surface for all but the innermost triangle strip, $\vec{n}_{\fs}$ will also alternate between pointing inwards and outwards. Thus $\Psi_{\fs}$ will alternate signs too, but it is consistent along one flux surface. Since $\Psi_{\phi}$ depends linearly on $\Psi_{\fs}$, on any one triangle the sign of $B_{n (\phi)}$ will differ from all three adjacent triangles, except for the innermost triangle strip. The resulting field is then clearly dependent on the choice of the grid. The procedure described before can be used to alleviate this problem. On every pair of triangles $\Omega^{(2k)}$ and $\Omega^{(2k+1)}$ -- which originally result from diagonally dividing a more regular quadrilateral -- we average $\Psi_{\phi}$ and set $C_{2k}$ to the deviation to counterbalance the change:
\begin{align}
  C_{2k} &= \frac{\im}{2 n} \left( \Psi_{\fs}^{(2k)} - \Psi_{\fs}^{(2k+1)} \right) \quad \forall k = 0, 1, \dotsc, \frac{N}{2} - 1.
\end{align}

\subsection{Analytical Solution for Very Large Aspect Ratios}
\label{sec:analytical}

To derive an analytical approximation, a very large major radius $R_{0}$ -- and thus aspect ratio -- is considered for a ring torus, i.e. with circular cross-section. The torus is \enquote{unbent} and the toroidal geometry is effectively transformed to cylindrical geometry with a very thin and elongated cylinder and periodic boundary conditions. To distinguish this cylinder from the cylindrical coordinates $(R, \phi, Z)$ established in \cref{sec:cocos}, we use coordinates $(r, \vartheta, z)$ for the former:
\begin{align}
  R &= R_{0} + r \cos \vartheta, & Z &= r \sin \vartheta, & \phi &= \frac{z}{R_0}. \label{eq:kilca-coord}
\end{align}
Thus $r$ corresponds to the minor radius and the pseudoradial component $\rho$, $z$ points in the $\phi$ direction in the infinitesimal limit and $\vartheta$ corresponds to the poloidal angle $\theta$, albeit defined as a geometrical angle and not field-aligned. As for toroidal geometry, a \textsc{Fourier} series expansion is possible for periodic coordinates:
\begin{gather}
  \Bpert (r, \vartheta, z) = \sum_{n = -\infty}^{\infty} \sum_{m = -\infty}^{\infty} \vec{B}_{m n} (r) \e^{\im m \vartheta} \e^{\im k_{z} z}.
\end{gather}
Here, $k_{z}$ is a wavenumber given by
\begin{gather}
  k_{z} = \frac{n}{R_{0}},
\end{gather}
ensuring the same same periodicity $2 \pi R_{0}$ as for the toroidal geometry. In the same sense, the wavenumber $k_{\vartheta}$ is defined as
\begin{gather}
  k_{\vartheta} = \frac{m}{r}
\end{gather}
and both are combined in the wavevector $\vec{k}$ as
\begin{align}
  \vec{k} = k_{z} \hat{\vec{e}}_{z} + k_{\vartheta} \hat{\vec{e}}_{\vartheta}.
\end{align}
Note that, due to the cylindrical geometry, there is no mode coupling among poloidal modes. As before, we only consider a single toroidal mode number $n$.

The general solution for the radial component in this geometry, on which KiLCA code is based, is given by eq.~(5) in the paper by \textcite{Heyn08}. For the vacuum perturbation $\Bvac$ it reduces to
\begin{gather}
  \pd{r} \frac{r}{k^{2}} \pd[B_{m n r}]{r} - r B_{m n r} \left ( 1 - \pd{r} \frac{1}{r k^{2}}\right ) = 0 \label{eq:kilca-ode}
\end{gather}
Before attempting to solve \cref{eq:kilca-ode}, we consider the constraints on the remaining components. Applying \textsc{Ampère}'s law and taking into account the lack of associated current in the plasma volume, we get
\begin{gather}
  0 = \frac{4 \pi}{c} \sqrt{g} J^{\vartheta} = \sqrt{g} \left ( \curl \Bvac \right )^{\vartheta} = \pd{z} \delta B_{\text{v} r} - \pd{r} \delta B_{\text{v} z}.
\end{gather}
Taking the \textsc{Fourier} transform as $\pd{z} \to \im k_{z}$ yields
\begin{gather}
  0 = \im k_{z} B_{m n r} - \pd{r} B_{m n z}. \label{eq:kilca-curlB}
\end{gather}
The condition of zero divergence then gives a relation for the third component $\delta B_{\text{v}}^{\vartheta}$:
\begin{gather}
  0 = \sqrt{g} \divg \Bvac = \pd{r} (r \delta B_{\text{v}}^{r}) + r \pd{\vartheta} \delta B_{\text{v}}^{\vartheta} + r \pd{z} \delta B_{\text{v}}^{z}.
\end{gather}
Applying the \textsc{Fourier} again with $\pd{\vartheta} \to \im m$ yields
\begin{gather}
  0 = \pd{r} (r B_{m n r}) + \im m B_{m n}^{\vartheta} + \im k_{z} r B_{m n z}. \label{eq:kilca-divB}
\end{gather}
For problems in cylindrical geometry, the radial solution typically involves \textsc{Bessel} functions. We start with
\begin{gather}
  B_{m n z} = \im C I_m(k_{z} r), \label{eq:kilca-Bz}
\end{gather}
where $I_m$ is the modified \textsc{Bessel} function of the first kind of order $m$ and $C$ is an arbitrary integration constant that scales the magnitude of the solution. Inserting this into \cref{eq:kilca-curlB}, we get at
\begin{gather}
  B_{m n r} = -\frac{\im}{k_{z}} \pd{r} B_{m n z} = \frac{C}{k_{z}} \pd{r} I_m(k_{z} r) = C I_m'(k_{z} r), \label{eq:kilca-Br}
\end{gather}
where the derivative of $I_{m}$ is taken with respect to the entire argument, not only $r$. By insertion\footnote{into Mathematica}, one can check that \cref{eq:kilca-Br} solves \cref{eq:kilca-ode}, so we proceed to calculate $\delta B_{\text{v}}^{\vartheta}$ from \cref{eq:kilca-divB}. Using functional identities of the \textsc{Bessel} functions\footnote{again, Mathematica}, one arrives at
\begin{gather}
  B_{m n \vartheta} = \im \frac{m}{k_{z}} C I_m(k_{z} r). \label{eq:kilca-Btheta}
\end{gather}
Since $I_{-m}(k_{z} r) = I_{m} (k_{z} r)$, this is the only vector component that changes with a negative mode number $m$. Even though there is no mode coupling in this geometry, our calculations are done in toroidal geometry, hence mode coupling is expected; thus we restrict ourselves to a single poloidal mode number $m$ and take the \textsc{Fourier} transform, i.e.
\begin{gather}
  \vec{B}_{n} (r, \theta) = \vec{B}_{m n} (r) \e^{\im m \vartheta}.
\end{gather}
To convert back to $(R, \phi, Z)$ vector components, we use \cref{eq:kilca-coord} and arrive at
\begin{align}
  B_{n}^{R} &= B_{n}^{r} \pd[R]{r} + B_{n}^{\vartheta} \pd[R]{\vartheta} = B_{n}^{r} \cos \vartheta - r B_{n}^{\vartheta} \sin \vartheta, \label{eq:kilca-BR} \\
  B_{n}^{Z} &= B_{n}^{r} \pd[Z]{r} + B_{n}^{\vartheta} \pd[Z]{\vartheta} = B_{n}^{r} \sin \vartheta + r B_{n}^{\vartheta} \cos \vartheta, \label{eq:kilca-BZ} \\
  B_{n}^{\phi} &= B_{n}^{z} \pd[\phi]{z} = \frac{1}{R_{0}} B_{n}^{z}. \label{eq:kilca-Bphi}
\end{align}
Note that the same conversions apply in the secondary cylindrical coordinate system analogously, e.g. $r B_{n}^{\vartheta} = B_{n (\vartheta)}$.

Up to this point, calculations have been performed in the cylindrical geometry used by KiLCA code where the torus is explicitly bent into a cylinder. To approximate this geometry in NEO-EQ code, we multiply $R_0$ by a scaling factor \gls{scaling} and implement the change by shifting the computational domain in the EFIT file (see \cref{sec:inputs}) to that value. This way, $R \approx R_{0}$ over the computational domain and the influence of the toroidal geometry is comparatively small. Furthermore, since the toroidal wavenumber $k_{z}$ -- inversely proportional to the wavelength of the standing wave imagined to go around in toroidal direction -- should stay constant, the toroidal mode number $n$ must also be scaled by $\gamma$, which should therefore be an integer. In summary we have
\begin{align}
  R_{0} \rvert_{\text{KiLCA}} &\to R_{0} \rvert_{\text{NEO-EQ}} = \gamma R_{0} \rvert_{\text{KiLCA}}, \\
  n \rvert_{\text{KiLCA}} &\to n \rvert_{\text{NEO-EQ}} = \gamma n \rvert_{\text{KiLCA}}.
\end{align}
The solution computed by NEO-EQ code then has to be transformed to the cylindrical geometry of KiLCA code to allow for a comparison. Using trigonometric identities, we can invert \cref{eq:kilca-BR,eq:kilca-BZ,eq:kilca-Bphi} to yield
\begin{align}
  B_{n}^{r} &= B_{n}^{R} \cos \vartheta + B_{n}^{Z} \sin \vartheta, \\
  r B_{n}^{\vartheta} &= B_{n}^{Z} \cos \vartheta - B_{n}^{R} \sin \vartheta, \\
  B_{n}^{z} &= R_{0} B_{n}^{\phi}.
\end{align}
The poloidal spectrum of this result is then approximated by a discrete \textsc{Fourier} transform:
\begin{gather}
  \vec{B}_{m n} (r) = \frac{1}{N} \sum_{k = 1}^{N} \vec{B}_{n} (r, \vartheta_{k}) \e^{-\im m \vartheta_{k}}, \quad \vartheta_{k} = \frac{2 \pi k}{N}.
\end{gather}
Here, $N$ is the number of sample points, which is taken to be one per triangle in the triangle strip at given $r$.

%%% Local Variables: 
%%% mode: latex
%%% TeX-master: "../magdif"
%%% End: 
